\documentclass[a4paper,twoside,10pt]{article}

\usepackage[left=2.5cm,right=2.5cm,top=0.5cm,bottom=2.5cm]{geometry}
\usepackage{graphicx,type1cm,eso-pic,color}
\usepackage{url}
\usepackage[latin1]{inputenc}
\usepackage[T1]{fontenc}
% Extension postscript
\usepackage{graphicx}
\usepackage{color}
\usepackage{amsmath}
\usepackage{amsfonts}
\usepackage[english,french]{babel}
\usepackage[toc,page]{appendix} 
\usepackage{palatino}
\usepackage{color}
\usepackage{listings}
\usepackage{ifthen}
\usepackage{tabularx}
\usepackage{array}
\usepackage{subfigure}
\usepackage{rotating}
\bibliographystyle{unsrt}
\usepackage{lastpage} 
\usepackage{multirow}
\usepackage{fancyhdr}
\newcommand{\topfigrule}{%
  \vspace*{5pt}\hrule\vspace{-5pt}}

\newcommand\musec{\mu\text{s}}

\title{Solveur multigrille pour le VDF}

\author{Benoit Mathieu}
\begin{document}
\selectlanguage{french}

\section{Mesures de performances}

Performances des diff�rentes versions du solveur sur diff�rentes machines.

\begin{table}

\centering
\begin{tabular}{c c c c c c}
\hline\hline
Machine & Nb coeurs & GCP & MG1 & MG2(ghost4)\\
\hline
Titane (Nehalem)
       & 1 & 42.5 & 23.1 \\
       & 2 & 40.1 & 12.2 & 6.1 \\
       & 4 & 37.9 & 6.48 & 3.45 \\
       & 8 & 23.6 & 3.23  & 1.77\\
\hline
Jade (Harpertown)
       & 1 & 92.3 & 23.7 \\
       & 2 & 101.1 & 12.4 \\
       & 4 & 53.0 & 6.34 \\
       & 8 & 54.3 & 3.6  \\
\hline
PC (Harpertown)
       & 1 &  &  & NON14.9 \\
       & 2 &  &  & NON7.9 \\
       & 4 &  &  & NON4.6 \\
       & 8 &  &  & NON3.2 \\


\end{tabular}
\small
\begin{tabular}{l l}
GCP & Gradient conjugu� de Trio\_U (SSOR, omega=1.5) \\
MG1 & Premi�re impl�mentation du multigrille (pas de cache-blocking, pas de SSE,
       double pr�cision, \\
    &  recalcul des coefficients de la matrice � chaque
       it�ration) \\
\end{tabular}
\caption{Performances des diff�rentes versions du solveur sur 
diff�rentes machines, cas 128x128x128 (2 millions de mailles).
Convergence �~$10^{-8}$.}
\end{table}

\end{document}