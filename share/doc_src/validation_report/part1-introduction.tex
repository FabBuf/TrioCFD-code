\lettrine[lines=2,slope=0pt,nindent=4pt]{\textbf{T}}{he} \texttt{TrioCFD}
database contains currently around 160 test cases of validation which
are called in this document ``validation sheets''. The present
report is the result of an inventory work of the most important of these validation
sheets to know (non exhaustive list): What type of flow do they simulate?
What is the degree of maturity of each validation? What sort of
comparison do they exhibit? and so on ... \smallskip\newline

This validation report has several objectives:
\begin{itemize}
\item Help users to identify application areas of \texttt{TrioCFD} code;
\item Give users examples of modeling (\texttt{TrioCFD} keywords and boundary
conditions) on specific cases;
\item Inform users of changes/improvements on the code validation;
\item Make an inventory of the code validation for each delivered version;
\item Account for physical and/or numerical impacts observed resulting from
implementation, corrections or modifications of test cases that have
been made between each version;
\item Update the validation status of the code.
\end{itemize}

In previous versions of \texttt{TrioCFD}, each validation sheet has
been written by different authors who used their own approach. In order
to improve the readability of this report, the content of all these validation
sheets has been harmonized by using identical titles and same general
organization. For this purpose, a new \texttt{PRM} template has been
updated since \texttt{TrioCFD v1.8.2}. The update has required
modifying the \texttt{Python} script and revising all \texttt{PRM}
files with the same content. More precisely, new tags have been added
and new keywords must be used by authors when writing the \texttt{PRM}
file. Details about this new \texttt{PRM} template and keywords are presented
in Part II. This new template will be helpful for future versions
of this document when a specific study using \texttt{TrioCFD} will
lead to write a new \texttt{PRM} sheet or to update an old one. The Part II
will also describe the methodology for making this report.\smallskip\newline

Moreover, until now, all validation sheets were placed in various
folders of the \texttt{TrioCFD} package. Hence, some of them could not
be found quickly by users who had to browse all directories. This
report gathers in one single document some of these sheets classified according
the type of flow or physics investigated. The selected
test cases are representative of five subdomains summarized in Table
\ref{tab:Type-of-flows} and successively presented for flows
of increasing complexity: \textquotedblleft Laminar flow\textquotedblright{}
(Part III), \textquotedblleft Thermal laminar flow\textquotedblright{}
(Part IV), \textquotedblleft Turbulent flow\textquotedblright{} (Part
V), \textquotedblleft Thermal turbulent flow\textquotedblright{} (Part
VI), \textquotedblleft Front Tracking\textquotedblright{} (Part
VII) and \textquotedblleft Arbitrary Lagrangian-Eulerian Method\textquotedblright{} (Part VIII).
The first four parts gather the test cases for single phase flow,
coupled or not with turbulence and thermal modeling. The
last part is dedicated to two-phase flows with interface tracking.
Finally Part IX concludes this report and perspectives will be sketched.
\newpage
\vspace{2cm}
\begin{table}[H]
\begin{centering}
\begin{tabular}{lll}
\hline 
\textbf{Parts} & $\qquad$ & \textbf{Type of flows}\tabularnewline
\hline 
Part III &  & Laminar Flow\tabularnewline
Part IV &  & Thermal Laminar Flow\tabularnewline
Part V &  & Turbulent Flow\tabularnewline
Part VI &  & Thermal Turbulent Flow\tabularnewline
Part VII &  & Two-phase Flows with Front Tracking\tabularnewline
Part VIII &  & ALE method for fluid/structure interactions\tabularnewline
\hline 
\end{tabular}
\par\end{centering}
\caption{\label{tab:Type-of-flows}Type of flow investigated to each Part.}
\end{table}

The sheets appearing in those parts are selected because they separately
simulate a particular flow well-known in the CFD literature,
such as the ``\emph{Poiseuille flow}'' and ``\emph{lid-driven cavity
flow}'' which are two examples of ``Laminar Flows''. For ``Thermal
Laminar Flows'' the ``\emph{Vahl Davis convection flow}'' is a
standard test case. Other sheets were selected because they compare
\texttt{TrioCFD} with other CFD results using alternative academic
or commercial codes (e.g. \texttt{Fluent} or other benchmarks) such
as ``\emph{OBI diffuser}'' (Turbulent flow). When available and
representative of flows, experimental data measured on facility tests
were included for comparisons and appear on the graphs (e.g. ``\emph{Thermal
stratification flow in a plenum}''). The number of tests presented in this
report will be gradually increased at each new version release of \texttt{TrioCFD}.
For example, the eighth part which concerns the modeling of the fluid/structure
interactions with the ALE method has been added since the last delivery\medskip\newline


