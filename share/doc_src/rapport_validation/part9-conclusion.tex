\lettrine[lines=2,slope=0pt,nindent=4pt]{\textbf{T}}{his} document
is the first version of a \texttt{TrioCFD} validation report resulting
of an analysis and a sorting work that have been done on its database.
First, an important inventory job was carried out in order to sort
the test cases for targeting quickly the use of \texttt{TrioCFD}
in different CFD configurations. The inventory resulted in a single
table with plenty information (LibreOffice format), where the test
cases are classified into several subdomains of fluid flows. In this
document, some of them have been selected and detailed because 1)
they are well-known in the literature, 2) they present comparisons
with other academic or commercial CFD codes, 3) they present comparisons
with experimental data and 4) they cover an important and representative part of the
physics of the code. In this report, the test cases are representative
of five subdomains: ``Laminar flow'' (Part III), ``Thermal laminar
flow'' (Part IV), ``Turbulent flow'' (Part V), ``Thermal turbulent
flow'' (Part VI), ``Front Tracking'' (Part VII) and ``Fluid/structure interactions with ALE'' (Part VIII).
The first four parts gather the test cases for single phase flows, coupled or not with
turbulence models and thermal effects. Part VII is dedicated
to two-phase flows with interface tracking and the last part (VIII) that has been added since
the last version of this report focuses on fluid/structure interactions with Arbitrary
Lagrangian-Eulerian Method (ALE). The corresponding datafiles
were run with the \texttt{1.8.2} version of \texttt{TrioCFD}
to check the achievement of computations. Meanwhile, an important
work was carried out to update a new \texttt{PRM} template in order
to standardize all validation sheets. For each one of them, let us
remind that the PDF file is generated by running a bash script (command
\texttt{Run\_fiche -not\_run}) which acts on a \texttt{PRM} file (previously, test cases must have
been run). A \texttt{PRM} file is a set of specific instructions for interfacing
the \LaTeX ~commands with the \texttt{TrioCFD} results post-processed
with \texttt{Gnuplot} or \texttt{Visit}. Its content can be freely
chosen by authors which has the consequence that the number and titles
of sections differ from one sheet to another one. The goal of the
new \texttt{PRM} template is to harmonize their contents for a more
homogeneous rendering of this report. The seven sections of the new
\texttt{PRM} template detail the main stages of CFD modeling and describe
the comparisons. All validation sheets in this report have been revised
and enhanced by taking into account this new \texttt{PRM} template.

\section*{Perspective}

Numerous other validation sheets already exist in the \texttt{TrioCFD}
database and the job must be pursued. Hence, this document does not
present an exhaustive list of what \texttt{TrioCFD} can do for CFD
applications. It can be viewed as a simple ``snapshot'' that will
be gradually improved and increased at each version release. The improvement
will be simplified by the methodology and the tools which have been
developed for \texttt{PRM} files. After checking and updating the
validations sheets, they will be added in the future versions of this
report. Among the available test cases, the sort will be pursued to
separate those currently ``in progress'' and the others that lack
quantitative comparisons. Multiple variations of the same test case
appear in the database (e.g. ``Poiseuille flow''). Some of them
are simply a 3D extension of the same test case, or an extension with
temperature equation or turbulence model. For instance, the laminar
test case of ``flow with a cylinder'' (Chapter III.3) exists in
turbulent version in two and three dimensions. Another example is
given by the test case named ``Backward facing step'' which appears
in four different versions: the first one is ``two-dimensional'',
the second one is ``implicit'', the third one is ``three-dimensional''
and the last one is ``heated in two dimensions''. More test cases
of turbulent flow are also available, such as ``Baglietto'' and
``Flow in curved pipe''. An effort will be done to extend the
number of CFD subdomains such as ``Quasi-compressible flow'', ``Flow
in porous media'' and ``fluid-structure interaction''. Finally
several tests are dedicated to the ``grid convergence'' or ``Miscellaneous
test'' of numerical options.
