%% LyX 2.1.3 created this file.  For more info, see http://www.lyx.org/.
%% Do not edit unless you really know what you are doing.
%%%\documentclass[11pt,english,french]{report}
\documentclass[11pt,a4paper,lmsf,nt]{docDM2S}
\usepackage{xspace}                    %% laisse un espace adequate pour raccourci
\usepackage{layout}
\usepackage{graphicx}
%\usepackage{subfig}
\usepackage{amssymb}
\usepackage{amsmath}
\usepackage{amsthm}
\usepackage{textcomp}
\usepackage[french]{babel}
\usepackage[T1]{fontenc}
\usepackage{epsfig}
\usepackage{epstopdf}
\usepackage{fancybox}
\usepackage{float}
\usepackage{pifont}
\usepackage{fancyhdr}
\usepackage{ulem}
\usepackage{url}
\usepackage{color}
\usepackage{subcaption}
\usepackage{rotating}
\newcommand{\DF}{\displaystyle\frac}

%%%%%%%%%%%%%%%%%%%%%%%%%%%
\unitlength=1cm
% \graphicspath{{./Images/}}
%%\input{def}
%%%%%%%%%%%%%%%%%%%%%%%

\settitre{Documentation \texttt{TrioCFD}}
\setnumero{2019-65912}
\setindiceA{17/12/2019}{Document initial}
\setauteur{P.-E. Angeli}
\setauteur{A. Cartalade}
\setauteur{E. Jamelot}

\setverificateur{U. Bieder}{Ing\'enieur-chercheur}
%\setreferencesaction{}{}{}
\setreferencesinterne{DISN}{SIMU}{SITHY}{A-SITHY-03-03-05}
\setjalon{}{}
\setvisa{N. Dorville}{Chef de Laboratoire}

%\setreferenceB{DEN/SAC}{DM2S/STMF/LATF/NT/18-XXX/A}
\setresume{
Cette note d\'ecrit les principaux mod\`eles physiques du code de calcul \texttt{TrioCFD}
d\'edi\'e \`a la simulation de probl\`emes physiques qui impliquent des fluides.
Il s'agit d'une toute premi\`ere documentation de ce code qui a \'evolu\'e
depuis plus de vingt-cinq ans (depuis \texttt{Trio\_U}) et trouv\'e
des applications dans plusieurs sous-domaines de la m\'ecanique des
fluides. Dans cette note, la pr\'esentation des mod\`eles se concentre
sur la description des \'ecoulements monophasiques, incompressibles,
Newtoniens et turbulents. Dans la section \ref{sec:Equations-de-conservations},
les \'equations de conservation de masse, quantit\'e de mouvement et d'\'energie
sont rappel\'ees. Cette section permet aussi de rappeler les d\'efinitions
des tenseurs de contraintes, de d\'eformation et de rotation qui reviendront
dans tout le document. La section se termine par une description des
matrices de masse et de rigidit\'e lorsque le probl\`eme de Stokes est
discr\'etis\'e par la m\'ethode des Volumes-El\'ements Finis (VEF). La section
\ref{sec:Modeles-de-turbulence} suivante pr\'esente les diff\'erentes
approches qui permettent de simuler la turbulence en LES (Large Eddy
Simulations) et en RANS (Reynolds Averaged Navier-Stokes). Pour les
mod\`eles RANS, plusieurs mod\`eles de type \og $\overline{k}-\overline{\epsilon}$ \fg{}
sont pr\'esent\'es : le \og $\overline{k}-\overline{\epsilon}$ standard \fg{},
le \og $\overline{k}-\overline{\epsilon}$ r\'ealisable \fg{}, et
le \og $\overline{k}-\overline{\epsilon}$ bas Reynolds \fg{}. Enfin,
une annexe constitu\'ee de deux tables fait \'etat des 160 fiches de validation
du code. 
}
\setdiffusion{normale}
\setresumeDO{R\'esum\'e}
\setmotscles{Documentation TrioCFD} 

\setdiffusioninterne{
\begin{tabular}[!H]{ll}
DEN/DANS/DM2S		   & Catherine Santucci         \\
                        & Daniel Caruge             \\
                        & Sylvie Naury              \\
                        & Christine Poinot-Salanon  \\
                        & Thomas Laporte          \\
                        & Fr\'ed\'eric Damian     \\
                        & Jean-Luc Fayard         \\
DEN/DANS/DM2S/STMF	& Pierre Gavoille             \\
                    & Gilles Rampal               \\
DEN/DANS/DM2S/SERMA	& Lo\"\i c de Carlan              \\
DEN/DANS/DM2S/SEMT	& Val\'erie Vandenberghe      \\
DEN/DANS/DM2S/STMF/LATF & Laurent Salmon          \\
                        & Alain Genty             \\
DEN/DANS/DM2S/STMF/LMSF & Tous les agents         \\
DEN/DANS/DM2S/STMF/LMEC & Didier Schneider        \\
                        & Yannick Gorsse          \\
                        & Antoine Geschenfeld     \\
DEN/DANS/DM2S/STMF/LGLS & Aymeric Canton          \\
                        & Pierre Ledac            \\
DEN/DANS/DM2S/STMF/LMES & Kim-Claire Le Thanh     \\
DEN/DANS/DM2S/STMF/LIEFT& Xavier Tiratay          \\
                        & Gilles Bernard-Michel   \\
DEN/DISN                & Xavier Raepsaet         \\
DEN/DISN                & St\'ephanie Martin      \\
DEN/CAD/DTN/SMTA/LMAG   & Laurent Saas            \\
                        & Mathieu Peybernes       \\
DRF/IRIG/DSBT/LRTH      & Alain Girard            \\
\end{tabular}
}

\setdiffusionresume{
\begin{tabular}[!H]{ll}
 & Tous les agents STMF \\
\end{tabular}
}

\begin{document}
\pagebreak

\section{Introduction}

Le code de calcul \texttt{TrioCFD} est un logiciel open-source \cite{TrioCFD}
pour les simulations en m\'ecanique des fluides, adapt\'e en particulier
aux calculs massivement parall\`eles d\textquoteright \'ecoulements turbulents
dans des configurations industrielles. Cette note technique est une
documentation des mod\`eles physiques de ce code qui existe depuis plus
de vingt-cinq ans (depuis les travaux sur les Volumes-El\'ements Finis
dans \cite{Emon92}) et qui a trouv\'e des applications dans plusieurs
sous-domaines de la m\'ecanique des fluides. De nombreuses r\'ef\'erences
accessibles dans la litt\'erature (dont 28 th\`eses) d\'ecrivent les m\'ethodes
num\'eriques du code ou des applications originales. Ces r\'ef\'erences
(th\`eses et articles) peuvent \^etre retrouv\'ees au format \texttt{PDF}
sur le site internet \cite{TrioCFD}. Cette note technique compl\`ete
la description qualitative du site web en pr\'ecisant les mod\`eles math\'ematiques
qui sont utilis\'es dans les simulations. Compte tenu de la diversification
des applications depuis 1992, on restreint la description aux \'ecoulements
monophasiques newtoniens, incompressibles et turbulents. Pour ce type
d'\'ecoulements de nombreux d\'etails sont d\'ej\`a disponibles dans plusieurs
r\'ef\'erences dont \cite[etc ...]{Angeli_etal_NURETH2015,Angeli_etal_FVCA2017,TrioCFD}.
Cette note technique initie une documentation en faisant un effort
de synth\`ese de la litt\'erature existante, mais aussi d'homog\'en\'eisation
des notations aussi bien entre les \og mod\`eles physiques \fg{} et
la partie \og m\'ethodes num\'eriques \fg{} qui font souvent l'objet
de notations diff\'erentes entre les documents. Ici la partie num\'erique
n'appara\^it que dans la section \ref{sub:Num=0000E9rique-dans-TrioCFD}
mais plusieurs nouvelles sections apparaitront dans les versions ult\'erieures
de la mise \`a jour de cette documentation.

Deux principales m\'ethodes de discr\'etisation peuvent \^etre utilis\'ees
dans \texttt{TrioCFD} : la m\'ethode des Volume-El\'ements Finis (VEF)
et celle des Volume-Diff\'erence Finies (VDF). Pour les VEF, les maillages
associ\'es \`a cette m\'ethode doivent n\'ecessairement \^etre triangulaires
(en 2D) ou t\'etra\'edriques (en 3D). Le code est programm\'e en langage
\texttt{C++} et est bas\'e sur la base logicielle \texttt{TRUST} (\texttt{TR}io\_\texttt{U}
\texttt{S}oftware for \texttt{T}hermalhydraulics), dans lequel sont
notamment programm\'e l\textquoteright ensemble des m\'ethodes num\'eriques
et des sch\'emas de discr\'etisation utilis\'es dans \texttt{TrioCFD}. La
base \texttt{TRUST} est aussi le noyau d'autres applications du service
STMF. Il est possible de r\'ealiser des impl\'ementations locales dans
le code via le concept de \texttt{BALTIK} (\texttt{B}uilding an \texttt{A}pplication
\texttt{L}inked to \texttt{T}r\texttt{I}o\_U \texttt{K}ernel), qui
correspond \`a une brique de code (un ensemble de fichiers \texttt{.cpp}
et \texttt{.h}) \`a modifier, puis de les int\'egrer \'eventuellement dans
le code source une fois les d\'eveloppements v\'erifi\'es et valid\'es. Pour
r\'esumer, \texttt{TrioCFD} s'appuie sur \texttt{TRUST} mais cette documentation
ne concerne que la partie \texttt{TrioCFD}.

Afin de pr\'eciser les mod\`eles math\'ematiques ainsi que leur domaine
de validit\'e, il est n\'ecessaire de faire un rappel des \'equations de
conservation de masse, de quantit\'e de mouvement et d'\'energie sans
consid\'eration d'hypoth\`ese simplificatrice. C'est l'objet de la section
\ref{sec:Equations-de-conservations} dans laquelle les \'equations
seront pr\'esent\'ees sous leur forme la plus g\'en\'erale. Dans la mesure
du possible, les mod\`eles math\'ematiques sont \'ecrits aussi bien en notations
vectorielles qu'en notations indicielles. En notation vectorielle,
les vecteurs, les tenseurs et les matrices sont \'ecrits en gras. Dans
cette section, des rappels sont \'egalement effectu\'es sur les d\'efinitions
des tenseurs du taux de d\'eformation et celui de rotation car ceux-ci
sont utilis\'es dans la section suivante d\'edi\'ee aux mod\`eles de turbulence.
Les mod\`eles de turbulence en m\'ecanique des fluides sont nombreux et
plusieurs livres et publications de synth\`ese existent d\'ej\`a (e.g. \cite{Book_Chassaing,Argyropoulos-Markatos_ReviewTurbulence_AMM2015}).
Dans \texttt{TrioCFD}, les mod\`eles qui sont d\'evelopp\'es sont ceux qui
sont les plus couramment utilis\'es dans la litt\'erature et qui font
l'objet d'un consensus dans le domaine tels que la LES (Smagorinski
et WALE) et les diff\'erentes d\'eclinaisons du RANS $\overline{k}-\overline{\epsilon}$.
Ils seront d\'etaill\'es dans la section \ref{sec:Modeles-de-turbulence}.
Une base de cas tests d'environ 160 fiches de validations existe d\'ej\`a
pour \texttt{TrioCFD}. La liste de ces fiches est \`a ce jour regroup\'ee
dans un fichier \texttt{Excel} dont la version \texttt{PDF} est accessible
en interne CEA. L'ensemble des cas tests de validation est synth\'etis\'e
dans les deux tables \ref{tab:FichesValid1} et \ref{tab:FichesValid2}
dans l'annexe \ref{sec:Annexe_ListeValid} de ce document. Dans les
versions ult\'erieures de cette documentation, les mod\`eles d\'ecrits seront
illustr\'es par quelques fiches de validation de liste en Annexe. Plusieurs
mises \`a jour de cette documentation sont \`a pr\'evoir pour l'\'elargir
petit \`a petit \`a l'ensemble des mod\`eles accessibles dans le code.


\section{\label{sec:Equations-de-conservations}\'equations de conservations
: masse, quantit\'e de mouvement et \'energie}


\subsection{Rappel des \'equations fondamentales de la dynamique des fluides}

Dans cette section on rappelle les \'equations fondamentales de la dynamique
des fluides. Elle permet d'introduire les principales notations et
les \'equations aux d\'eriv\'ees partielles fondamentales sans hypoth\`eses
physiques simplificatrices. Les d\'emonstrations peuvent \^etre trouv\'ees
dans les r\'ef\'erences classiques. Dans la section \ref{sub:ConservMasse},
on rappelle l'\'equation de conservation de la masse (encore appel\'ee
\'equation de continuit\'e) et dans la section \ref{sub:QDM} on rappelle
l'\'equation de conservation de la quantit\'e de mouvement. Plusieurs
formes math\'ematiques \'equivalentes entre elles existent dans la litt\'erature
: forme locale, forme locale notation vectorielle, forme locale conservative,
forme locale conservative en notation vectorielle, formes macroscopiques,
etc ... Ici on choisit la forme locale conservative avec notations
vectorielles.


\subsection{\label{sub:ConservMasse}Conservation de la masse}

\begin{equation}
\frac{\partial\rho}{\partial t}+\boldsymbol{\nabla}\cdot(\rho\mathbf{u})=0\label{eq:ConvMasse}
\end{equation}
o\`u $\rho\equiv\rho(\mathbf{x},\,t)$ est la densit\'e avec $\mathbf{x}$
la position et $t$ le temps, et $\mathbf{u}\equiv\mathbf{u}(\mathbf{x},\,t)$
la vitesse. D'autres formes \'equivalentes de cette \'equation peuvent
\^etre rencontr\'ees en appliquant l'identit\'e vectorielle $\boldsymbol{\nabla}\cdot(\rho\mathbf{u})=\rho\boldsymbol{\nabla}\cdot\mathbf{u}+\mathbf{u}\cdot\boldsymbol{\nabla}\rho$
et en faisant appara\^itre la d\'eriv\'ee mat\'erielle $d\rho/dt=\partial\rho/\partial t+\mathbf{u}\cdot\boldsymbol{\nabla}\rho$.


\subsection{\label{sub:QDM}Conservation de la quantit\'e de mouvement}

L'\'equation de conservation de la Quantit\'e De Mouvement (QDM) traduit
le principe fondamental de la dynamique qui indique que la variation
de quantit\'e de mouvement \`a l'int\'erieur d'un volume de contr\^ole est
\'egale \`a la somme de toutes les forces ext\'erieures qui lui sont appliqu\'ees.
Les forces qui s'appliquent sur le volume \'el\'ementaires peuvent \^etre
s\'epar\'ees en forces de volume et forces de surface. Ces derni\`eres s'expriment
comme un vecteur contrainte qui agit sur une surface, et ce vecteur
contrainte s'exprime \`a son tour comme le produit scalaire d'un tenseur
des contraintes $\mathbf{T}$ (de composante $T_{ij}$) et du vecteur
normal \`a la surface $\mathbf{n}$. La contrainte totale est d\'ecompos\'ee
en deux parties : $T_{ij}=-p\delta_{ij}+\tau_{ij}$. La premi\`ere est
le tenseur des contraintes associ\'ees \`a la pression $-p\delta_{ij}$
o\`u $p\equiv p(\mathbf{x},\,t)$ est la pression et $\delta_{ij}$
est le symbole de Kronecker qui vaut un si $i=j$ et z\'ero sinon. La
seconde, not\'ee $\tau_{ij}$ est associ\'ee aux contraintes visqueuses.
La pression agit de fa\c con isotrope et sa valeur d\'epend de l'\'etat thermodynamique
du fluide. Les contraintes visqueuses sont li\'ees \`a l'\'etat de d\'eformation
du fluide

Comme pour l'\'equation de conservation de la masse, plusieurs formes
math\'ematiques et \'equivalentes entre elles peuvent \^etre d\'eduites. Son
\'ecriture sous forme locale conservative est :
\begin{equation}
\frac{\partial(\rho\mathbf{u})}{\partial t}+\boldsymbol{\nabla}\cdot(\rho\mathbf{u}\mathbf{u})=-\boldsymbol{\nabla}p+\boldsymbol{\nabla}\cdot\boldsymbol{\tau}+\mathbf{F}_{v}\label{eq:QDM}
\end{equation}


Dans l'\'equation (\ref{eq:QDM}), le membre de gauche de l'\'equation
repr\'esente la quantit\'e d'acc\'el\'eration par unit\'e de volume. Les termes
du membre de droite repr\'esentent respectivement (\emph{i}) les forces
associ\'ees \`a la pression par unit\'e de volume, (\emph{ii}) les contraintes
visqueuses par unit\'e de volume et (\emph{iii}) la force externes par
unit\'e de volume. Lorsqu'on ne consid\`ere que la gravit\'e elle s'exprime
sous la forme : $\mathbf{F}_{v}=\rho\mathbf{g}$.


\subsection{Forme du tenseur des contraintes visqueuses}

Le tenseur des contraintes visqueuses $\mathbf{\tau}$ est g\'en\'eralement
exprim\'e en fonction des taux de d\'eformation dans l'\'ecoulement. On
rappelle ci-dessous les d\'efinitions des tenseurs des taux de d\'eformation
et des taux de rotation \`a partir desquels sera exprim\'e le tenseur
des contraintes visqueuses.


\subsubsection*{Rappel du tenseur des taux de d\'eformation}

L'accroissement de vitesse de deux particules fluides positionn\'ees
respectivement en $\mathbf{r}$ et $\mathbf{r}+d\mathbf{r}$ et de
vitesse $\mathbf{u}$ et $\mathbf{u}+d\mathbf{u}$ s'exprime sous
la forme $du_{i}=\sum_{j=1}^{3}(\partial u_{i}/\partial x_{j})dx_{j}$
au premier ordre par rapport aux composantes $dx_{j}$ (pour $j=1,2,3$).
Dans cette expression, les quantit\'es $G_{ij}=\partial u_{i}/\partial x_{j}$
sont les \'el\'ements d'un tenseur de rang deux, le tenseur des taux de
d\'eformation du fluide (ou des gradients de vitesse). En trois dimensions,
il s'\'ecrit sous la forme d'une matrice $3\times3$ qui peut \^etre d\'ecompos\'ee
en une partie sym\'etrique et une partie antisym\'etrique :

\begin{equation}
G_{ij}=\frac{\partial u_{i}}{\partial x_{j}}=\frac{1}{2}\left(\frac{\partial u_{i}}{\partial x_{j}}+\frac{\partial u_{j}}{\partial x_{i}}\right)+\frac{1}{2}\left(\frac{\partial u_{i}}{\partial x_{j}}-\frac{\partial u_{j}}{\partial x_{i}}\right)\label{eq:TauxDeformation}
\end{equation}


Le premier terme est le tenseur des taux des d\'eformations :

\begin{equation}
S_{ij}=\frac{1}{2}\left(\frac{\partial u_{i}}{\partial x_{j}}+\frac{\partial u_{j}}{\partial x_{i}}\right)\label{eq:TenseurDeformations}
\end{equation}
et il est sym\'etrique ($S_{ij}=S_{ji}$). Le second terme est le tenseur
des taux de rotation :

\begin{equation}
\Omega_{ij}=\frac{1}{2}\left(\frac{\partial u_{i}}{\partial x_{j}}-\frac{\partial u_{j}}{\partial x_{i}}\right)\label{eq:TenseurRotations}
\end{equation}
et ce tenseur est antisym\'etrique ($\Omega_{ij}=-\Omega_{ji}$).


\subsubsection*{Forme du tenseur des contraintes visqueuses pour un fluide newtonien}

Lorsque les fluides sont newtoniens la relation contrainte-d\'eformation
est lin\'eaire et isotrope. La relation g\'en\'erale s'\'ecrit :

\begin{equation}
\tau_{ij}=\eta\left(2S_{ij}-\frac{2}{3}S_{kk}\delta_{ij}\right)+\zeta S_{kk}\delta_{ij}\label{eq:Contraintes-Newtonien}
\end{equation}
qui fait appara\^itre deux viscosit\'es, la viscosit\'e dynamique $\eta\equiv\eta(\mathbf{x},\,t)$
et la viscosit\'e de volume $\zeta$ (ou deuxi\`eme viscosit\'e). Le premier
terme correspond \`a une d\'eformation sans changement de volume tandis
que le second terme correspond \`a une dilatation isotrope. Dans la
majeure partie des applications on ne tient pas compte de la viscosit\'e
en volume ($\zeta=0$) et le tenseur des contraintes s'\'ecrit $\tau_{ij}=2\eta S_{ij}-(2/3)\eta S_{kk}\delta_{ij}$,
soit en utilisant la relation (\ref{eq:TenseurDeformations}) :

\begin{equation}
\tau_{ij}=\eta\left(\frac{\partial u_{i}}{\partial x_{j}}+\frac{\partial u_{j}}{\partial x_{i}}\right)-\frac{2}{3}\eta\left(\frac{\partial u_{k}}{\partial x_{k}}\right)\delta_{ij}\label{eq:Contraintes}
\end{equation}
ou encore en notations vectorielles :

\begin{equation}
\boldsymbol{\tau}=\eta(\boldsymbol{\nabla}\mathbf{u}+\boldsymbol{\nabla}^{T}\mathbf{u})-\frac{2}{3}\eta(\boldsymbol{\nabla}\cdot\mathbf{u})\mathbf{I}\label{eq:Contraintes-Deformations_Newtonien}
\end{equation}
o\`u $\mathbf{I}$ est la matrice diagonale unit\'e.


\subsubsection*{Fluide newtonien de viscosit\'e constante}

Lorsque la viscosit\'e du fluide $\eta$ est constante (i.e. $\eta(\mathbf{x},\,t)=\eta_{0}=\mbox{Cte}$),
le terme des contraintes visqueuses $\boldsymbol{\nabla}\cdot\boldsymbol{\tau}$
dans l'\'equation (\ref{eq:QDM}) s'exprime sous la forme :

\begin{eqnarray*}
\frac{\partial\tau_{ij}}{\partial x_{j}} & = & \eta_{0}\left[\frac{\partial^{2}u_{i}}{\partial x_{j}\partial x_{j}}+\frac{\partial}{\partial x_{j}}\left(\frac{\partial u_{j}}{\partial x_{i}}\right)\right]-\frac{2}{3}\eta_{0}\frac{\partial}{\partial x_{j}}(\boldsymbol{\nabla}\cdot\mathbf{u})\delta_{ij}\\
 & = & \eta_{0}\left[\frac{\partial^{2}u_{i}}{\partial x_{j}\partial x_{j}}+\frac{\partial}{\partial x_{i}}(\boldsymbol{\nabla}\cdot\mathbf{u})\right]-\frac{2}{3}\eta_{0}\frac{\partial}{\partial x_{i}}(\boldsymbol{\nabla}\cdot\mathbf{u})\\
 & = & \eta_{0}\boldsymbol{\nabla}^{2}u_{i}+\frac{\eta_{0}}{3}\frac{\partial}{\partial x_{i}}(\boldsymbol{\nabla}\cdot\mathbf{u})
\end{eqnarray*}


C'est-\`a-dire en notations vectorielles :

\begin{equation}
\boldsymbol{\nabla}\cdot\boldsymbol{\tau}=\eta_{0}\boldsymbol{\nabla}^{2}\mathbf{u}+\frac{\eta_{0}}{3}\boldsymbol{\nabla}(\boldsymbol{\nabla}\cdot\mathbf{u})\label{eq:ContraintVisq-constante}
\end{equation}



\subsection{R\'esum\'e}


\subsubsection{Cas g\'en\'eral pour un fluide newtonien}

Les \'equations de conservation de la masse et de la quantit\'e de mouvement
s'\'ecrivent sous la forme g\'en\'erale :

\begin{subequations}

\begin{eqnarray}
\frac{\partial\rho}{\partial t}+\boldsymbol{\nabla}\cdot(\rho\mathbf{u}) & = & 0\label{eq:Bilan_ConservMasse}\\
\frac{\partial(\rho\mathbf{u})}{\partial t}+\boldsymbol{\nabla}\cdot(\rho\mathbf{u}\mathbf{u}) & = & -\boldsymbol{\nabla}p+\boldsymbol{\nabla}\cdot\boldsymbol{\tau}+\rho\mathbf{F}_{v}\label{eq:Bilan_QDM}
\end{eqnarray}
o\`u $\rho\equiv\rho(\mathbf{x},\,t)$ et $\mathbf{u}\equiv\mathbf{u}(\mathbf{x},\,t)$
sont les inconnues du syst\`eme d'\'equations. La loi d'\'etat sur la pression
et l'hypoth\`ese de fluide newtonien pour le tenseur des contraintes
permettent de fermer le syst\`eme. La loi d'\'etat sur la pression et
les mod\`eles de type \og bas Mach \fg{} qui s\'eparent la pression
en une pression thermodynamique et une pression hydrodynamique sont
pr\'esent\'es dans le chapitre suivant. Pour un fluide newtonien, le tenseur
des contraintes $\mathbf{\tau}$ est reli\'e \`a celui des d\'eformations
$\mathbf{D}$ par la relation :

\end{subequations}

\begin{equation}
\boldsymbol{\tau}=\eta(\boldsymbol{\nabla}\mathbf{u}+\boldsymbol{\nabla}^{T}\mathbf{u})-\frac{2}{3}\eta(\boldsymbol{\nabla}\cdot\mathbf{u})\mathbf{I}\label{eq:Bilan_TenseurContrainte}
\end{equation}
o\`u la viscosit\'e dynamique est not\'ee $\eta\equiv\eta(\mathbf{x},\,t)$
et la viscosit\'e de volume $\zeta$ a \'et\'e n\'eglig\'ee.


\subsubsection{Cas particulier d'une viscosit\'e constante}

Lorsque la viscosit\'e est consid\'er\'ee constante (i.e. $\eta(\mathbf{x},\,t)=\eta_{0}=\mbox{Cte}$),
le terme de divergence du tenseur des contraintes $\boldsymbol{\nabla}\cdot\boldsymbol{\tau}$
se simplifie \`a l'aide de la relation (\ref{eq:ContraintVisq-constante})
et le syst\`eme d'\'equations devient :

\begin{subequations}

\begin{eqnarray}
\frac{\partial\rho}{\partial t}+\boldsymbol{\nabla}\cdot(\rho\mathbf{u}) & = & 0\label{eq:Bilan_ConservMasse-1}\\
\frac{\partial(\rho\mathbf{u})}{\partial t}+\boldsymbol{\nabla}\cdot(\rho\mathbf{u}\mathbf{u}) & = & -\boldsymbol{\nabla}p+\eta_{0}\boldsymbol{\nabla}^{2}\mathbf{u}+\frac{\eta_{0}}{3}\boldsymbol{\nabla}(\boldsymbol{\nabla}\cdot\mathbf{u})+\rho\mathbf{F}_{v}\label{eq:Bilan_QDM-1}
\end{eqnarray}


\end{subequations}


\subsubsection{Cas particulier des \'ecoulements incompressibles avec viscosit\'e variable}

Lorsque le fluide est consid\'er\'e incompressible (i.e. $\rho(\mathbf{x},\,t)=\rho_{0}=\mbox{Cte})$
alors l'\'equation de conservation de la masse devient $\boldsymbol{\nabla}\cdot\mathbf{u}=0$
(car $\partial\rho_{0}/\partial t=0$ et $\boldsymbol{\nabla}\rho_{0}=\mathbf{0}$),
le terme non lin\'eaire s'\'ecrit $\boldsymbol{\nabla}\cdot(\rho_{0}\mathbf{u}\mathbf{u})=\rho_{0}\mathbf{u}\cdot\boldsymbol{\nabla}\mathbf{u}$
et le tenseur des contraintes visqueuses (Eq. (\ref{eq:ContraintVisq-constante}))
se simplifie lui aussi en $\boldsymbol{\tau}=\eta(\boldsymbol{\nabla}\mathbf{u}+\boldsymbol{\nabla}^{T}\mathbf{u})$. 

\begin{subequations}

\begin{align}
\boldsymbol{\nabla}\cdot\mathbf{u} & =0,\label{eq:ContinuiteMP-1}\\
\rho_{0}\frac{\partial\mathbf{u}}{\partial t}+\rho_{0}\mathbf{u}\cdot\boldsymbol{\nabla}\mathbf{u} & =-\boldsymbol{\nabla}p+\boldsymbol{\nabla}\cdot\left[\eta(\boldsymbol{\nabla}\mathbf{u}+\boldsymbol{\nabla}^{T}\mathbf{u})\right]+\rho_{0}\mathbf{F}_{v}\label{eq:DBF-1-1}
\end{align}


\end{subequations}

Dans cette formulation, la viscosit\'e dynamique $\eta$ reste dans
le terme entre crochets car il peut d\'ependre de la position comme
dans les mod\`eles de turbulence.


\subsubsection{Cas particulier des \'ecoulements incompressibles avec viscosit\'e constante}

Enfin, lorsque le fluide est consid\'er\'e incompressible et de viscosit\'e
dynamique $\eta_{0}$ constante, le tenseur des contraintes visqueuses
(Eq. (\ref{eq:ContraintVisq-constante})) se simplifie une nouvelle
fois en $\boldsymbol{\nabla}\cdot\boldsymbol{\tau}=\eta_{0}\boldsymbol{\nabla}^{2}\mathbf{u}$
et le syst\`eme d'\'equations devient :

\begin{subequations}

\begin{align}
\boldsymbol{\nabla}\cdot\mathbf{u} & =0,\label{eq:ContinuiteMP}\\
\rho_{0}\frac{\partial\mathbf{u}}{\partial t}+\rho_{0}\mathbf{u}\cdot\boldsymbol{\nabla}\mathbf{u} & =-\boldsymbol{\nabla}p+\eta_{0}\boldsymbol{\nabla}^{2}\mathbf{u}+\rho_{0}\mathbf{F}_{v}\label{eq:DBF-1}
\end{align}


\end{subequations}


\subsection{Conservation de l'\'energie}

Plusieurs formes de l'\'equation de bilan de l'\'energie sont possibles
selon que l'on consid\`ere la conservation de l'\'energie totale, l'\'energie
interne, l'enthalpie totale ou l'enthalpie. Des formulations peuvent
\^etre d\'eduites sur la temp\'erature ou m\^eme l'entropie. Dans la suite,
on restreint la pr\'esentation \`a l'\'ecriture de la conservation de l'\'energie
interne qui sera \'ecrite sous forme \'equivalente sur l'\'equation de la
temp\'erature.

L'\'equation de bilan de l'\'energie interne $e$ s'\'ecrit \cite[p. 126]{Book_Candel}
:

\begin{equation}
\frac{\partial\rho e}{\partial t}+\boldsymbol{\nabla}\cdot(\rho e\mathbf{u})=-\boldsymbol{\nabla}\cdot\mathbf{q}-p\boldsymbol{\nabla}\cdot\mathbf{u}+\boldsymbol{\tau}:\boldsymbol{\nabla}\mathbf{u}\label{eq:Energie}
\end{equation}
Dans cette \'equation, le membre de gauche repr\'esente le taux de variation
de l'\'energie interne par unit\'e de volume. Dans le membre de droite,
le premier terme repr\'esente le flux de chaleur par unit\'e de volume
; le second terme repr\'esente la puissance des forces de pression par
unit\'e de volume ; et le dernier terme repr\'esente la puissance des
forces visqueuses par unit\'e de volume. Ce dernier est la fonction
de dissipation visqueuse qui est toujours positive ou nulle. Ainsi
les forces visqueuses entra\^inent toujours un accroissement de l'\'energie
interne du fluide et donc de sa temp\'erature.

L'\'equation de conservation de l'\'energie interne peut se reformuler
en une \'equation sur la temp\'erature $T$. Cette \'equation prend deux
formes diff\'erentes selon que l'on utilise la chaleur sp\'ecifique \`a
volume constant $C_{v}$ ou bien \`a pression constante $C_{p}$. Formul\'ee
en $C_{v}$, elle s'\'ecrit :

\begin{subequations}

\begin{equation}
\frac{\partial(\rho C_{v}T)}{\partial t}+\boldsymbol{\nabla}\cdot(\rho C_{v}T\mathbf{u})=-\boldsymbol{\nabla}\cdot\mathbf{q}-T\left(\frac{\partial p}{\partial T}\right)_{\rho}\boldsymbol{\nabla}\cdot\mathbf{u}+\boldsymbol{\tau}:\boldsymbol{\nabla}\mathbf{u}+\rho T\frac{dC_{v}}{dt}\label{eq:Energie_Cv}
\end{equation}
Formul\'ee en $C_{p}$ elle s'\'ecrit :

\begin{equation}
\frac{\partial(\rho C_{p}T)}{\partial t}+\boldsymbol{\nabla}\cdot(\rho C_{p}T\mathbf{u})=-\boldsymbol{\nabla}\cdot\mathbf{q}-\left(\frac{\partial\ln\rho}{\partial\ln T}\right)\frac{dp}{dt}+\boldsymbol{\tau}:\boldsymbol{\nabla}\mathbf{u}+\rho T\frac{dC_{p}}{dt}\label{eq:Energie_Cp}
\end{equation}


\end{subequations}
\begin{description}
\item [{Remarque}] : lorsque le gaz est consid\'er\'e parfait, i.e. $\rho=p_{th}/(RT)$
o\`u $R$ est la constante des gaz parfaits et $p_{th}$ est la pression
thermodynamique, le coefficient $(\partial\ln\rho/\partial\ln T)$
vaut $(\partial\ln\rho/\partial\ln T)=-1$. En supposant que le flux
de chaleur est d\'efini par la loi de Fourier $\mathbf{q}=-\lambda\boldsymbol{\nabla}T$
o\`u $\lambda$ est la conductivit\'e thermique, l'Eq. (\ref{eq:Energie_Cp})
devient :
\end{description}
\begin{equation}
\frac{\partial(\rho C_{p}T)}{\partial t}+\boldsymbol{\nabla}\cdot(\rho C_{p}T\mathbf{u})=\boldsymbol{\nabla}\cdot(\lambda\boldsymbol{\nabla}T)+\frac{dp_{th}}{dt}+\boldsymbol{\tau}:\boldsymbol{\nabla}\mathbf{u}+\rho T\frac{dC_{p}}{dt}\label{eq:Energie_Cp-1}
\end{equation}


L'introduction de la pression thermodynamique $p_{th}$ sera utile
pour les mod\`eles \og bas Mach \fg{}.


\subsection{Approximation de Boussinesq en incompressible}

Pour des \'ecoulements incompressibles, lorsque la densit\'e est suppos\'ee
constante $\rho=\rho_{0}=\mbox{Cte}$, la densit\'e n'est ni une fonction
de la temp\'erature ni de la composition du fluide (pour les m\'elanges
miscibles). Dans ce cas, les effets de flottabilit\'e sont uniquement
pris en compte par les forces gravitationnelles. Cette simplification
est connue comme l'\og approximation de Boussinesq \fg{} et valable
en consid\'erant que la variation de densit\'e $\Delta\rho\ll\rho_{0}$.
Dans ce cas, le terme force s'\'ecrit dans les \'equations (\ref{eq:ContinuiteMP-1})--(\ref{eq:DBF-1-1})
ou (\ref{eq:ContinuiteMP})--(\ref{eq:DBF-1}) :

\begin{equation}
\mathbf{F}_{v}=-\mathbf{g}\beta_{T}(T-T_{0})\label{eq:ApproxBoussinesq}
\end{equation}
o\`u $\beta_{T}$ est le coefficient de dilatation thermique et $T_{0}$
une temp\'erature de r\'ef\'erence. Dans cette relation, le signe n\'egatif
indique que si la diff\'erence de temp\'erature est positive $\Delta T=T-T_{0}>0$
(i.e. pr\`es de la paroi chaude en convection naturelle), alors la force
est dirig\'ee dans le sens oppos\'e \`a la gravit\'e $\mathbf{g}$.


\subsection{\label{sub:Num=0000E9rique-dans-TrioCFD}Num\'erique dans \texttt{TrioCFD}}

Dans ce document on d\'etaille les m\'ethodes num\'eriques mises en \oe uvre
dans \texttt{TrioCFD} pour le mod\`ele incompressible d\'efini par les
\'equations (\ref{eq:ContinuiteMP-1})--(\ref{eq:DBF-1-1}). Deux m\'ethodes
de discr\'etisation spatiales sont possibles dans l'outil de calculs
: la m\'ethode des Volume-El\'ements Finis (VEF) et celle des Volumes
Diff\'erences Finies (VDF) mais on ne d\'ecrit que la partie VEF. Dans
la suite, le domaine de calcul est not\'e $\Omega$.


\subsubsection{Sch\'ema en temps (valable en VEF et VDF)}

Apr\`es discr\'etisation, le syst\`eme matrice-vecteur r\'esolu s'\'ecrit :

\begin{equation}
\left\{ \begin{array}{rcl}
\delta t^{-1}\mathbf{M}U^{n+1}+\mathbf{A}U^{n+1}+\mathbf{L}(U^{n})U^{n+1}+\mathbf{B}^{T}P^{n+1} & = & F^{n}+\delta t^{-1}\mathbf{M}U^{n},\\
\mathbf{B}U^{n+1} & = & 0.
\end{array}\right.\label{eq:NavierStokes-dis-FV}
\end{equation}
o\`u $U^{n+1}\in\mathbb{R^{N_{\mathbf{u}}}}$ repr\'esente le vecteur
vitesse discr\'etis\'e au temps $(n+1)\delta t$ o\`u $\delta t$ est le
pas de temps et $N_{\mathbf{u}}$ est le nombre de degr\'es de libert\'e
pour discr\'etiser spatialement la vitesse. Les matrices en gras seront
d\'efinies ci-dessous car d\'ependantes de la discr\'etisation en espace.
$P^{n+1}\in\mathbb{R}^{N_{p}}$ repr\'esente la pression discr\'etis\'ee
au temps $(n+1)\delta t$ et $N_{p}$ est le nombre de degr\'es de libert\'e
pour discr\'etiser spatialement la pression.

Afin de d\'ecoupler la vitesse et la pression, la r\'esolution des \'equations
(\ref{eq:ContinuiteMP-1})--(\ref{eq:DBF-1-1}) est r\'ealis\'ee en trois
\'etapes \cite{Chor68,Tema68}: 
\begin{itemize}
\item \textbf{\'etape de pr\'ediction :} calculer $U^{*}$ solution de 
\[
{\delta t}^{-1}\mathbf{M}U^{*}+\mathbf{A}U^{*}+\mathbf{L}(U^{n})U^{*}+\mathbf{B}^{T}P^{n}=F^{n}+\delta t^{-1}\mathbf{M}U^{n}.
\]
\`a cette \'etape $\mathbf{B}U^{*}\neq0$. 
\item \textbf{Calcul de la pression :} calculer $P'$ solution de 
\[
\mathbf{B}\mathbf{M}^{-1}\mathbf{B}^{T}P'={\delta t}^{-1}\mathbf{B}U^{*},\quad P^{n+1}=P'+P^{n}.
\]

\item \textbf{\'etape de correction :} calculer $U^{n+1}$ solution de 
\[
\mathbf{M}U^{n+1}=\mathbf{M}U^{*}-{\delta t}\mathbf{B}^{T}P'.
\]
\end{itemize}
\begin{description}
\item [{Remarque}] : plusieurs solveurs ont \'et\'e test\'es (SIMPLE, SIMPLER,
PISO) qui, selon les probl\`emes ont montr\'e une convergence relativement
faible. Le solveur utilis\'e \`a ce jour est inspir\'e de la r\'ef\'erence \cite{Guermond-Quartapelle_IJNMF1998}.
\end{description}

\subsubsection{Sch\'ema en espace VEF }

La m\'ethode num\'erique est bas\'ee sur la m\'ethode des \'el\'ements finis de
Crouzeix-Raviart non conformes \cite{CrRa73}, et d\'etaill\'es dans \cite{Emon92,Heib03,Fort06,Angeli_etal_FVCA2017}.
Pour ($d=2$) (resp. $d=3$), on consid\`ere l'espace $xy$ de $\mathbb{R}^{2}$
(resp. l'espace $xyz$ de $\mathbb{R}^{3}$) d'origine $O$. On note
($\mathbf{e}_{x},\,\mathbf{e}_{y},\,\mathbf{e}_{z})$ ($\mathbf{e}_{1},\,\mathbf{e}_{2},\,\mathbf{e}_{3})$
les vecteurs de la base canonique. On consid\`ere $\mathcal{T}_{h}:={\displaystyle \cup}{}_{\ell=1}^{N_{T}}T_{\ell}$,
un maillage r\'egulier du domaine $\Omega$ constitu\'e de simplexes (triangles
en 2D et t\'etra\`edres en 3D), de sommets $(S_{i})_{i=1}^{N_{S}}$, o\`u
$i$ est l'indice de $N_{s}$ sommets. Le nombre de simplexes (sommets)
est not\'e par $N_{T}$ (resp. $N_{S}$). Soit $T_{\ell}\in\mathcal{T}_{h},$la
fronti\`ere de $T_{\ell}$ est constitu\'ee d'ar\^etes en 2D ou de faces
en 3D, mais on les appellera \og face \fg{} dans tous les cas. Soit
$\overline{\mathcal{F}}_{h}=\cup_{k=1}^{\overline{N}_{F}}F_{k}$ l'ensemble
des faces du maillage, et $\mathcal{F}_{h}=\cup_{k=1}^{N_{F}}F_{k}$
l'ensemble des faces int\'erieures, o\`u $\overline{N}_{F}$ (resp. $N_{F}$)
est le nombre total (resp. interne) de faces. On note $M_{k}$ le
barycentre de la face $F_{k}$, et $\mathbf{n}_{k}$ le vecteur normal
unitaire sortant \`a $F_{k}$.

Soit $P_{1}(T)$ l'ensemble des polyn\^omes d'ordre 1 d\'efinis sur $T$.
L'espace de discr\'etisation des vitesses est :

\begin{equation}
X_{h}:=\{\mathbf{v}_{h}\,|\,\forall T\in\mathcal{T}_{h},\,\mathbf{v}_{h}\in P_{1}(T)^{d}\,\mbox{et}\,\forall\,F\in\mathcal{F}{}_{h}:\,[\mathbf{v}_{h}](\mathbf{x}_{F})=\mathbf{0}\,\},\label{eq:CrRa}
\end{equation}
o\`u $\mathbf{x}_{F}$ repr\'esente le barycentre de la face $F$, et
$[\mathbf{v}_{h}](\mathbf{x}_{F})$ est le saut de $\mathbf{v}_{h}$
sur la face $F$. On suppose que $\overline{F}=\overline{T}\cap\overline{T'}$,
telle que $\mathbf{n}_{F}=\mathbf{n}_{T,F}$. Le saut $[\mathbf{v}_{h}](\mathbf{x}_{F})$
est d\'efini par : 
\[
[\mathbf{v}_{h}](\mathbf{x}_{F}):=\mathbf{v}_{h|T_{\ell}}(\mathbf{x}_{F})-\mathbf{v}_{h|T_{\ell'}}(\mathbf{x}_{F})\mbox{ if }\overline{F}=\overline{T}_{\ell}\cap\overline{T}_{\ell'},\mbox{ et }\mathbf{n}_{F}=\mathbf{n}_{T_{\ell}|F}.
\]
L'espace $X_{h}$ est muni de la semi-norme $||\mathbf{v}_{h}||_{h}=\left(\sum_{\ell=1}^{T_{\ell}}|\mathbf{v}_{h|T_{\ell}}|_{1,T_{\ell}}^{2}\right)^{1/2}$,
o\`u $|\mathbf{v}_{h|T_{\ell}}|_{1,T_{\ell}}$ est la semi-norme de
$\mathbf{v}_{h|T_{\ell}}\in H^{1}(T_{\ell})$. On note $X_{0,h}:=\{\mathbf{v}_{h}\in X_{h}\,|\,\mathbf{u}_{h|\partial\Omega}=0\}$. 

Soit $\lambda_{i}|_{T}$ la coordonn\'ees barycentrique associ\'ee au
sommet $S_{i}|_{T}$ et $\left(\mathbf{e}^{\beta}\right)_{\beta=1}^{d}$
les vecteurs de la base canonique. Les fonctions de base associ\'ees
\`a l'espace $X_{h}$ sont les vecteurs $\left(\left(\boldsymbol{\varphi}_{i}^{\beta}\right)_{i=1}^{N_{F}}\right)_{\beta=1}^{d}$
tels que $\boldsymbol{\varphi}_{i}^{\beta}|_{T}=\left(1-d\lambda_{i}|_{T}\right)\mathbf{e}^{\beta}$.
On appelle $\psi_{j}$ la fonction caract\'eristique associ\'ee au triangle
$T_{j}$.


\subsubsection{Matrices volume-\'el\'ements finis pour l'approximation P1NC/P0}

Soit $(\mathbf{u}{}_{h}^{n},\,p_{h}^{n})\in X_{h}\times L_{h}$ l'approximation
spatiale de $(\mathbf{u}^{n},\,p^{n})$ dans $X_{h}\times L_{h}$
telle que : 
\[
\mathbf{u}_{h}^{n}:=\sum_{\beta=1}^{d}\sum_{i=1}^{N_{F}}\left(U_{i}^{\beta}\right)^{n}\boldsymbol{\varphi}{}_{i}^{\beta},\quad p_{h}^{n}:=\sum_{\ell=1}^{N_{T}}P_{\ell}^{n}\psi_{\ell}.
\]
Posons : $U_{n}=(\,(U_{i}^{\beta})^{n}\,)_{\beta,i}\in\mathbb{R}^{N_{\mathbf{u}}}$,
$F^{n}=(\,(\mathbf{f}^{n+1},\boldsymbol{\varphi}_{i}^{\beta})_{0}\,)_{\beta,i}\in\mathbb{R}^{N_{\mathbf{u}}}$,
avec $N_{\mathbf{u}}:=d\,N_{F}$, et $p_{h}^{n}=(P_{\ell}^{n})_{\ell}\in\mathbb{R}^{N_{T}}$. 
\begin{itemize}
\item La matrice de masse $\mathbf{M}\in\mathbb{R}^{N_{\mathbf{u}}\times N_{\mathbf{u}}}$
est compos\'ee de $d$ blocs diagonaux \'egaux \`a $\mathbf{M}_{F}\in\mathbb{R}^{N_{F}\times N_{F}}$
tels que $\left(\mathbf{M}_{F}\right)_{i,j}=(\phi_{i},\phi_{j})_{0}$
avec $\phi_{i}|_{T}=\left(1-d\lambda_{i}|_{T}\right)$. 


Pour $d=2$, on obtient que la matrice de masse $2D$ est diagonale
: 
\[
\left(\mathbf{M}_{F}\right)_{i,j}=\delta_{ij}\sum_{\ell\,|\,M_{i}\in T_{\ell}}\frac{|T_{\ell}|}{3}.
\]
La matrice de masse 2D de la m\'ethode des \'el\'ements finis non conformes
de Crouzeix-Raviart est \'egale \`a la matrice de masse 2D obtenue par
la m\'ethode des volumes finis d\'ecrite dans la th\`ese de Emonot \cite{Emon92}.
Ce n'est plus le cas en 3D, mais dans le code \texttt{TrioCFD} c'est
la la matrice de masse de la discr\'etisation en volumes finis qui est
impl\'ement\'ee pour laquelle

\end{itemize}
\[
\left(\mathbf{M}_{F}\right)_{i,j}=\delta_{ij}\sum_{\ell\,|\,M_{i}\in T_{\ell}}\frac{|T_{\ell}|}{4}.
\]

\begin{itemize}
\item La matrice de rigidit\'e $\mathbf{A}\in\mathbb{R}^{N_{\mathbf{u}}\times N_{\mathbf{u}}}$
est compos\'ee de $d$ blocs diagonaux \'egaux \`a $\mathbf{A}_{F}\in\mathbb{R}^{N_{F}\times N_{F}}$tels
que $\left(\mathbf{A}_{F}\right)_{i,j}=(\nabla\phi_{i},\nabla\phi_{j})_{0}$.
On obtient : 
\[
\left(\mathbf{A}_{F}\right)_{i,j}=\sum_{\ell\,|\,M_{i},M_{j}\in T_{\ell}}|T_{\ell}|^{-1}\mathbf{S}_{i,\ell}\cdot\mathbf{S}_{j,\ell},
\]
o\`u $\mathbf{S}_{i,\ell}$ est le vecteur \og face normale \fg{}
associ\'e \`a la face oppos\'ee au sommet $S_{i}$ dans le triangle $T_{\ell}$.


La matrice de rigidit\'e de la m\'ethode des \'el\'ements finis non conformes
de Crouzeix-Raviart est \'egale \`a la matrice de rigidit\'e obtenue par
la m\'ethode des volumes finis d\'ecrite dans la th\`ese de Emonot.

\item La matrice de couplage $\mathbf{B}\in\mathbb{R}^{N_{T}\times N_{\mathbf{u}}}$
est compos\'ee de $d$ blocs tels que $\mathbf{B}=(\mathbf{B}^{\beta})_{\beta=1}^{d}$,
$\mathbf{B}^{\beta}\in\mathbb{R}^{N_{T}\times N_{F}}$ et :


\[
(\mathbf{B}^{\beta})_{\ell,j}=-\mathbf{S}_{j,\ell}\cdot\mathbf{e}^{\beta}
\]


\item Dans le cas o\`u la pression est $P_{1}$, on a $p_{h}^{n}=(P_{i}^{n})_{i}\in\mathbb{R}^{N_{S}}$
et la matrice de couplage $\mathbf{B}\in\mathbb{R}^{N_{S}\times N_{\mathbf{u}}}$
est telle que : 
\[
\begin{array}{rcl}
(\mathbf{B}^{\beta})_{i,k} & = & -\sum_{\ell\,|\,M_{k},S_{i}\in T_{\ell}}(\,(d+1)d\,)^{-1}\mathbf{S}_{i,\ell}\cdot\mathbf{e}^{\beta}\end{array}.
\]

\end{itemize}

\section{\label{sec:Modeles-de-turbulence}Mod\`eles de turbulence}


\subsection{Introduction}

Les mod\`eles et m\'ethodes num\'eriques de turbulence peuvent \^etre class\'ees
en trois cat\'egories selon les \'echelles r\'esolues : (1) la m\'ethode de
simulation num\'erique directe (Direct Numerical Simulation -- DNS),
(2) la simulation des grandes \'echelles (Large Eddy Simulation -- LES)
et (3) la m\'ethode des \'equations de Navier-Stokes moyenn\'ees (Reynolds-Averaged
Navier-Stokes -- RANS). La DNS r\'esout les \'equations de Navier-Stokes
sans mod\`ele de turbulence et toutes les \'echelles spatiales et temporelles
de la turbulence sont r\'esolues. Par cons\'equent, le maillage en DNS
doit \^etre suffisamment fin pour capturer les tourbillons de tailles
s'\'etendant de la plus petite \'echelle de dissipation (\'echelle de Kolmogorov)
jusqu'\`a l'\'echelle de longueur caract\'eristique de la taille du domaine.
La th\'eorie de la turbulence montre que le nombre de points du maillage
en DNS 3D est de l'ordre de $O(\mbox{Re}^{9/4})$ o\`u $\mbox{Re}$
est le nombre de Reynolds turbulent. Les co\^uts de calcul de la DNS
sont par cons\'equent tr\`es importants pour des forts nombres de Reynolds.

Contrairement \`a la DNS, la LES r\'esout seulement les grandes structures
des \'ecoulements en filtrant les \'equations de Navier-Stokes avec un
filtre spatial et la petite \'echelle non r\'esolue est mod\'elis\'ee en utilisant
des mod\`eles de sous-grille. La gamme des \'echelles r\'esolues en LES
est beaucoup plus petite qu'en DNS et par cons\'equent les co\^uts de
calcul sont r\'eduits de fa\c con significative. On pr\'esentera les deux
principaux mod\`eles LES dans la section \ref{sec:LES}.

Enfin, les mod\`eles RANS sont compos\'es d'un ensemble d'\'equations de
Navier-Stokes moyenn\'ees, avec des mod\`eles de turbulence pour fermer
le tenseur de Reynolds suppl\'ementaire qui est induit par les fluctuations.
Les mod\`eles RANS r\'esolvent seulement l'\'ecoulement moyen aux \'echelles
macroscopiques et il s'agit de la m\'ethode la plus \'economique pour
la simulation de la turbulence. Dans la suite on pr\'esente un des mod\`eles
RANS dans la section \ref{sec:Mod=0000E8le-RANS}.


\subsection{\label{sec:LES}Simulations des grandes \'echelles}

L'approche de la turbulence bas\'ee sur la Simulation des Grandes Echelles
(SGE, ou Large Eddy Simulation -- LES) consiste \`a obtenir par r\'esolution
directe des \'equations de Navier-Stokes, les caract\'eristiques de grande
taille de la turbulence pour n'avoir \`a mod\'eliser que les mouvements
de \og petite taille \fg{}. Les contributions des grandes \'echelles
sont isol\'ees en introduisant un op\'erateur de moyenne spatiale filtr\'ee
: $\tilde{f}(\mathbf{x},\,t)=\int_{V}G(\mathbf{x},\,\mathbf{x}')f(\mathbf{x}',\,t)dV$
et toute fonction du champ de l'\'ecoulement est d\'ecompos\'ee en $f(\mathbf{x},\,t)=\tilde{f}(\mathbf{x},\,t)+f'(\mathbf{x},\,t)$
o\`u $f'(\mathbf{x},\,t)$ est la fluctuation de sous-maille.

En appliquant l'op\'erateur de moyenne filtr\'ee aux \'equations du mouvement
on obtient :

\begin{subequations}

\begin{eqnarray}
\frac{\partial\tilde{u}_{j}}{\partial x_{j}} & = & 0\label{eq:divu_LES}\\
\frac{\partial\tilde{u}_{i}}{\partial t}+\frac{\partial}{\partial x_{j}}(\widetilde{u_{i}u_{j}}) & = & \frac{1}{\rho_{0}}\frac{\partial\tilde{p}}{\partial x_{i}}+\nu\frac{\partial^{2}\tilde{u}_{i}}{\partial x_{j}\partial x_{j}}\label{eq:QDM_LES}
\end{eqnarray}


\end{subequations}

La moyenne spatiale filtr\'ee du produit $\widetilde{u_{i}u_{j}}$ est
r\'e\'ecrite sous la forme :

\[
\widetilde{u_{i}u_{j}}=\tilde{u}_{i}\tilde{u}_{j}+\tilde{L}_{ij}+\tilde{R}_{ij}
\]
o\`u :

\begin{subequations}

\begin{eqnarray}
\tilde{L}_{ij} & = & \widetilde{\tilde{u}_{i}\tilde{u}_{j}}-\tilde{u}_{i}\tilde{u}_{j}\label{eq:Tension-Leonard}\\
\tilde{R}_{ij} & = & \widetilde{\tilde{u}_{i}u_{j}'}+\widetilde{\tilde{u}_{j}u_{i}'}+\widetilde{u_{i}'u_{j}'}\label{eq:Tension-Reynolds}
\end{eqnarray}


\end{subequations}

La relation (\ref{eq:Tension-Leonard}) caract\'erise les tensions de
Leonard tandis que la relation (\ref{eq:Tension-Reynolds}) caract\'erise
les tensions de Reynolds de sous-maille. Le probl\`eme de fermeture
de la proc\'edure SGE consiste \`a d\'eterminer une relation de $\tilde{R}_{ij}$
pour obtenir la solution d'une r\'ealisation de l'\'ecoulement.


\subsubsection{Smagorinski}

Le tenseur des contraintes de sous-maille $\tilde{R}_{ij}$ peut \^etre
repris sous la forme :

\[
\tilde{R}_{ij}=\tilde{T}_{ij}+\frac{1}{3}\tilde{R}_{kk}\delta_{ij}
\]


Un des mod\`eles de sous-maille tr\`es r\'epandu est celui de Smagorinsky
qui suppose une relation lin\'eaire du tenseur anisotrope $\tilde{T}_{ij}$
avec le champ des d\'eformations filtr\'e $\tilde{S}_{ij}$ telle que
:

\[
\tilde{T}_{ij}=-2\nu_{T}\tilde{S}_{ij}
\]


La viscosit\'e tourbillonnaire des structures de sous-maille $\nu_{T}$
est choisie telle que :

\begin{equation}
\nu_{T}=(C_{S}\Delta)^{2}\sqrt{\sum_{ij}\tilde{S}_{ij}\tilde{S}_{ij}}\label{eq:nu_T_smago}
\end{equation}
o\`u $\Delta$ est l'\'epaisseur du filtre et $C_{S}$ est une constante
positive qui peut varier selon les applications. Dans \cite[p. 2203]{Weickert_etal_CAMWA2009}
les auteurs disent que la valeur peut varier de $C_{S}=0.05$ \`a $C_{S}=0.16$.
Dans \texttt{TrioCFD} la valeur peut \^etre sp\'ecifi\'ee dans le fichier
de donn\'ees d'entr\'ee. Par d\'efaut, elle est fix\'ee \`a $C_{S}=0.18$ si
elle n'est pas.


\subsubsection{LES-WALE}

Le mod\`ele alternatif est le mod\`ele LES-WALE (Wall Adaptative Local
Eddy-viscosity) \cite{Nicoud-Ducros_LES-WALE_FTC1999} :

\begin{equation}
\nu_{T}=(C_{W}\Delta)^{2}\frac{(\tilde{S}_{ij}^{d}\tilde{S}_{ij}^{d})^{3/2}}{(\tilde{S}_{ij}\tilde{S}_{ij})^{5/2}+(\tilde{S}_{ij}^{d}\tilde{S}_{ij}^{d})^{5/4}}\label{eq:nu_WALE}
\end{equation}
avec

\[
\tilde{S}_{ij}^{d}=\tilde{S}_{ik}\tilde{S}_{kj}+\tilde{\Omega}_{ik}\tilde{\Omega}_{kj}-\frac{1}{3}\delta_{ij}\left(\tilde{S}_{mn}\tilde{S}_{mn}-\tilde{\Omega}_{mn}\tilde{\Omega}_{mn}\right)
\]
o\`u $\tilde{\Omega}{}_{ij}$ est d\'efini par 

\[
\tilde{\Omega}_{ij}=\frac{1}{2}\left(\frac{\partial\tilde{u}_{i}}{\partial x_{j}}-\frac{\partial\tilde{u}_{j}}{\partial x_{i}}\right)
\]


Dans l'Eq. (\ref{eq:nu_WALE}), $\Delta$ est choisi \`a la taille de
maille. Lorsque la valeur $C_{S}$ est \'egale \`a 0.18, une valeur appropri\'ee
de $C_{w}$ est comprise entre $0.55\leq C_{w}\leq0.6$ \cite[p. 170]{Nicoud-Ducros_LES-WALE_FTC1999}.
Dans certaines conditions d'\'ecoulements d\'ecrites dans \cite[sec 3.1]{Nicoud-Ducros_LES-WALE_FTC1999},
la valeur la plus adapt\'ee est $C_{w}=0.5$, et c'ewst celle choisie
dans les simulations (\cite[pp. 191 and 192]{Nicoud-Ducros_LES-WALE_FTC1999}).


\subsection{\label{sec:Mod=0000E8le-RANS}Mod\`ele RANS}


\subsubsection{\'equations de Reynolds}

La vitesse $\mathbf{u}(\mathbf{x},\,t)$ et la pression $p(\mathbf{x},\,t)$
du champ d'\'ecoulement d'un fluide incompressible sont r\'egies ind\'ependamment
de la temp\'erature par les \'equations de continuit\'e et de la quantit\'e
de mouvement :

\begin{align}
\boldsymbol{\nabla}\cdot\mathbf{u} & =0,\label{eq:ContinuiteMP-2}\\
\rho_{0}\left[\frac{\partial\mathbf{u}}{\partial t}+\mathbf{u}\cdot\boldsymbol{\nabla}\mathbf{u}\right] & =-\boldsymbol{\nabla}p+\eta_{0}\boldsymbol{\nabla}^{2}\mathbf{u}+\rho_{0}\mathbf{F}_{v}\label{eq:DBF-1-2}
\end{align}


Lorsque la vitesse et la pression sont trait\'ees comme des fonctions
al\'eatoires de l'espace et du temps dont on d\'ecompose les valeurs instantan\'ees
en :

\begin{eqnarray*}
\mathbf{u}(\mathbf{x},\,t) & = & \overline{\mathbf{U}}(\mathbf{x},\,t)+\tilde{\mathbf{u}}(\mathbf{x},\,t)\\
p(\mathbf{x},\,t) & = & \overline{P}(\mathbf{x},\,t)+\tilde{p}(\mathbf{x},\,t)
\end{eqnarray*}
o\`u le symbole $\overline{()}$ indique l'op\'erateur de moyenne statistique
(ou moyenne d'ensemble) et le symbole $\tilde{()}$ les fluctuations
(ou \'ecarts par rapport \`a ces moyennes), les \'equations moyenn\'ees de
masse et de quantit\'e de mouvement se traduisent par \cite[sec 4 p 73--76]{Book_Chassaing}
:

\begin{align}
\boldsymbol{\nabla}\cdot\overline{\mathbf{U}} & =0,\label{eq:ContinuiteMasse_Moyennee}\\
\rho_{0}\left[\frac{\partial\overline{\mathbf{U}}}{\partial t}+\overline{\mathbf{U}}\cdot\boldsymbol{\nabla}\overline{\mathbf{U}}\right] & =\boldsymbol{\nabla}\cdot\overline{\boldsymbol{\Sigma}}+\rho_{0}\overline{\mathbf{F}}_{v}\label{eq:QDM_Moyennee}
\end{align}
avec :

\begin{equation}
\overline{\boldsymbol{\Sigma}}=-\overline{P}\mathbf{I}+2\eta_{0}\overline{\mathbf{S}}-\rho_{0}\overline{\tilde{\mathbf{u}}\tilde{\mathbf{u}}}\qquad\mbox{et}\qquad\overline{\mathbf{S}}=\frac{1}{2}(\boldsymbol{\nabla}\overline{\mathbf{U}}+\boldsymbol{\nabla}^{T}\overline{\mathbf{U}})\label{eq:TenseurReynolds}
\end{equation}


Le bilan de quantit\'e de mouvement moyenne est appel\'ee l'\'equation de
Reynolds. Dans cette \'equation, les forces de surface font appara\^itre
un terme suppl\'ementaire $-\rho_{0}\overline{\tilde{\mathbf{u}}\tilde{\mathbf{u}}}$
qui repr\'esente l'agitation turbulente. Le syst\`eme d'\'equations est
ouvert en raison de la pr\'esence des corr\'elations des vitesses fluctuantes
$\overline{\tilde{\mathbf{u}}\tilde{\mathbf{u}}}$.


\subsubsection{Mod\`ele $(\overline{k},\,\overline{\epsilon})$}

De nombreuses mod\'elisations du tenseur de Reynolds sont possibles,
mais nous nous int\'eressons ici \`a la plus classique, bas\'ee sur l\textquoteright hypoth\`ese
de Boussinesq : 

\begin{equation}
\overline{\tilde{\mathbf{u}}\tilde{\mathbf{u}}}=-\nu_{T}2\mathbf{\overline{S}}+\frac{2}{3}\overline{k}\mathbf{I}\label{eq:Hyp_Boussinesq}
\end{equation}
o\`u $\nu_{T}$ est une viscosit\'e turbulente scalaire qui traduit les
effets d\textquoteright agitation turbulente. Le terme en $\overline{k}$
au second membre s\textquoteright apparente \`a une pression par agitation
turbulente et est int\'egr\'e dans la pression $\overline{P}$. On obtient
ainsi une \'equation ferm\'ee pour la vitesse moyenne. Dans le cadre du
mod\`ele \`a deux \'equations $\overline{k}$--$\overline{\epsilon}$ qui
nous int\'eresse, une analyse dimensionnelle donne pour la viscosit\'e
turbulente :

\begin{equation}
\nu_{T}=C'_{\eta}\frac{\overline{k}^{2}}{\overline{\epsilon}}\label{eq:nu_T_RANS}
\end{equation}


Le mod\`ele $\overline{k}$--$\overline{\epsilon}$ permet de fermer
le syst\`eme d'\'equations (\ref{eq:ContinuiteMasse_Moyennee})--(\ref{eq:TenseurReynolds})
en r\'esolvant deux \'equations suppl\'ementaires, une sur l'\'energie cin\'etique
turbulente $\overline{k}$ et l'autre sur le taux de dissipation $\overline{\epsilon}$
suivantes \cite[p. 469]{Book_Chassaing} (repris de \cite[Eq. (2.2-1 et 2.2-2)]{Launder-Spalding_NumCompTurbFlow1974})
:

\begin{subequations}

\begin{eqnarray}
\frac{\partial\overline{k}}{\partial t}+\overline{U}_{j}\frac{\partial\overline{k}}{\partial x_{j}} & = & \nu_{T}\left(\frac{\partial\overline{U}_{i}}{\partial x_{j}}+\frac{\partial\overline{U}_{j}}{\partial x_{i}}\right)\frac{\partial\overline{U}_{i}}{\partial x_{j}}+\frac{\partial}{\partial x_{j}}\left[\frac{\nu_{T}}{\sigma_{k}}\frac{\partial\overline{k}}{\partial x_{j}}\right]-\overline{\epsilon}\label{eq:EnerCinTurb_k}\\
\frac{\partial\overline{\epsilon}}{\partial t}+\overline{U}_{j}\frac{\partial\overline{\epsilon}}{\partial x_{j}} & = & C_{\epsilon_{1}}\nu_{T}\frac{\overline{\epsilon}}{\overline{k}}\left(\frac{\partial\overline{U}_{i}}{\partial x_{j}}+\frac{\partial\overline{U}_{j}}{\partial x_{i}}\right)\frac{\partial\overline{U}_{i}}{\partial x_{j}}+\frac{\partial}{\partial x_{j}}\left[\frac{\nu_{T}}{\sigma_{\epsilon}}\frac{\partial\overline{\epsilon}}{\partial x_{j}}\right]-C_{\epsilon_{2}}\frac{\overline{\epsilon}^{2}}{\overline{k}}\label{eq:TauxDissip_eps}
\end{eqnarray}


\end{subequations}

Les \'equations (\ref{eq:EnerCinTurb_k}) et (\ref{eq:TauxDissip_eps})
sont des \'equations de type advection-diffusion avec des termes source.
Le terme de production d\textquoteright \'energie cin\'etique turbulente
(premier terme du membre de droite) joue un r\^ole important dans les
mod\'elisations pari\'etales. Les valeurs standards des cinq constantes
du mod\`ele $C'_{\eta}$, $C_{\epsilon_{1}}$, $C_{\epsilon_{2}}$,
$\sigma_{k}$ et $\sigma_{\epsilon}$ sont fix\'ees par d\'efaut \`a : $C'_{\eta}=0.09$,
$C_{\epsilon_{1}}=1.44$, $C_{\epsilon_{2}}=1.92$, $\sigma_{k}=1.0$
et $\sigma_{\epsilon}=1.3$. Certaines d'entre elles peuvent aussi
varier selon le type d'\'ecoulement consid\'er\'e (voir table \ref{tab:Synthese-valeurs_Coeffkeps})
:

\begin{table}
\begin{centering}
\begin{tabular}{ccccccc}
\hline 
R\'ef\'erence & $C_{\eta}'$ & $\sigma_{k}$ & $\sigma_{\epsilon}$ & $C_{\epsilon_{1}}$ & $C_{\epsilon_{2}}$ & \'ecoulement\tabularnewline
\hline 
\cite{Jones-Launder_IJHMT1972} & 0.09 & 1.0 & 1.3 & 1.55 & 2.00 & Haut Reynolds\tabularnewline
\cite{Launder-Sharma_LettHMT1974} & 0.09 & 1.0 & 1.3 & 1.44 & 1.92 & Tourbillonnant\tabularnewline
\cite{Chien_AIAA1982} & 0.09 & 1.0 & 1.3 & 1.35 & 1.92 & Bas Reynolds\tabularnewline
\cite{Fan_etal_AIAA1993} & 0.09 & 1.0 & 1.3 & 1.39 & 1.80 & Bas Reynolds\tabularnewline
\cite{Morgans-etal_Conf1999} & 0.09 & 1.0 & 1.3 & 1.60 & 1.92 & Jet\tabularnewline
\cite{Bahari-Hejazi_IJPMS2009} & 0.09 & 1.0 & 1.3 & 1.40 & 1.92 & Flottabilit\'e\tabularnewline
\hline 
\end{tabular}
\par\end{centering}

\protect\caption{\label{tab:Synthese-valeurs_Coeffkeps}Synth\`ese des valeurs des param\`etres
du mod\`ele $\overline{k}-\overline{\epsilon}$ (repris de \cite{Genty_RapportCEA_2019}).}
\end{table}



\subsubsection{Autres mod\`eles \`a deux \'equations}


\subsubsection*{Mod\`ele $(\overline{k},\,\overline{\epsilon})$-\og r\'ealisable \fg{}}

Alternativement au mod\`ele $(\overline{k},\,\overline{\epsilon})$,
d'autres mod\`eles de fermeture \`a deux \'equations existent dans la litt\'erature
: les mod\`eles $(\overline{k},\,\overline{\epsilon})$ modifi\'es (parmi
eux le $(\overline{k},\,\overline{\epsilon})$-\og r\'ealisable \fg{}
, mod\`ele pour lequel les validations sont en cours). Le mod\`ele $(\overline{k},\,\overline{\epsilon})$-\og r\'ealisable \fg{}
s'\'ecrit \cite[Eq. (22) et (23) p. 233]{Shi_etal_keps-realisable_CF1995}
:

\begin{subequations}

\begin{eqnarray}
\frac{\partial\overline{k}}{\partial t}+\overline{U}_{j}\frac{\partial\overline{k}}{\partial x_{j}} & = & \frac{\partial}{\partial x_{j}}\left(\frac{\nu_{T}}{\sigma_{k}}\frac{\partial\overline{k}}{\partial x_{j}}\right)+\left(2\nu_{T}S_{ij}-\frac{2}{3}\overline{k}\delta_{ij}\right)\frac{\partial\overline{U}_{i}}{\partial x_{j}}-\overline{\epsilon}\label{eq:Eq_k_real}\\
\frac{\partial\overline{\epsilon}}{\partial t}+\overline{U}_{j}\frac{\partial\overline{\epsilon}}{\partial x_{j}} & = & \frac{\partial}{\partial x_{j}}\left[\frac{\nu_{T}}{\sigma_{\epsilon}}\frac{\partial\overline{\epsilon}}{\partial x_{j}}\right]+C_{1}S\epsilon-C_{2}\frac{\overline{\epsilon}^{2}}{\overline{k}+\sqrt{\nu\overline{\epsilon}}}\label{eq:Eq_eps_real}
\end{eqnarray}
avec

\end{subequations}

\begin{subequations}

\begin{equation}
S=\sqrt{2S_{ij}S_{ij}},\quad C_{1}=\max\left\{ 0.43,\,\frac{\eta}{5+\eta}\right\} ,\quad\eta=\frac{S\overline{k}}{\overline{\epsilon}}\label{eq:Coeff1_real}
\end{equation}
et \cite[Eq. (19) p. 232 et (21) p. 233]{Shi_etal_keps-realisable_CF1995}
:

\begin{equation}
C_{\eta}=\frac{1}{A_{0}+A_{s}U^{(*)}\frac{\overline{k}}{\overline{\epsilon}}},\quad\mbox{o\`u }A_{0}=4,\quad A_{s}=\sqrt{6}\cos\phi,\quad\phi=\frac{1}{3}\arccos\left(\sqrt{6}W\right),\quad W=\frac{S_{ij}S_{jk}S_{ki}}{(S_{ij}S_{ij})^{3/2}}\label{eq:Coeff2_real}
\end{equation}
et une vitesse $U^{(*)}$ calcul\'ee par \cite[Eq. (20) p. 232]{Shi_etal_keps-realisable_CF1995}

\begin{equation}
U^{(*)}=\sqrt{S_{ij}S_{ij}+\tilde{\Omega}_{ij}\tilde{\Omega}_{ij}},\quad\tilde{\Omega}_{ij}=\Omega_{ij}-2\epsilon_{ijk}\omega_{k},\quad\Omega_{ij}=\overline{\Omega}_{ij}\epsilon_{ijk}\omega_{k}\label{eq:Coeff3_real}
\end{equation}
o\`u $\overline{\Omega}_{ij}$ est le taux de rotation moyen dans un
rep\`ere de r\'ef\'erence en rotation de vitesse angulaire $\omega_{k}$.

\end{subequations}

Dans \texttt{TrioCFD}, ce mod\`ele $(\overline{k},\,\overline{\epsilon})$-\og r\'ealisable \fg{}
a \'et\'e d\'evelopp\'e et valid\'e dans \cite{Angeli_Leterrier_keps-real_NT2018}.


\subsubsection{Mod\`ele \og bas Reynolds \fg{}}

Les lois de paroi, \'etablies \`a partir de r\'esultats exp\'erimentaux, permettent
d\textquoteright \'eviter de calculer la solution des \'equations de Navier-Stokes
et du mod\`ele de turbulence proche de la paroi. Le mod\`ele $(\overline{k},\,\overline{\epsilon})$,
coupl\'e \`a une loi de paroi, permet ainsi de simuler le c\oe ur de
l\textquoteright \'ecoulement en se pr\'eservant d\textquoteright un co\^ut
de calcul trop grand d\^u \`a un maillage trop fin proche de la paroi.
Cependant, ce type de mod\`ele est inadapt\'e lorsque le premier point
du maillage se trouve dans la sous-couche visqueuse ($y^{+}<+30$).
Plus le nombre de Reynolds est haut plus cette sous-couche visqueuse
est d'\'epaisseur n\'egligeable, ce qui rend le mod\`ele $(\overline{k},\,\overline{\epsilon})$
avec loi de paroi bien adapt\'e aux \'ecoulements avec de grand nombre
de Reynolds. C\textquoteright est pourquoi ce mod\`ele est appel\'e \'egalement
mod\`ele $(\overline{k},\,\overline{\epsilon})$ \og haut-Reynolds \fg{}.
Lorsque l\textquoteright on \'etudie des \'ecoulements \`a faible nombre
de Reynolds, la sous-couche visqueuse devient plus importante, ce
qui rend l\textquoteright utilisation d'une loi de paroi inadapt\'ee.
\`a faible nombre de Reynolds, il peut \^etre pr\'ef\'erable d\textquoteright utiliser
des mod\`eles, appel\'es ainsi \og bas-Reynolds \fg{}, qui font d\'esormais
appel \`a des fonctions d\textquoteright amortissement et des termes
d\'ependants de la discr\'etisation pour prendre en compte la r\'esolution
num\'erique de la sous-couche visqueuse. Ces mod\`eles permettent \'egalement
d\textquoteright \'etudier l\textquoteright ensemble de l\textquoteright \'ecoulement
notamment proche de la paroi (par exemple \`a des effets de recirculation,
de d\'ecollement pour des g\'eom\'etries complexes). Le mod\`ele $(\overline{k},\,\overline{\epsilon})$
bas-Reynolds laisse l\textquoteright \'equation de transport de $\overline{k}$
du mod\`ele $(\overline{k},\,\overline{\epsilon})$ de base inchang\'ee
mais modifie celle de $\overline{\epsilon}$ par adjonction de termes
d\textquoteright att\'enuation dans la zone proche de la paroi o\`u le
nombre de Reynolds est localement plus faible. Du fait d\textquoteright un
maillage de paroi important, ce mod\`ele est donc plus co\^uteux que le
mod\`ele $(\overline{k},\,\overline{\epsilon})$ standard. Il existe
plusieurs mod\`eles bas-Reynolds dans la litt\'erature \cite{Launder-Sharma_LettHMT1974,Launder-Spalding_NumCompTurbFlow1974,Jones-Launder_IJHMT1972,Lam-Bremhorst_JFE1981}.
La forme g\'en\'erale de ces mod\`eles peut s'\'ecrire sous la forme (\'ecriture
inspir\'ee de \cite{Jones-Launder_IJHMT1972}) :

\begin{subequations}

\begin{eqnarray}
\frac{\partial\overline{k}}{\partial t}+\overline{U}_{j}\frac{\partial\overline{k}}{\partial x_{j}} & = & \nu_{T}\left(\frac{\partial\overline{U}_{i}}{\partial x_{j}}+\frac{\partial\overline{U}_{j}}{\partial x_{i}}\right)\frac{\partial\overline{U}_{i}}{\partial x_{j}}+\frac{\partial}{\partial x_{j}}\left[\left(\nu+\frac{\nu_{T}}{\sigma_{k}}\right)\frac{\partial\overline{k}}{\partial x_{j}}\right]-\overline{\epsilon}-\overline{\mathcal{K}}\label{eq:EnerCinTurb_k_LowRe}\\
\frac{\partial\overline{\epsilon}}{\partial t}+\overline{U}_{j}\frac{\partial\overline{\epsilon}}{\partial x_{j}} & = & C_{\epsilon_{1}}\nu_{T}\frac{\overline{\epsilon}}{\overline{k}}f_{\epsilon_{1}}\left(\frac{\partial\overline{U}_{i}}{\partial x_{j}}+\frac{\partial\overline{U}_{j}}{\partial x_{i}}\right)\frac{\partial\overline{U}_{i}}{\partial x_{j}}+\frac{\partial}{\partial x_{j}}\left[\left(\nu+\frac{\nu_{T}}{\sigma_{\epsilon}}\right)\frac{\partial\overline{\epsilon}}{\partial x_{j}}\right]-C_{\epsilon_{2}}f_{\epsilon_{2}}\frac{\overline{\epsilon}^{2}}{\overline{k}}+\overline{\mathcal{E}}\label{eq:TauxDissip_eps_LowRe}\\
\nu_{T} & = & C_{\eta}f_{\eta}\frac{\overline{k}^{2}}{\overline{\epsilon}}\label{eq:Nu_T_LowRe}
\end{eqnarray}


\end{subequations}


\subsubsection*{Mod\`ele \og bas-Reynolds \fg{} de Launder \& Spalding \cite{Launder-Spalding_NumCompTurbFlow1974}}

Pour ce mod\`ele, les termes $\overline{\mathcal{K}}$ et $\overline{\mathcal{E}}$
s'\'ecrivent \cite[Eqs. (2.3-4) et (2.3-5)]{Launder-Spalding_NumCompTurbFlow1974}
:

\begin{equation}
\overline{\mathcal{K}}=2\nu\left(\frac{\partial\overline{k}^{1/2}}{\partial x_{j}}\right)^{2},\quad\overline{\mathcal{E}}=2.0\nu\nu_{T}\left(\frac{\partial^{2}\overline{U}_{i}}{\partial x_{j}\partial x_{l}}\right)\label{eq:K-E_Model1}
\end{equation}
et les fonctions $f_{\epsilon_{1}}$, $f_{\epsilon_{2}}$, $f_{\eta}$
sont donn\'ees par \cite[Eqs. (2.3-6) et (2.3-7)]{Launder-Spalding_NumCompTurbFlow1974}
:

\[
f_{\epsilon_{1}}=1,\quad f_{\epsilon_{2}}=1.0-0.3e^{-Re_{t}^{2}},\quad f_{\eta}=e^{-2.5/(1+Re_{t}/50)}
\]
o\`u $Re_{t}$ est le nombre de Reynolds turbulent d\'efini par $Re_{t}=\overline{k}^{2}/\nu\overline{\epsilon}$.
Les valeurs des coefficients empiriques sont $C_{\eta}=0.09$, $C_{\epsilon_{1}}=1.44$,
$C_{\epsilon_{2}}=1.92$, $\sigma_{k}=1.0$ et $\sigma_{\epsilon}=1.3$.


\subsubsection*{Mod\`ele de Jones \& Launder \cite{Jones-Launder_IJHMT1972}}

Ce mod\`ele est formul\'e dans la r\'ef\'erence originale \cite{Jones-Launder_IJHMT1972}
en ne consid\'erant qu'une seule d\'eriv\'ee spatiale $\partial/\partial y$
dans les \'equations (\ref{eq:EnerCinTurb_k_LowRe}) et (\ref{eq:TauxDissip_eps_LowRe}).
Dans ce cas, les termes $\overline{\mathcal{K}}$ et $\overline{\mathcal{E}}$
de ce mod\`ele s'\'ecrivent \cite[Eqs. (8) et (9)]{Jones-Launder_IJHMT1972}
:

\[
\overline{\mathcal{K}}=2\nu\left(\frac{\partial\overline{k}^{1/2}}{\partial y}\right)^{2},\quad\overline{\mathcal{E}}=2.0\nu\nu_{T}\left(\frac{\partial^{2}\overline{U}_{i}}{\partial y^{2}}\right)
\]
qui, en 3D s'\'ecrivent de mani\`ere identique aux deux relations de l'\'equation
(\ref{eq:K-E_Model1}). les fonctions $f_{\epsilon_{1}}$, $f_{\epsilon_{2}}$,
$f_{\eta}$ sont donn\'ees par \cite[Eq. (12)]{Jones-Launder_IJHMT1972}
:

\[
f_{\epsilon_{1}}=1,\quad f_{\epsilon_{2}}=1.0-0.3e^{-Re_{t}^{2}},\quad f_{\eta}=e^{-2.5/(1+Re_{t}/50)}
\]


Les param\`etres empiriques du mod\`ele sont donn\'es par \cite[Table 1]{Jones-Launder_IJHMT1972}
: $C_{\eta}=0.09$, $C_{\epsilon_{1}}=1.55$, $C_{\epsilon_{2}}=2$,
$\sigma_{k}=1.0$ et $\sigma_{\epsilon}=1.3$. Ce mod\`ele ne se diff\'erencie
du pr\'ec\'edent (celui de Launder \& Spalding) que par les valeurs des
param\`etres $C_{\epsilon_{1}}$et $C_{\epsilon_{2}}$.


\subsubsection*{Mod\`ele de Lam \& Bremhorst \cite{Lam-Bremhorst_JFE1981}}

Dans ce mod\`ele les termes $\overline{\mathcal{K}}$ et $\overline{\mathcal{E}}$
dans les \'equations (\ref{eq:EnerCinTurb_k_LowRe}) et (\ref{eq:TauxDissip_eps_LowRe})
sont nuls et les fonctions $f_{\epsilon_{1}}$, $f_{\epsilon_{2}}$,
$f_{\eta}$ sont donn\'ees par \cite[Eq. (11), (12) et (13)]{Lam-Bremhorst_JFE1981}
:

\[
f_{\epsilon_{1}}=1+\left(\frac{A_{c}}{f_{\eta}}\right)^{3},\quad f_{\epsilon_{2}}=1-e^{-Re_{t}^{2}},\quad f_{\eta}=(1-e^{-A_{\eta}Re_{y}})^{2}\left(1+\frac{A_{t}}{Re_{t}}\right)
\]
o\`u $Re_{y}$ est le nombre de Reynolds turbulent qui varie avec la
distance $y$ \`a la paroi et qui est d\'efini par $Re_{y}=\overline{k}^{1/2}y/\nu$.
Les coefficients $A_{\eta}$, $A_{t}$ et $A_{c}$ sont cal\'es en comparant
les r\'esultats num\'eriques \`a des mesures exp\'erimentales. Les valeurs
obtenues sont (\cite[sec. 3.1]{Lam-Bremhorst_JFE1981}) $A_{\eta}=0.0165$,
$A_{t}=20.5$, et $A_{c}=0.05$. Les valeurs des autres coefficients
sont $C_{\epsilon_{1}}=1.44$ et $C_{\epsilon_{2}}=1.92$.


\subsubsection*{Mod\`ele de Launder \& Sharma \cite{Launder-Sharma_LettHMT1974}}

Ce mod\`ele est formul\'e pour les \'ecoulements tourbillonnants (disque
en rotation) pour lesquels les coordonn\'ees ind\'ependantes sont $r$
et $y$ o\`u $r$ est la distance radiale \`a l'axe du disque et $y$
est la distance normale \`a la surface du disque. Dans ce cas, des termes
suppl\'ementaires qui impliquent le gradient de $V_{\theta}/r$ apparaissent
dans les \'equations de $\overline{k}$ et de $\overline{\epsilon}$.
Les fonctions $f_{\epsilon_{2}}$ et $f_{\eta}$ et les constantes
sont l\'eg\`erement diff\'erentes \cite[Eqs. (5), (6)]{Launder-Sharma_LettHMT1974}
:

\[
f_{\epsilon_{2}}=1.0-0.3e^{-Re_{t}^{2}},\quad f_{\eta}=e^{-3.4/(1+Re_{t}/50)^{2}}
\]
et \cite[Eqs. (5), (6) et valeurs au-dessous]{Launder-Sharma_LettHMT1974}
$C_{\eta}=0.09$, $C_{\epsilon_{1}}=1.44$, $C_{\epsilon_{2}}=1.92$,
$\sigma_{k}=1.0$ et $\sigma_{\epsilon}=1.3$.


\subsubsection*{Dans \texttt{TrioCFD}}

Dans \texttt{TrioCFD}, les mod\`eles de turbulence en $(\overline{k},\,\overline{\epsilon})$
\og bas Reynolds \fg{} de Jones \& Launder \cite{Jones-Launder_IJHMT1972}
et celui de Lam \& Bremhorst \cite{Lam-Bremhorst_JFE1981} ont \'et\'e
mis en \oe uvre dans \cite{Peybernes_LowReynolds_NT2016}. Les fonctions
et param\`etres de Launder \& Sharma \cite{Launder-Sharma_LettHMT1974}
sont \'egalement disponibles.


\subsection{Lois de parois}

Dans le cadre des \'etudes de turbulence, en particulier pour des cas
avec des g\'eom\'etries complexes, la r\'esolution des ph\'enom\`enes physiques
ayant lieu en proche paroi est g\'en\'eralement co\^uteuse en terme de temps
de calcul car elle demande un maillage tr\`es fin dans ces zones o\`u
les gradients de vitesse et de temp\'erature sont tr\`es \'elev\'es. Or il
est primordial de pr\'edire correctement ces ph\'enom\`enes car par simple
conservation du d\'ebit, une mauvaise description pari\'etale de la vitesse
entrainera une mauvaise description au c\oe ur de l\textquoteright \'ecoulement.
Plut\^ot que de faire porter l\textquoteright effort sur une r\'esolution
fine, il est alors classique de faire porter l\textquoteright effort
sur une mod\'elisation du gradient de vitesse pari\'etal, qui permet de
conserver un maillage relativement grossier \`a la paroi. Ces approches
sont connues sous le nom de " lois de paroi " ou " traitement de paroi
" (on rencontre tr\`es souvent les termes anglais de " wall functions
" et " wall treatment ") et permettent une diminution notable des
temps de simulation. Elles sont depuis longtemps int\'egr\'ees \`a la plupart
des codes de calcul industriels.

Les lois disponibles avec le mod\`ele $\overline{k}$--$\overline{\epsilon}$
dans \texttt{TrioCFD} sont formul\'ees de mani\`ere \`a d\'ecrire contin\^ument
toute la couche limite. Pour la vitesse adimensionn\'ee, la loi de paroi
de Reichardt est utilis\'ee \cite{Reichardt1940} :

\[
U^{+}=\frac{1}{\kappa}\ln(1+\kappa y^{+})+A\left(1-e^{-y^{+}/11}-\frac{y^{+}}{11}e^{-y^{+}/3}\right)
\]
avec

\[
A=\frac{1}{\kappa}\ln\left(\frac{E}{\kappa}\right)
\]


Les valeurs des constantes sont $\kappa=0.415$ et $E=9.11$. Asymptotiquement,
on retrouve le comportement lin\'eaire lorsque $y^{+}$ tend vers z\'ero,
et le comportement logarithmique lorsque $y^{+}$ devient " grand
". De plus, cette loi donne une description raisonnable de la zone
tampon. Les quantit\'es $\overline{k}$ et $\overline{\epsilon}$ sont
d\'ecrites, pour tout $y^{+}$, par :

\begin{eqnarray*}
\overline{k}^{+} & = & 0.07y^{+2}e^{-y^{+}/9}+\frac{1}{\sqrt{C_{\eta}}}\left(1-e^{-y^{+}/20}\right)^{2}\\
\overline{\epsilon}^{+} & = & \frac{1}{\kappa(y^{+4}+15^{4})^{1/4}}
\end{eqnarray*}


Ces formulations respectent le comportement g\'en\'eralement admis \`a la
paroi, \`a savoir :

\[
\overline{k}(y=0)=0\,;\qquad\frac{d\overline{k}}{dy}(y=0)=0\,;\qquad\frac{d\overline{\epsilon}}{dy}(y=0)=0
\]



\subsubsection{Autres mod\`eles de turbulence dans \texttt{TrioCFD}}

Signalons que la loi d\'evelopp\'ee par Ciofalo et Collins \cite{Ciofalo-Collins_NHTB1989}
est \'egalement disponible mais uniquement pour la discr\'etisation VDF
de \texttt{TrioCFD} (maillages cart\'esiens). Il est \'egalement possible
de modifier la valeur des constantes de la loi logarithmique, ou d\textquoteright imposer
la vitesse de frottement, mais cela semble assez peu utile dans les
\'etudes industrielles. Concernant la LES, la loi de Werner et Wengle
est \'egalement impl\'ement\'ee \cite{Werner-Wengle_Proc1991} ainsi que
l\textquoteright approche TBLE (Thin Boundary Layer Equation).


\subsubsection*{Mod\`ele $(\overline{k},\,\overline{\omega})$ \`a venir}

D'autres mod\`eles de type $(\overline{k},\,\overline{\omega})$ existent
dans la litt\'erature tels que les mod\`eles $(\overline{k},\,\overline{\omega})$
et $(\overline{k},\,\overline{\epsilon})$-SST (Shear Stress Transport).
Les syst\`emes d'\'equations de ces mod\`eles sont pr\'esent\'es dans la r\'ef\'erence
\cite[sec 2.3.6 pp. 701--705]{Argyropoulos-Markatos_ReviewTurbulence_AMM2015}
de laquelle on reprend le mod\`ele $(\overline{k},\,\omega)$ pour lequel
la viscosit\'e turbulence est :

\[
\nu_{T}=\frac{\overline{k}}{\tilde{\omega}},\qquad\tilde{\omega}=\max\left\{ \omega,\,C_{lim}\sqrt{\frac{2S_{ij}S_{ij}}{\beta^{*}}}\right\} ,\qquad\mbox{avec }C_{lim}=\frac{7}{8}
\]


Le mod\`ele s'\'ecrit \cite[p. 702]{Argyropoulos-Markatos_ReviewTurbulence_AMM2015}
(repris de \cite{Wilcox_AIAA1988}) :

\begin{eqnarray*}
\frac{\partial\overline{k}}{\partial t}+\overline{U}_{j}\frac{\partial\overline{k}}{\partial x_{j}} & = & \frac{\partial}{\partial x_{j}}\left[\left(\nu+\sigma^{*}\frac{\overline{k}}{\omega}\right)\frac{\partial\overline{k}}{\partial x_{j}}\right]-\beta^{*}\overline{k}\omega+\tau_{ij}\frac{\partial\overline{U}_{i}}{\partial x_{j}}\\
\frac{\partial\omega}{\partial t}+\overline{U}_{j}\frac{\partial\omega}{\partial x_{j}} & = & \frac{\partial}{\partial x_{j}}\left[\left(\nu+\sigma\frac{\overline{k}}{\omega}\right)\frac{\partial\omega}{\partial x_{j}}\right]-\beta\omega^{2}+\frac{\sigma_{d}}{\omega}\frac{\partial\overline{k}}{\partial x_{j}}\frac{\partial\omega}{\partial x_{j}}+a\frac{\omega}{\overline{k}}\tau_{ij}\frac{\partial\overline{U}_{i}}{\partial x_{j}}
\end{eqnarray*}
avec :

\[
\sigma_{d}=\begin{cases}
0 & \mbox{si }\frac{\partial\overline{k}}{\partial x_{j}}\frac{\partial\omega}{\partial x_{j}}\leq0\\
\sigma_{d0} & \mbox{si }\frac{\partial\overline{k}}{\partial x_{j}}\frac{\partial\omega}{\partial x_{j}}>0
\end{cases},\qquad f_{\beta}=\frac{1+85\chi_{\omega}}{1+100\chi_{\omega}},\qquad\chi_{\omega}=\left|\frac{\Omega_{ij}\Omega_{jk}S_{ki}}{(\beta^{*}\omega)^{3}}\right|,\qquad\Omega_{ij}=\frac{1}{2}\left(\frac{\partial\overline{U}_{i}}{\partial x_{j}}-\frac{\partial\overline{U}_{j}}{\partial x_{i}}\right)
\]


Les coefficients sont les suivants :

\[
a=0.52,\quad\beta=\beta_{0}f_{\beta},\quad\beta_{0}=0.0708,\quad\beta^{*}=0.09,\quad\sigma=0.5,\quad\sigma^{*}=0.6,\quad\sigma_{d0}=0.125
\]



\section{Perspectives}

Plusieurs perspectives d'am\'elioration de ce document initial sont
d'ores et d\'ej\`a \`a pr\'evoir. D'une part, il s'agit de compl\'eter et de
d\'etailler les mod\`eles physiques des cas tests de validation d\'ej\`a existants
tels que les mod\`eles d'\'ecoulement quasi-compressibles (mod\`eles de
type \og bas Mach \fg{}) ou encore les mod\`eles qui impliquent des
couplages avec les \'equations du transport d'esp\`eces. D'autre part,
les sections seront compl\'et\'ees avec les m\'ethodes num\'eriques d\'edi\'ees
\`a la r\'esolution des mod\`eles physiques.

\appendix

\section{\label{sec:Annexe_ListeValid}Annexe : liste des fiches de validation
dans \texttt{TrioCFD}}

Cette annexe pr\'esente deux tables qui regroupent l'ensemble des fiches
de validation disponibles dans TrioCFD. Ce travail de synth\`ese a \'et\'e
r\'ealis\'e sur la version 1.7.5. Le fichier \texttt{Excel} est accessible
dans le r\'epertoire \texttt{/data/tmptrust/validation/main\_Validation175}
qui contient \'egalement le fichier PDF \texttt{Synthese\_fiches\_validation\_V2.pdf}
dont les deux tables ci-dessous en sont une synth\`ese. Dans la derni\`ere
colonne des deux tables, les mots cl\'es suivants sont utilis\'es :

\begin{table}[H]
\begin{centering}
\begin{tabular}{lll}
\hline 
\textbf{\footnotesize{}Mot cl\'e} &  & \textbf{\footnotesize{}Correspondance}\tabularnewline
\hline 
\texttt{\footnotesize{}C\_d\_t} &  & \texttt{\footnotesize{}Convection\_diffusion\_temperature}\tabularnewline
\texttt{\footnotesize{}C\_d\_t\_t} &  & \texttt{\footnotesize{}Convection\_diffusion\_temperature\_turbulent}\tabularnewline
\texttt{\footnotesize{}C\_d\_c\_t} &  & \texttt{\footnotesize{}Convection\_diffusion\_concentration\_turbulent}\tabularnewline
\texttt{\footnotesize{}C\_d\_ch\_QC} &  & \texttt{\footnotesize{}Convection\_diffusion\_chaleur\_QC}\tabularnewline
\texttt{\footnotesize{}C\_d\_ch\_t\_QC} &  & \texttt{\footnotesize{}Convection\_diffusion\_chaleur\_turbulent\_QC}\tabularnewline
\texttt{\footnotesize{}pb\_thermohyd\_t\_s\_p} &  & \texttt{\footnotesize{}pb\_thermohydraulique\_turbulent\_scalaires\_passifs}\tabularnewline
\texttt{\footnotesize{}pb\_thermohyd\_t\_QC} &  & \texttt{\footnotesize{}pb\_thermohydraulique\_turbulent\_QC}\tabularnewline
\hline 
\end{tabular}
\par\end{centering}

\protect\caption{Correspondance des mots cl\'es utilis\'es dans les tables \ref{tab:FichesValid1}
et \ref{tab:FichesValid2}.}


\end{table}


\begin{sidewaystable}
\protect\caption{\label{tab:FichesValid1}Liste des cas tests et des fiches de validations
regroup\'ees par th\`emes}


\begin{centering}
\texttt{\scriptsize{}}%
\begin{tabular}{|l|l|l|l|l|}
\hline 
 & \texttt{\textbf{\scriptsize{}Th\`emes}} & \texttt{\textbf{\scriptsize{}Nb}} & \texttt{\textbf{\scriptsize{}Problem}} & \texttt{\textbf{\scriptsize{}Equations}}\tabularnewline
\hline 
\hline 
\texttt{\scriptsize{}1.} & \texttt{\scriptsize{}Conduction} & \texttt{\scriptsize{}3} & \texttt{\scriptsize{}}%
\begin{tabular}{ll}
\texttt{\scriptsize{}Keyword} & \texttt{\scriptsize{}Nb}\tabularnewline
\texttt{\scriptsize{}pb\_conduction} & \texttt{\scriptsize{}3}\tabularnewline
\texttt{\scriptsize{}~~~+ pb\_thermohydraulique} & \texttt{\scriptsize{}1}\tabularnewline
 & \tabularnewline
\texttt{\scriptsize{}~~~+ pb\_thermohydraulique\_turbulent} & \texttt{\scriptsize{}1}\tabularnewline
 & \tabularnewline
\end{tabular} & \texttt{\scriptsize{}}%
\begin{tabular}{ll}
\texttt{\scriptsize{}Keyword} & \texttt{\scriptsize{}Nb}\tabularnewline
\texttt{\scriptsize{}Conduction +} & \texttt{\scriptsize{}3}\tabularnewline
\texttt{\scriptsize{}}%
\begin{tabular}{l}
\texttt{\scriptsize{}Navier\_Stokes\_standard}\tabularnewline
\texttt{\scriptsize{}~~~+ C\_d\_t}\tabularnewline
\end{tabular} & \texttt{\scriptsize{}1}\tabularnewline
\texttt{\scriptsize{}}%
\begin{tabular}{l}
\texttt{\scriptsize{}Navier\_Stokes\_turbulent}\tabularnewline
\texttt{\scriptsize{}~~~+ C\_d\_t\_t}\tabularnewline
\end{tabular} & \texttt{\scriptsize{}1}\tabularnewline
 & \tabularnewline
\end{tabular}\tabularnewline
\hline 
\texttt{\scriptsize{}2.} & \texttt{\scriptsize{}}%
\begin{tabular}{l}
\texttt{\scriptsize{}Incompressible laminar}\tabularnewline
\texttt{\scriptsize{}pipe and channel flow}\tabularnewline
\end{tabular}\texttt{\scriptsize{} } & \texttt{\scriptsize{}3} & \texttt{\scriptsize{}}%
\begin{tabular}{ll}
\texttt{\scriptsize{}Keyword} & \texttt{\scriptsize{}Nb}\tabularnewline
\texttt{\scriptsize{}pb\_thermohydraulique} & \texttt{\scriptsize{}3}\tabularnewline
\texttt{\scriptsize{}~~~+ pb\_conduction} & \texttt{\scriptsize{}2}\tabularnewline
 & \tabularnewline
\end{tabular} & \texttt{\scriptsize{}}%
\begin{tabular}{ll}
\texttt{\scriptsize{}Keyword} & \texttt{\scriptsize{}Nb}\tabularnewline
\texttt{\scriptsize{}}%
\begin{tabular}{l}
\texttt{\scriptsize{}Navier\_Stokes\_standard}\tabularnewline
\texttt{\scriptsize{}~~~+ C\_d\_t}\tabularnewline
\end{tabular} & \texttt{\scriptsize{}3}\tabularnewline
\texttt{\scriptsize{}}%
\begin{tabular}{l}
\texttt{\scriptsize{}~~~+ Conduction}\tabularnewline
\end{tabular} & \texttt{\scriptsize{}2}\tabularnewline
\end{tabular}\tabularnewline
\hline 
\texttt{\scriptsize{}3.} & \texttt{\scriptsize{}}%
\begin{tabular}{l}
\texttt{\scriptsize{}Incompressible turbulent}\tabularnewline
\texttt{\scriptsize{}pipe and channel flow}\tabularnewline
\end{tabular}\texttt{\scriptsize{} } & \texttt{\scriptsize{}45} & \texttt{\scriptsize{}}%
\begin{tabular}{ll}
\texttt{\scriptsize{}Keyword} & \texttt{\scriptsize{}Nb}\tabularnewline
\texttt{\scriptsize{}pb\_thermohydraulique\_turbulent} & \texttt{\scriptsize{}19}\tabularnewline
\texttt{\scriptsize{}pb\_hydraulique\_turbulent} & \texttt{\scriptsize{}22}\tabularnewline
\texttt{\scriptsize{}pb\_thermohydraulique\_turbulent} & \texttt{\scriptsize{}4}\tabularnewline
\texttt{\scriptsize{}~~~~~+ pb\_conduction} & \tabularnewline
\end{tabular} & \texttt{\scriptsize{}}%
\begin{tabular}{ll}
\texttt{\scriptsize{}Keyword} & \texttt{\scriptsize{}Nb}\tabularnewline
\texttt{\scriptsize{}Navier\_Stokes\_turbulent} & \texttt{\scriptsize{}45}\tabularnewline
\texttt{\scriptsize{}+ C\_d\_t\_t} & \texttt{\scriptsize{}24}\tabularnewline
\texttt{\scriptsize{}+ Conduction} & \texttt{\scriptsize{}2}\tabularnewline
 & \tabularnewline
\end{tabular}\tabularnewline
\hline 
\texttt{\scriptsize{}4.} & \texttt{\scriptsize{}Flow through curved pipes} & \texttt{\scriptsize{}2} & \texttt{\scriptsize{}}%
\begin{tabular}{ll}
\texttt{\scriptsize{}Keyword} & \texttt{\scriptsize{}Nb}\tabularnewline
\texttt{\scriptsize{}pb\_hydraulique\_turbulent} & \texttt{\scriptsize{}1}\tabularnewline
\texttt{\scriptsize{}pb\_thermohydraulique\_turbulent} & \texttt{\scriptsize{}1}\tabularnewline
\end{tabular} & \texttt{\scriptsize{}}%
\begin{tabular}{ll}
\texttt{\scriptsize{}Keyword} & \texttt{\scriptsize{}Nb}\tabularnewline
\texttt{\scriptsize{}Navier\_Stokes\_turbulent} & \texttt{\scriptsize{}2}\tabularnewline
\texttt{\scriptsize{}+ C\_d\_t\_t} & \texttt{\scriptsize{}1}\tabularnewline
\end{tabular}\tabularnewline
\hline 
\texttt{\scriptsize{}5.} & \texttt{\scriptsize{}Impinging or submerged jet} & \texttt{\scriptsize{}6} & \texttt{\scriptsize{}}%
\begin{tabular}{ll}
\texttt{\scriptsize{}Keyword} & \texttt{\scriptsize{}Nb}\tabularnewline
\texttt{\scriptsize{}pb\_hydraulique\_concentration\_turbulent} & \texttt{\scriptsize{}3}\tabularnewline
\texttt{\scriptsize{}pb\_thermohydraulique\_turbulent} & \texttt{\scriptsize{}2}\tabularnewline
\texttt{\scriptsize{}??} & \texttt{\scriptsize{}1}\tabularnewline
 & \tabularnewline
\end{tabular} & \texttt{\scriptsize{}}%
\begin{tabular}{ll}
\texttt{\scriptsize{}Keyword} & \texttt{\scriptsize{}Nb}\tabularnewline
\texttt{\scriptsize{}Navier\_Stokes\_turbulent} & \texttt{\scriptsize{}5}\tabularnewline
\texttt{\scriptsize{}+ C\_d\_c\_t} & \texttt{\scriptsize{}3}\tabularnewline
\texttt{\scriptsize{}+ C\_d\_t\_t} & \texttt{\scriptsize{}2}\tabularnewline
\texttt{\scriptsize{}??} & \texttt{\scriptsize{}1}\tabularnewline
\end{tabular}\tabularnewline
\hline 
\texttt{\scriptsize{}6.} & \texttt{\scriptsize{}Thermal stratification} & \texttt{\scriptsize{}5} & \texttt{\scriptsize{}}%
\begin{tabular}{ll}
\texttt{\scriptsize{}Keyword} & \texttt{\scriptsize{}Nb}\tabularnewline
\texttt{\scriptsize{}pb\_thermohydraulique\_turbulent} & \texttt{\scriptsize{}2}\tabularnewline
 & \tabularnewline
\texttt{\scriptsize{}pb\_thermohydraulique} & \texttt{\scriptsize{}2}\tabularnewline
 & \tabularnewline
\texttt{\scriptsize{}??} & \texttt{\scriptsize{}1}\tabularnewline
\end{tabular} & \texttt{\scriptsize{}}%
\begin{tabular}{ll}
\texttt{\scriptsize{}Keyword} & \texttt{\scriptsize{}Nb}\tabularnewline
\texttt{\scriptsize{}}%
\begin{tabular}{l}
\texttt{\scriptsize{}Navier\_Stokes\_standard}\tabularnewline
\texttt{\scriptsize{}~~~+ C\_d\_t}\tabularnewline
\end{tabular} & \texttt{\scriptsize{}1}\tabularnewline
\texttt{\scriptsize{}}%
\begin{tabular}{l}
\texttt{\scriptsize{}Navier\_Stokes\_turbulent}\tabularnewline
\texttt{\scriptsize{}~~~+ C\_d\_t\_t}\tabularnewline
\end{tabular} & \texttt{\scriptsize{}3}\tabularnewline
\texttt{\scriptsize{}??} & \texttt{\scriptsize{}1}\tabularnewline
\end{tabular}\tabularnewline
\hline 
\texttt{\scriptsize{}7.} & \texttt{\scriptsize{}Natural/mixed convection} & \texttt{\scriptsize{}7} & \texttt{\scriptsize{}}%
\begin{tabular}{ll}
\texttt{\scriptsize{}Keyword} & \texttt{\scriptsize{}Nb}\tabularnewline
\texttt{\scriptsize{}pb\_thermohydraulique\_turbulent} & \texttt{\scriptsize{}1 (+1)}\tabularnewline
\texttt{\scriptsize{}pb\_thermohydraulique} & \texttt{\scriptsize{}3 (+1)}\tabularnewline
\texttt{\scriptsize{}pb\_thermohydraulique\_QC} & \texttt{\scriptsize{}1}\tabularnewline
\texttt{\scriptsize{}}%
\begin{tabular}{l}
\texttt{\scriptsize{}pb\_thermohydraulique}\tabularnewline
\texttt{\scriptsize{}~~~~+ pb\_conduction }\tabularnewline
\texttt{\scriptsize{}~~~~+ pb\_couple\_rayonnement}\tabularnewline
\end{tabular} & \texttt{\scriptsize{}1}\tabularnewline
\end{tabular} & \texttt{\scriptsize{}}%
\begin{tabular}{ll}
\texttt{\scriptsize{}Keyword} & \texttt{\scriptsize{}Nb}\tabularnewline
\texttt{\scriptsize{}Navier\_Stokes\_standard} & \texttt{\scriptsize{}4 (+1)}\tabularnewline
\texttt{\scriptsize{}Navier\_Stokes\_turbulent} & \texttt{\scriptsize{}1 (+1)}\tabularnewline
\texttt{\scriptsize{}Navier\_Stokes\_QC} & \texttt{\scriptsize{}1}\tabularnewline
\texttt{\scriptsize{}+ C\_d\_t} & \texttt{\scriptsize{}3 (+1)}\tabularnewline
\texttt{\scriptsize{}+ C\_d\_t\_t} & \texttt{\scriptsize{}1 (+1)}\tabularnewline
\texttt{\scriptsize{}+ C\_d\_ch\_QC} & \texttt{\scriptsize{}1}\tabularnewline
\end{tabular}\tabularnewline
\hline 
\end{tabular}
\par\end{centering}{\scriptsize \par}

\pagebreak
\end{sidewaystable}


\begin{sidewaystable}
\protect\caption{\label{tab:FichesValid2}Liste des cas tests et des fiches de validations
regroup\'ees par th\`emes (suite)}


\centering{}\texttt{\scriptsize{}}%
\begin{tabular}{|l|l|l|l|l|}
\hline 
 & \texttt{\textbf{\scriptsize{}Th\`emes}} & \texttt{\textbf{\scriptsize{}Nb}} & \texttt{\textbf{\scriptsize{}Problem}} & \texttt{\textbf{\scriptsize{}Equations}}\tabularnewline
\hline 
\hline 
\texttt{\scriptsize{}8.} & \texttt{\scriptsize{}Obstacles in axial or cross flow} & \texttt{\scriptsize{}9} & \texttt{\scriptsize{}}%
\begin{tabular}{ll}
\texttt{\scriptsize{}Keyword} & \texttt{\scriptsize{}Nb}\tabularnewline
\texttt{\scriptsize{}pb\_hydraulique} & \texttt{\scriptsize{}1}\tabularnewline
\texttt{\scriptsize{}pb\_hydraulique\_turbulent} & \texttt{\scriptsize{}5}\tabularnewline
\texttt{\scriptsize{}pb\_thermohydraulique\_turbulent} & \texttt{\scriptsize{}1}\tabularnewline
 & \tabularnewline
\texttt{\scriptsize{}??} & \texttt{\scriptsize{}2}\tabularnewline
\end{tabular} & \texttt{\scriptsize{}}%
\begin{tabular}{ll}
\texttt{\scriptsize{}Keyword} & \texttt{\scriptsize{}Nb}\tabularnewline
\texttt{\scriptsize{}Navier\_Stokes\_standard} & \texttt{\scriptsize{}1}\tabularnewline
\texttt{\scriptsize{}Navier\_Stokes\_turbulent} & \texttt{\scriptsize{}5}\tabularnewline
\texttt{\scriptsize{}}%
\begin{tabular}{l}
\texttt{\scriptsize{}Navier\_Stokes\_turbulent}\tabularnewline
\texttt{\scriptsize{}+ C\_d\_t\_t}\tabularnewline
\end{tabular} & \texttt{\scriptsize{}1}\tabularnewline
\texttt{\scriptsize{}??} & \texttt{\scriptsize{}2}\tabularnewline
\end{tabular}\tabularnewline
\hline 
\texttt{\scriptsize{}9.} & \texttt{\scriptsize{}Diffusor} & \texttt{\scriptsize{}1} & \texttt{\scriptsize{}}%
\begin{tabular}{ll}
\texttt{\scriptsize{}Keyword} & \texttt{\scriptsize{}Nb}\tabularnewline
\texttt{\scriptsize{}pb\_hydraulique\_turbulent} & \texttt{\scriptsize{}1}\tabularnewline
\end{tabular} & \texttt{\scriptsize{}}%
\begin{tabular}{ll}
\texttt{\scriptsize{}Keyword} & \texttt{\scriptsize{}Nb}\tabularnewline
\texttt{\scriptsize{}Navier\_Stokes\_turbulent} & \texttt{\scriptsize{}1}\tabularnewline
\end{tabular}\tabularnewline
\hline 
\texttt{\scriptsize{}10.} & \texttt{\scriptsize{}}%
\begin{tabular}{l}
\texttt{\scriptsize{}Channel or pipe flow with abrupt}\tabularnewline
\texttt{\scriptsize{}changing cross section}\tabularnewline
\end{tabular} & \texttt{\scriptsize{}7} & \texttt{\scriptsize{}}%
\begin{tabular}{ll}
\texttt{\scriptsize{}Keyword} & \texttt{\scriptsize{}Nb}\tabularnewline
\texttt{\scriptsize{}pb\_hydraulique\_turbulent} & \texttt{\scriptsize{}6}\tabularnewline
\texttt{\scriptsize{}pb\_thermohydraulique\_turbulent} & \texttt{\scriptsize{}1}\tabularnewline
 & \tabularnewline
\end{tabular} & \texttt{\scriptsize{}}%
\begin{tabular}{ll}
\texttt{\scriptsize{}Keyword} & \texttt{\scriptsize{}Nb}\tabularnewline
\texttt{\scriptsize{}Navier\_Stokes\_turbulent} & \texttt{\scriptsize{}6}\tabularnewline
\texttt{\scriptsize{}}%
\begin{tabular}{l}
\texttt{\scriptsize{}Navier\_Stokes\_turbulent}\tabularnewline
\texttt{\scriptsize{}+ C\_d\_t\_t}\tabularnewline
\end{tabular} & \texttt{\scriptsize{}1}\tabularnewline
\end{tabular}\tabularnewline
\hline 
\texttt{\scriptsize{}11.} & \texttt{\scriptsize{}Poiseuille flow} & \texttt{\scriptsize{}16} &  & \tabularnewline
\hline 
\texttt{\scriptsize{}12.} & \texttt{\scriptsize{}Isotropic Homogenous Turbulence} & \texttt{\scriptsize{}2} & \texttt{\scriptsize{}}%
\begin{tabular}{ll}
\texttt{\scriptsize{}Keyword} & \texttt{\scriptsize{}Nb}\tabularnewline
\texttt{\scriptsize{}pb\_thermohyd\_t\_s\_p} & \texttt{\scriptsize{}1}\tabularnewline
\texttt{\scriptsize{}pb\_hydraulique\_turbulent} & \texttt{\scriptsize{}1}\tabularnewline
\end{tabular} & \texttt{\scriptsize{}}%
\begin{tabular}{ll}
\texttt{\scriptsize{}Keyword} & \texttt{\scriptsize{}Nb}\tabularnewline
\texttt{\scriptsize{}Navier\_Stokes\_turbulent} & \texttt{\scriptsize{}2}\tabularnewline
\texttt{\scriptsize{}+ C\_d\_t\_t} & \texttt{\scriptsize{}1}\tabularnewline
\end{tabular}\tabularnewline
\hline 
\texttt{\scriptsize{}13.} & \texttt{\scriptsize{}Flow through porous media} & \texttt{\scriptsize{}3} &  & \tabularnewline
\hline 
\texttt{\scriptsize{}14.} & \texttt{\scriptsize{}Quasi-compressible flow} & \texttt{\scriptsize{}9} & \texttt{\scriptsize{}}%
\begin{tabular}{ll}
\texttt{\scriptsize{}Keyword} & \texttt{\scriptsize{}Nb}\tabularnewline
\texttt{\scriptsize{}pb\_thermohydraulique\_QC} & \texttt{\scriptsize{}6}\tabularnewline
\texttt{\scriptsize{}pb\_thermohyd\_t\_QC} & \texttt{\scriptsize{}3}\tabularnewline
\texttt{\scriptsize{}~~~~+ pb\_conduction} & \texttt{\scriptsize{}1}\tabularnewline
 & \tabularnewline
 & \tabularnewline
 & \tabularnewline
 & \tabularnewline
 & \tabularnewline
 & \tabularnewline
\end{tabular} & \texttt{\scriptsize{}}%
\begin{tabular}{ll}
\texttt{\scriptsize{}Keyword} & \texttt{\scriptsize{}Nb}\tabularnewline
\texttt{\scriptsize{}}%
\begin{tabular}{l}
\texttt{\scriptsize{}Navier\_Stokes\_QC}\tabularnewline
\texttt{\scriptsize{}+ C\_d\_ch\_QC}\tabularnewline
\end{tabular} & \texttt{\scriptsize{}6}\tabularnewline
\texttt{\scriptsize{}}%
\begin{tabular}{l}
\texttt{\scriptsize{}Navier\_Stokes\_turbulent\_QC}\tabularnewline
\texttt{\scriptsize{}+ C\_d\_ch\_t\_QC}\tabularnewline
\texttt{\scriptsize{}+ Conduction}\tabularnewline
\end{tabular} & \texttt{\scriptsize{}2}\tabularnewline
\texttt{\scriptsize{}}%
\begin{tabular}{l}
\texttt{\scriptsize{}Navier\_Stokes\_turbulent}\tabularnewline
\texttt{\scriptsize{}Navier\_Stokes\_turbulent\_QC}\tabularnewline
\texttt{\scriptsize{}C\_d\_t\_t}\tabularnewline
\texttt{\scriptsize{}C\_d\_ch\_t\_QC}\tabularnewline
\end{tabular} & \texttt{\scriptsize{}1}\tabularnewline
\end{tabular}\tabularnewline
\hline 
\texttt{\scriptsize{}15.} & \texttt{\scriptsize{}}%
\begin{tabular}{l}
\texttt{\scriptsize{}Advection of passive scalars using}\tabularnewline
\texttt{\scriptsize{}different convection schemes}\tabularnewline
\end{tabular} & \texttt{\scriptsize{}3} &  & \tabularnewline
\hline 
\texttt{\scriptsize{}16.} & \texttt{\scriptsize{}Test of grid convergence} & \texttt{\scriptsize{}4} &  & \tabularnewline
\hline 
\texttt{\scriptsize{}17.} & \texttt{\scriptsize{}Miscellaneous tests} & \texttt{\scriptsize{}14} &  & \tabularnewline
\hline 
\texttt{\scriptsize{}18.} & \texttt{\scriptsize{}Front-Tracking} & \texttt{\scriptsize{}21} &  & \tabularnewline
\hline 
\texttt{\textbf{\scriptsize{}Total}} &  & \texttt{\textbf{\scriptsize{}160}} &  & \tabularnewline
\hline 
\end{tabular}{\scriptsize \par}
\end{sidewaystable}


\pagebreak

\bibliographystyle{plain}
\bibliography{BibTeX_TrioCFD,18-Stokes_bib}



\end{document}
