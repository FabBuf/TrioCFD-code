\newpage

\rhead{\'ECOULEMENTS DIPHASIQUES}
\lhead{Front-Tracking discontinu}

\chapter{Front-Tracking discontinu}
\label{sec:FTD}

\section{Pr\'esentation du mod\`ele}

Le mod\`ele \textit{Front-Tracking discontinu} a pour objectif de r\'esoudre des probl\`emes instationnaires
de type croissance de bulles en paroi et instabilit\'es par exemple.
Les m\'ethodes num\'eriques existantes ont toutes des d\'efauts qui rendent difficiles ou impossibles
ce type de calculs.
C'est pour cette raison qu'une m\'ethode num\'erique adapt\'ee \`a ces probl\`emes a \'et\'e construite :
la m\'ethode mixte Front-Tracking/VOF.
A l'heure actuelle, celle-ci traite principalement des simulations axisym\'etriques ou bidimensionnelles
afin de concentrer les efforts sur la qualit\'e de la physique des simulations,
mais la m\'ethode tridimensionnelle est d'ores et d\'ej\`a en cours de mise en oeuvre.

Apr\`es avoir d\'efini les crit\`eres de qualit\'e attendus de la solution num\'erique,
les diff\'erentes m\'ethodes disponibles vont rapidement \^etre pr\'esent\'ees .
Les manquements des m\'ethodes existantes sur les crit\`eres d\'efinis seront ensuite explicit\'es.

La m\'ethode choisie comme base de travail sera alors d\'efinie et l'algorithme le plus proche
des crit\`eres impos\'es sera finalement d\'ecrit.


\subsection{Discussion sur les m\'ethodes num\'eriques}

L'objectif de ce module est de simuler num\'eriquement les interactions des interfaces
avec des objets de petite taille (sondes optiques par exemple)
et les ph\'enom\'enes li\'es à l'\'ebullition et \`a la crise d'\'ebullition.

L'interaction des interfaces avec les petites structures est un problème gouvern\'e
par la dynamique des lignes de contact et la tension de surface.
Lors de l'\'ebullition, les m\'ecanismes suivants sont susceptibles de jouer un r\^ole fondamental
et doivent pouvoir \^etre trait\'es num\'eriquement :
\begin{itemize}
  \item la croissance d'une bulle de vapeur sur une paroi chauff\'ee,
  \item le processus de formation d'un film de liquide sous la bulle,
  \item l'instabilit\'e de recul de la ligne de contact
        (pour d\'eterminer si elle peut jouer un r\^ole dans la d\'etermination du flux critique),
  \item le remouillage des parois.
\end{itemize}
Tous ces probl\`emes n\'ecessitent la prise en compte d'un \'ecoulement diphasique avec changement de phase
o\`u la tension de surface est du m\^eme ordre de grandeur que les forces d'inertie,
de dissipation visqueuse ou de gravit\'e.
Les lignes de contact y jouent un r\^ole essentiel dont la m\'ethode num\'erique doit rendre compte.


\subsubsection{Revue de quelques m\'ethodes num\'eriques existantes}

La revue pr\'esent\'ee ici sera relativement sommaire car un tel exercice
a d\'ej\`a \'et\'e men\'e par ailleurs (voir \cite{Jamet2002}, \cite{Duquennoy2000}, \cite{Lebaigue1998}).
Seules les diff\'erentes caract\'eristiques des m\'ethodes disponibles seront explicit\'ees
afin de justifier le choix de la m\'ethode de d\'epart.
\smallskip \\

\textit{\textbf{Les m\'ethodes de suivi lagrangiennes}}
\smallskip \\

L'approche qui vient imm\'ediatement \'a l'esprit consiste \'a d\'ecrire les interfaces par un maillage mobile.
Cette approche conduit \`a deux classes de m\'ethodes :
d'une part les m\'ethodes de suivi lagrangiennes et mixtes eul\'eriennes lagrangiennes
(appel\'ees m\'ethodes ALE pour Arbitrary Lagrangian Eulerian)
et d'autre part les m\'ethodes de front-tracking.

Dans les premi\`eres (voir par exemple \cite{Maury1996}), tout le maillage du fluide se d\'eforme
de sorte que les interfaces soient \`a chaque instant une ligne du maillage.
Avec une telle formulation, les conditions aux limites de contrainte m\'ecanique,
de vitesse ou de flux de chaleur sont naturelles, mais les changements de topologie des interfaces
sont pratiquement hors de port\'ee pour des raisons de complexit\'e g\'eom\'etrique.
Cette m\'ethode est utilis\'ee lorsqu'on cherche des r\'esultats très pr\'ecis
sur des g\'eom\'etries relativement simples.
\smallskip \\

\textit{\textbf{Les m\'ethodes \`a double maillage eul\'erien/lagrangien}}
\smallskip \\

Dans la m\'ethode de front-tracking \cite{Unverdi1992}, \cite{Shin2002}, \cite{Duquennoy2000},
un maillage surfacique mobile repr\'esente les interfaces tandis que la vitesse,
la pression, la temp\'erature et les autres grandeurs volumiques sont discr\'etis\'ees sur un maillage fixe.

Dans cette approche, les conditions aux limites aux interfaces sont difficiles \`a ma\^itriser.
Certaines propri\'et\'es accessibles dans une approche lagrangienne,
comme les propri\'et\'es de conservation du volume ou les bilans d'\'energie sur les phases
sont tr\`es compliqu\'ees à v\'erifier lorsque les interfaces ne coincident pas avec les lignes du maillage.

Ces m\'ethodes sont donc caract\'eris\'ees par des impr\'ecisions au niveau des bilans,
compens\'ees par un grand nombre d'astuces permettant de rendre ces impr\'ecisions acceptables.\\

Beaucoup moins lourdes que les m\'ethodes lagrangiennes, les m\'ethodes de Front-Tracking b\'en\'eficient
de toute l'exp\'erience acquise en simulation d'\'ecoulements monophasiques.
La gestion du maillage surfacique des interfaces reste une difficult\'e majeure,
notamment lors des changements de topologie, mais les contraintes sur ce maillage sont
nettement moins s\'ev\`eres que pour le maillage volumique des m\'ethodes lagrangiennes.
Elles se r\'ev\`elent bien plus efficaces pour r\'esoudre l'\'ecoulement monophasique \`a l'int\'erieur des phases,
o\`u le sch\'ema classique "Marker And Cell" reste une valeur s\^ure.
\smallskip \\

\textit{\textbf{Les m\'ethodes purement eul\'eriennes}}
\smallskip \ \\

La m\'ethode la plus r\'epandue et l'une des plus robustes \`a ce jour est
la m\'ethode VOF (\emph{Volume Of Fluid} \cite{Hirt1981}).
Dans cette m\'ethode, on discr\'etise le champ de masse volumique sur le m\^eme maillage fixe que la vitesse,
la pression et la temp\'erature.
De ce champ de masse volumique, on d\'eduit par des algorithmes g\'eom\'etriques la position
des interfaces dans chaque maille. Cette position permet de calculer l'effet de la tension de surface sur le fluide
et les variations de masse volumique dans chaque maille.

Selon le niveau de raffinement de l'algorithme de reconstruction, on obtient une m\'ethode plus ou moins facile
à mettre en oeuvre. La prise en compte de la tension de surface d\'epend \'enorm\'ement de cet algorithme
de reconstruction et les m\'ethodes les plus pr\'ecises sont tout aussi complexes à mettre en oeuvre
que les m\'ethodes de front-tracking.

En revanche, le suivi des interfaces est r\'ealis\'e par un algorithme de transport sur le maillage eul\'erien.
On \'evite ainsi une grande partie des difficult\'es d'ordre algorithmique li\'ees au maillage mobile.\\

Parmi les autres m\'ethodes purement eul\'eriennes, on trouve aussi les m\'ethodes dites "level-set",
o\`u l'indicatrice de phase est une fonction continue discr\'etis\'ee.
Les interfaces sont implicitement d\'efinies comme l'isovaleur 0.5 de l'indicatrice de phase.
L'indicatrice est associ\'ee à une \'equation de transport qui comporte obligatoirement un terme anti-diffusif
(la d\'efinition de ce terme pose quelques difficult\'es).

Ces m\'ethodes pr\'esentent encore moins de difficult\'es num\'eriques que les m\'ethodes VOF,
car les grandeurs physiques sont r\'eguli\`eres.\\

On peut citer la m\'ethode fond\'ee sur les \'equations du second-gradient \cite{Jamet2001}.
Plus proche de la physique, elle ne manque pas d'\'el\'egance puisque les propri\'et\'es physiques
propres aux interfaces -telles que la tension de surface ou la chaleur latente de changement d'\'etat-
sont naturellement prises en compte par les \'equations d'\'evolution des variables eul\'eriennes.
Les grandeurs physiques sont continues \`a l'\'echelle du maillage et c'est sans doute
la m\'ethode la plus simple à mettre en oeuvre num\'eriquement.
\smallskip \\

\textit{\textbf{Le compromis complexit\'e/pr\'ecision/efficacit\'e}}
\smallskip \\

Les m\'ethodes ont \'et\'e cit\'ees dans l'ordre d\'ecroissant
d'efficacit\'e num\'erique, de pr\'ecision et de complexit\'e.

Les m\'ethodes lagrangiennes sont naturellement les plus pr\'ecises car la discr\'etisation des champs
de grandeurs physiques respecte la topologie de ces champs :
les lignes du maillage suivent les discontinuit\'es du champ physique.

Ces m\'ethodes sont aussi les plus compliqu\'ees à mettre en oeuvre.\\

Les m\'ethodes mixtes eul\'eriennes-lagrangiennes constituent un compromis int\'eressant
car elles \'evitent la gestion compliqu\'ee d'un maillage volumique mobile qui suive
les champs physiques mais conservent une information pr\'ecise sur la localisation des discontinuit\'es.\\

Les m\'ethodes purement eul\'eriennes sont en g\'en\'eral les plus simples
car elles \'evitent la gestion d'un maillage mobile,
mais les discontinuit\'es des champssont repr\'esent\'ees par une grandeur sur le maillage eul\'erien
et la localisation n'est donc pas très pr\'ecise.

Ces m\'ethodes requi\`erent donc souvent un maillage plus fin pour obtenir la même pr\'ecision.\\

Il semble qu'\`a l'heure actuelle, toutes les m\'ethodes cit\'ees
ont \'et\'e appliqu\'ees \`a des calculs d'\'ecoulements diphasiques en 3D,
et la plupart l'ont \'et\'e pour des \'ecoulements avec transfert de masse.


\subsubsection{Critères de qualit\'e de la m\'ethode}
Duquennoy (\cite{Duquennoy2000}, \cite{Duquennoy2000_2}) a mis en oeuvre une m\'ethode de front-tracking
dans laquelle il a plus sp\'ecifiquement am\'elior\'e la prise en compte
du changement de phase et des lignes de contact.

Le travail qu'il a r\'ealis\'e a mis en \'evidence plusieurs points faibles de la m\'ethode de front-tracking :
\begin{itemize}
  \item les courants parasites,
  \item le d\'efaut de conservation de la masse,
  \item les limitations li\'ees aux lignes de contact,
  \item la coh\'erence de la description du champ de temp\'erature.
\end{itemize}

Ces points faibles constituent autant de crit\`eres de s\'election d'une m\'ethode
mieux adapt\'ee aux problèmes consid\'er\'es.
\smallskip \\

\textit{\textbf{Les courants parasites}}
\smallskip \\

On appelle courants parasites les courants observ\'es dans une simulation num\'erique
ayant atteint un \'etat stationnaire d'\'equilibre alors qu'aucune \'energie n'est inject\'ee dans le syst\`eme.

Ces courants r\'esultent d'erreurs de discr\'etisation de la tension de surface
et ont de graves cons\'equences sur les r\'esultats des calculs.\\

La première cons\'equence est qu'ils rendent certains calculs impossibles.
D'apr\`es Duquennoy \cite{Duquennoy2000}, l'intensit\'e des courants parasites augmente lorsque
la valeur du nombre d'Ohnesorge diminue (la longueur D est la dimension caract\'eristique des gouttes ou des bulles) :
\begin{equation}
  Oh = \dfrac{\mu_{l}}{\sigma \rho_{l} D}
\end{equation}

En pratique, avec de l'eau \`a pression atmosph\'erique et une dimension caract\'eristique de l'ordre du centim\`etre,
les courants parasites rendent les interfaces compl\`etement instables aux temps longs.
La simulation de ph\'enom\`enes lents dont la dur\'ee exc\`ede quelques secondes est alors impossible.\\

La deuxi\`eme cons\'equence est qu'ils rendent irr\'ealistes les calculs anisothermes
o\`u le transfert de chaleur n'est pas domin\'e par la conduction.
Lors de la croissance d'une bulle de vapeur \`a haute pression par exemple,
la g\'eom\'etrie des couches limites thermiques d\'etermine l'intensit\'e du transfert de masse
en dehors des lignes de contact et de la micro-couche.
Or, les courants parasites d\'etruisent complètement ces couches limites,
rendant le calcul du champ de temp\'erature irr\'ealiste.

Il est donc extr\^emement important de r\'eduire l'intensit\'e des courants parasites au moins jusqu'\`a
un niveau tel que la valeur du nombre de P\'eclet soit tr\`es inf\'erieure \`a 1.
Dans ce cas, les courants parasites n'ont plus aucune influence sur le champ de temp\'erature.
\smallskip \\

\textit{\textbf{La conservation du volume}}
\smallskip \\

Cette propri\'et\'e est un enjeu tr\`es important pour la m\'ethode de front-tracking comme pour la m\'ethode VOF.
Par nature, la m\'ethode VOF a de bonnes propri\'et\'es de conservation du volume
mais certaines \'etapes de l'algorithme sont approximatives
et la conservation est inexacte dans la plupart des algorithmes existants \cite{Aulisa2003}.

La m\'ethode de front-tracking mise en oeuvre par Duquennoy pr\'esente une erreur encore plus importante
que les m\'ethodes VOF sur le bilan de volume des phases,
d'une part en raison de la m\'ethode d'interpolation par splines utilis\'ee pour garantir
la r\'egularit\'e du maillage lagrangien,
d'autre part en raison de l'algorithme de transport de ce maillage.\\

L'erreur sur le bilan de volume a des cons\'equences importantes
sur la simulation de la croissance de bulles en paroi :
dans certains cas, le volume de la bulle diminue alors que le fluide est partout surchauff\'e.

Si l'erreur est moins importante, le temps de croissance
et le diam\`etre au d\'etachement peuvent \^etre mal pr\'edits.\\

Par cons\'equent, l'exactitude du bilan de volume des phases
est un ingr\'edient indispensable de la m\'ethode num\'erique.
\smallskip \\

\textit{\textbf{Les lignes de contact}}
\smallskip \\

La m\'ethode de front-tracking \cite{Duquennoy2000} n'est pas la seule dans laquelle
des conditions aux limites sur les angles de contact ont \'et\'e mises en oeuvre.
Cependant, c'est peut-\^etre, avec les m\'ethodes lagrangiennes, celle pour laquelle l'expression
d'une condition aux limites d'angle de contact est la plus directe.
La formulation utilis\'ee par Duquennoy emp\^eche pour l'instant la simulation
d'angles inf\'erieurs \`a $30^{\circ}$ environ
et la mod\'elisation de l'angle de contact est mal contr\^ol\'ee
(l'angle de contact dynamique notamment n'est pas mod\'elis\'e rigoureusement). 
\smallskip \\

\textit{\textbf{La coh\'erence du champ de temp\'erature}}
\smallskip \\

La formulation pr\'ec\'edente du front-tracking n'assure pas exactement la
condition aux limites $T = T_{sat}$ aux interfaces.
Ainsi, on peut observer qu'une bulle de vapeur initialement \`a $T_{sat}$ se r\'echauffe
au contact d'un liquide surchauff\'e, ce qui contredit le sens physique.

Cette incoh\'erence, ainsi que l'impr\'ecision de la formulation du transfert de masse aux interfaces n'est pas
compatible avec la mise en oeuvre d'un mod\`ele pr\'ecis de transfert de chaleur singulier aux lignes de contact.


\subsubsection{Choix d'impl\'ementation : une m\'ethode de front-tracking}

\textit{\textbf{Argumentaire}}
\smallskip \\
Le choix s'est port\'e sur une m\'ethode de front-tracking pour les raisons suivantes :
\begin{itemize}
  \item Il nous faut une m\'ethode capable de g\'erer des grandes d\'eformations,
        relativement facile \`a mettre en oeuvre et extensible au 3D,
        ce qui fait des m\'ethodes purement lagrangiennes de mauvais candidats (notamment pour le passage au 3D).
  \item La m\'ethode doit pouvoir traiter des probl\`emes o\`u la tension de surface
        est dominante sans aucun champ de vitesse parasite.
        Or, \`a notre connaissance, il n'existe pas de formulation de la m\'ethode VOF sans maillage lagrangien
        qui v\'erifie ce critère, alors que nous en avons trouv\'e une pour le front-tracking.
  \item Nous disposons d\'ej\`a d'une bonne exp\'erience des m\'ethodes de Front-Tracking
        ainsi que d'une base logicielle suite \`a la th\`ese de Duquennoy \cite{Duquennoy2000}.
\end{itemize}

Pourtant, la m\'ethode originale de front-tracking pr\'esente de nombreux inconv\'enients,
dont le plus important est peut-\^etre la pr\'esence de courants parasites.
C'est la raison pour laquelle de nombreux aspects de la formulation initiale ont \'et\'e profond\'ement modifi\'es,
\`a commencer par la d\'emarche de discr\'etisation des \'equations continues.
\smallskip \\

\textit{\textbf{D\'emarche de discr\'etisation des \'equations continues}}
\smallskip \\

Sur le fond, cette d\'emarche change radicalement par rapport \`a la m\'ethode initiale \cite{Unverdi1992} :
l'ancienne approche consiste \`a appliquer un op\'erateur de filtrage aux \'equations de Navier-Stokes
pour rendre toutes les grandeurs continues (masse volumique, champ de vitesse, etc.)
puis \`a discr\'etiser ces \'equations filtr\'ees.

Les outils d'analyse num\'erique classiques pr\'edisent alors une convergence rapide en fonction de la
discr\'etisation car la solution du probl\`eme est r\'eguli\`ere.\\

L'inconv\'enient de la m\'ethode est que l'on converge rapidement vers la solution du problème filtr\'e
et non vers la solution du probl\`eme r\'eel.
En ce sens, la m\'ethode n'est m\^eme pas "consistante".

Si on fait d\'ecroître le support de la fonction de filtrage en m\^eme temps que l'on raffine le maillage,
on ne dispose d'aucun r\'esultat de convergence de la m\'ethode num\'erique.

Ainsi, dans la m\'ethode de front-tracking mise en oeuvre par Duquennoy,
l'intensit\'e des courants parasites est constante lorsqu'on raffine le maillage.

La m\'ethode ne converge donc pas vers la solution du probl\`eme continu.\\

La nouvelle approche est fond\'ee sur des bilans sur des volumes de contr\^ole,
exactement comme dans une m\'ethode de volumes finis classique.
On discr\'etise donc directement les \'equations de Navier-Stokes, y compris les discontinuit\'es aux interfaces.

Pour obtenir la convergence en maillage de la m\'ethode, un mod\`ele de sous-maille des grandeurs physiques
doit \^etre introduit dans les \'el\'ements o\`u ces grandeurs sont discontinues.
Ainsi, on utilise une repr\'esentation discr\`ete un peu plus fine des champs physiques
qui tient compte de la position des interfaces.
La convergence obtenue est d'ordre 1 si on utilise un mod\`ele de sous-maille d'ordre z\'ero
dans les \'el\'ements contenant des interfaces.
Des mod\`eles de sous-maille plus raffin\'es permettraient d'augmenter la pr\'ecision de la m\'ethode
et on devrait pouvoir obtenir les m\^emes qualit\'es qu'une m\'ethode lagrangienne.
\smallskip \\

\textit{\textbf{R\'esum\'e des apports essentiels de la m\'ethode}}
\smallskip \\

Cette nouvelle approche de discr\'etisation est emprunt\'ee aux m\'ethodes VOF
et n'est donc pas nouvelle.
Les apports essentiels du travail men\'e dans TrioCFD concernent :
\begin{itemize}
  \item les algorithmes de transport, pour assurer un bilan de masse exact des phases,
  \item la discr\'etisation de la tension de surface
        et de la gravit\'e pour \'eliminer pratiquement les courants parasites,
  \item le calcul du flux de chaleur aux interfaces pour une meilleure pr\'ecision,\\
  \item la prise en compte des lignes de contact.
\end{itemize}

La m\'ethode propos\'ee peut \^etre \'etendue \`a des sch\'emas en trois dimensions
et \`a d'autres types de discr\'etisations eul\'eriennes.

On propose en effet une formulation des diff\'erentes grandeurs physiques
(en particulier les forces de tension de surface et de gravit\'e)
et des op\'erateurs d'interpolation qui s'\'etend directement en trois dimensions.

D'autre part, pour peu que la discr\'etisation de la vitesse et de la pression permette d'\'ecrire
un bilan de volume discret, le sch\'ema est extensible à des discr\'etisations plus riches que celle
du sch\'ema MAC (par exemple la discr\'etisation Volume-\'el\'ements finis \cite{Emonot2003}, \cite{Heib2003}).

On peut remarquer que les approches "VOF" et "front-tracking" semblent converger vers une m\^eme formulation.
La formulation VOF d'origine a ainsi \'et\'e modifi\'ee \cite{Popinet2000} par
l'ajout d'un maillage lagrangien des interfaces.

Elle constitue maintenant une m\'ethode "VOF avec marqueurs"
(\`a notre connaissance, une telle m\'ethode n'existe cependant pas encore en 3D).
Cette m\'ethode de front-tracking comporte d\'ej\`a un maillage lagrangien des interfaces
et la discr\'etisation actuelle des grandeurs à partir de bilans sur les volumes de contr\^oles la rapproche des
m\'ethodes VOF.

Une telle convergence est peut-\^etre le signe que l'on s'approche
d'une formulation optimale pour ce type de probl\`emes...


\subsection{D\'efinition des grandeurs discr\`etes}
Nous utiliserons les conventions de notation suivantes pour les grandeurs discr\`etes :\\
- les grandeurs discr\'etis\'ees sur le maillage eul\'erien (fixe) sont not\'ees avec une barre sup\'erieure : pression $\overline{P}$, temp\'erature $\overline{T}$, divergence de la vitesse $\overline{\triangledown \cdot \overline{v}}$, etc.\\
- les grandeurs discr\'etis\'ees sur le maillage lagrangien (des interfaces) sont
not\'ees avec un chapeau : vitesse $\hat{v}$, courbure $\hat{c}$, tension de surface $\hat{\sigma}$, flux de masse $\hat{\dot{m}}$, etc.

\subsubsection{Le maillage de l'interface}
En deux dimensions, l'interface est d\'efinie par un maillage constitu\'e de noeuds reli\'es par des segments. Par convention, les segments sont orient\'es de sorte que la vapeur se trouve \`a gauche. En trois dimensions, l'interface sera une r\'eunion de triangles dont la normale est orient\'ee vers la vapeur.\\
Pour le bon fonctionnement des algorithmes, le maillage doit v\'erifier certaines propri\'et\'es topologiques :\\
- deux segments d'interface ne se coupent jamais,\\
- soit les interfaces sont ferm\'ees, soit leurs extr\'emit\'es (les noeuds n'ayant qu'un seul segment raccord\'e) sont situ\'ees sur un bord du domaine,\\
- les interfaces d\'efinissent donc des volumes ferm\'es, on demande que toutes les interfaces d\'efinissant le bord d'un volume aient leurs normales orient\'ees dans la m\^eme direction.\\
Ces propri\'et\'es assurent la coh\'erence topologique du maillage, en particulier la d\'efinition du contenu (gaz ou liquide) des volumes d\'efinis par les interfaces. Les interfaces sont consid\'er\'ees comme une succession de
segments. En particulier, on ne cherche pas \`a augmenter l'ordre de la m\'ethode par l'utilisation de splines pour le calcul de l'indicatrice. En effet, les propri\'et\'es topologiques \'enonc\'ees plus haut sont plus faciles \`a v\'erifier avec l'utilisation des segments. Les algorithmes sont plus faciles à mettre en oeuvre et plus robustes. Si cette m\'ethode simple donne des r\'esultats satisfaisants, on peut penser qu'en trois dimensions, un maillage en triangles n'ayant pas plus de propri\'et\'es de r\'egularit\'e conviendra aussi. Au vu des difficult\'es rencontr\'ees dans la gestion des maillages surfaciques en trois dimensions, il semble judicieux de privil\'egier les m\'ethodes les plus simples et les plus robustes.
\smallskip \\

\textit{\textbf{D\'efinitions}}
\smallskip \\

On appellera $\Gamma$ l'ensemble des interfaces et E un \'el\'ement d'interface (segment en 2D et triangle en 3D).\\
Nous utiliserons quelques fois la notion de "portion d'interface connexe complète". La connexit\'e signifie que l'on peut passer d'un noeud \`a n'importe quel autre de la portion d'interface en traversant des \'el\'ements de proche en proche. On parle de portion compl\`ete si pour tout \'el\'ement de la portion, ses voisins y sont aussi. Ainsi, si une portion compl\`ete a des bords, ces derniers sont forc\'ement sur un bord du domaine.

\subsubsection{Courbure des interfaces}

La courbure est calcul\'ee aux noeuds du maillage surfacique. Ce choix est imp\'eratif pour que la tension de surface qui en r\'esulte n'admette que la solution "courbure constante" pour solution stationnaire du probl\`eme. Si on choisissait de discr\'etiser la courbure aux centres des segments, elle serait nulle pour un profil d'interface ondul\'e et ce profil serait une solution num\'erique stationnaire.\\
On peut envisager deux formulations de la courbure discr\`ete :\\
- une formulation fond\'ee sur une interpolation g\'eom\'etrique,\\
- une formulation fond\'ee sur la diff\'erentielle de l'\'energie d'interface.
\smallskip \\

\textit{\textbf{Formulation g\'eom\'etrique}}
\smallskip \\

La première est la plus intuitive en deux dimensions : on utilise la d\'efinition g\'eom\'etrique de la courbure
\begin{equation}
c(s) \hat{=} \frac{\partial t}{\partial s} \cdot n
\end{equation}
o\`u s est l'abscisse curviligne en param\'etrage normal. Tra\c cons le cercle passant par le point de l'interface et ses deux voisins. La courbure bidimensionnelle est l'inverse sign\'e du rayon $R$ du cercle (courbure positive si le centre du cercle est dans la vapeur, n\'egative sinon). En g\'eom\'etrie axisym\'etrique, il faut ajouter la courbure dans l'autre direction qui s'\'ecrit $c_{2} = sin \theta / x$.

\subsubsection{Relations de passage conservatives entre le maillage eul\'erien et le maillage lagrangien}
Plusieurs \'etapes de l'algorithme n\'ecessitent de passer d'une grandeur d\'efinie aux noeuds des interfaces à une grandeur d\'efinie sur le maillage fixe et r\'eciproquement.
On voudra \'ecrire des bilans exacts sur ces grandeurs et nous avons besoin d'une d\'efinition rigoureuse des op\'erateurs d'interpolation. On d\'efinit ci-dessous
des op\'erateurs d'interpolation et les relations de conservation qu'ils v\'erifient.
On note $\overline{\mathcal{G}}$ l'op\'erateur permettant de passer d'une grandeur surfacique \`a une grandeur volumique, et $\hat{\mathcal{G}}$ l'op\'erateur r\'eciproque. Pour des champs discrets $\hat{f}$ et $\overline{f}$ , on note :
\begin{equation}
\hat{f} = \hat{\mathcal{G}}(\overline{f})
\end{equation}
\begin{equation}
\overline{f} = \overline{\mathcal{G}}(\hat{f})
\end{equation}
Remarque importante : ces op\'erateurs ne sont pas inverses l'un de l'autre. Si l'on passe d'une grandeur surfacique \`a une grandeur volumique, puis \`a nouveau \`a une grandeur surfacique, l'int\'egrale est conserv\'ee mais les valeurs aux noeuds subissent une diffusion num\'erique due aux interpolations successives :
\begin{equation}
\hat{\mathcal{G}} \left( \overline{\mathcal{G}} \left( \hat{f} \right) \right) \neq  
\hat{f} 
\end{equation}
La propri\'et\'e la plus importante v\'erifi\'ee par ces op\'erateurs est la relation de conservation suivante, o\`u $\Gamma$ repr\'esente une portion connexe compl\`ete d'interfaces et $\Omega$ l'ensemble des \'el\'ements du maillage eul\'erien travers\'es par $\Gamma$ :
\begin{equation}
\int_{\Gamma} \tilde{f}(x) ds = \int_{\Omega} \overline{f}(\Omega) d\Omega
\end{equation}

\subsubsection{Indicatrice de phase et masse volumique}

L'indicatrice discrète $\overline{I}$ d'un \'el\'ement de volume $\Omega$ est d\'efinie comme la fraction du volume de l'\'el\'ement occup\'ee par la phase gazeuse. C'est donc le taux de vide moyen dans l'\'el\'ement. La masse volumique $\overline{\rho}$ d'un \'el\'ement est la masse volumique moyenne dans cet \'el\'ement, d\'efinie par :
\begin{equation}
\overline{\rho} \hat{=} \dfrac{1}{\overline{V_{\Omega}}} \int_{\Omega} \rho d\Omega
\end{equation}
On a les relations suivantes :
\begin{equation}
\overline{\rho} = \rho_{v} \overline{I} + \rho_{l} (1-\overline{I})
\end{equation}
\begin{equation}
\overline{I} = \dfrac{\rho - \rho_{l}}{\rho_{v} - \rho_{l}} \label{eq:Indicatrice_phase}
\end{equation}
L'indicatrice est calcul\'ee en fonction de la position des noeuds de l'interface par un algorithme g\'eom\'etrique exact. L'algorithme est identique \`a celui propos\'e par Popinet \cite{Popinet2000}, sauf que nous n'utilisons que les segments de l'interface et non une interpolation par splines.

\subsubsection{Discr\'etisation de la vitesse et de la pression}

\textit{\textbf{Choix d'une formulation en vitesse ou en quantit\'e de mouvement}}
\smallskip \\

Nous disposons d'une discr\'etisation de la masse volumique, au travers de l'indicatrice.\\
Il s'agit maintenant de choisir si on discr\'etise la vitesse $v$ ou la quantit\'e de mouvement $\rho v$. Il s'agit en fait de faire le choix de privil\'egier la conservation du volume ou la conservation de la quantit\'e de mouvement.\\
Dans le premier cas, on d\'efinit la vitesse discr\`ete sur une face comme la moyenne de la vitesse sur cette face et au cours du pas de temps. La condition d'incompressibilit\'e est alors tr\`es facile \`a \'ecrire et se traduit de mani\`ere exacte en termes de variables discr\`etes :
\begin{equation}
\overline{\nabla \cdot \overline{v}} = 0 \Leftrightarrow \forall\Omega, \int_{\partial\Gamma} \overline{v} \cdot \overline{n} ds = 0
\end{equation}
En revanche, un bilan de masse local et la conservation de la quantit\'e de mouvement sont beaucoup plus compliqu\'es \`a obtenir. On peut cependant \'ecrire un bilan de masse global, comme on le verra par la suite. Il faut pour cela d\'efinir la valeur de la quantit\'e de mouvement discr\`ete $\overline{\rho v}$ \`a partir des autres grandeurs discr\`etes.\\
Si l'on choisit de discr\'etiser la quantit\'e de mouvement $\overline{\rho v}$, sa conservation est facile \`a obtenir (il suffit d'\'ecrire sa variation sous forme de flux au bord des volumes de contr\^ole) et un bilan de masse exact peut \^etre \'ecrit. En revanche, la condition d'incompressibilit\'e devient ambigue localement, dans les r\'egions où la masse volumique varie. En effet, on doit \'evaluer la vitesse par une formule du type :
\begin{equation}
\overline{v} \hat{=} \dfrac{\overline{\rho v}}{\overline{\rho}}
\end{equation}
Pour illustrer les problèmes li\'es à cette discr\'etisation, consid\'erons un probl\`eme dont la solution est un champ de vitesse $\overline{v}$ uniforme. On discr\'etise la quantit\'e $\overline{\rho v}$ qui est donc discontinue aux interfaces. Les diff\'erents \'el\'ements du sch\'ema num\'erique (op\'erateurs de diffusion et de convection notamment) introduisent des erreurs num\'eriques lors du traitement de cette grandeur discontinue, en particulier une diffusion num\'erique. Si on tente ensuite de calculer une vitesse $\overline{v}$ en divisant par $\overline{\rho}$, la vitesse n'est plus uniforme.\\
Une telle erreur aurait des cons\'equences f\^acheuses en front-tracking. Elle augmenterait d'ailleurs avec le rapport de masse volumique. Les interfaces ne seraient pas transport\'ees sans d\'eformation m\^eme si la solution du probl\`eme est un champ de vitesse uniforme. Cela implique que l'on modifie l'\'energie de surface des interfaces, ce qui conduit \`a des courants parasites ou des instabilit\'es num\'eriques.\\
Il n'est pas exclu que l'on puisse construire un sch\'ema num\'erique de transport
de $\overline{v}$ qui ait de bonnes propri\'et\'es, mais le choix retenu pour l'instant est d'utiliser une discr\'etisation de la vitesse.
\smallskip \\

\textit{\textbf{Discr\'etisation}}
\smallskip \\

Dans la mise en oeuvre actuelle, la discr\'etisation de la vitesse et de la pression est de type "marker and cell". La pression est discr\'etis\'ee au centre des volumes de contr\^ole $\Omega_{x}$ et $\Omega_{y}$. Dans la formulation classique on donne les d\'efinitions suivantes pour la vitesse :\\
- les faces verticales portent une composante de vitesse horizontale $\overline{v_{x}}$,\\
- les faces horizontales portent une composante de vitesse verticale $\overline{v_{y}}$,\\
- et en 3D, on d\'efinit de même la troisi\`eme composante de vitesse.\\
Les composantes sont suppos\'ees constantes sur les volumes de contr\^ole $\Omega'$ 0 centr\'es sur chaque face. La vitesse a alors les deux interpr\'etations suivantes :
\begin{equation}
\overline{v_{x}} = \int_{\Gamma_{x}} v \cdot x ds
\end{equation}
\begin{equation}
\overline{v_{x}} = \dfrac{1}{\rho} \int_{\Omega_{x}} \rho v_{x} d\Omega
\end{equation}
La premi\`ere permet d'\'ecrire le bilan de masse conservatif sur $\Omega$ et de d\'efinir l'incompressibilit\'e du fluide sur les \'equations discr\`etes :
\begin{equation}
\int_{\partial\Omega} v \cdot n ds = \int_{\Omega} \nabla \cdot v d\Omega = 0
\end{equation}
La deuxi\`eme permet d'\'ecrire un bilan de quantit\'e de mouvement discret conservatif, \`a condition que $\overline{\rho}$ soit constant.\\
Si la masse volumique n'est pas constante dans le temps, l'\'ecriture des bilans
devient tr\`es compliqu\'ee. Avec le sch\'ema en temps explicite utilis\'e actuellement, le bilan n'est respect\'e qu'approximativement dans les \'el\'ements o\`u la masse volumique varie.

\subsubsection{Discr\'etisation de l'\'energie interne}

L\`a encore, il faut choisir la grandeur discr\`ete pour repr\'esenter l'\'energie interne du fluide. Une discr\'etisation de l'\'energie permet d'\'ecrire facilement un sch\'ema conservatif, ce qui semble important dans le cas du changement de phase. Dans ce cas, le calcul du flux de chaleur de la loi de Fourier et de l'\'evaporation implique une estimation de la temp\'erature en fonction de l'\'energie interne. Or tout comme avec la quantit\'e de mouvement, on doit pour cela diviser l'\'energie par une grandeur qui tend vers z\'ero dans la vapeur. Une mauvaise estimation de cette grandeur peut conduire \`a des temp\'eratures non born\'ees, en violation du deuxi\`eme principe de la thermodynamique.\\
Si on discr\'etise la temp\'erature au contraire, le deuxi\`eme principe est facile à v\'erifier (il conduit aux critères de stabilit\'e en temps des sch\'emas explicites), mais la conservation de l'\'energie est plus difficile \`a assurer dans les r\'egions o\`u $\rho c_{P}$ varie.\\
La formulation actuelle utilise une discr\'etisation de la temp\'erature et n'est
donc pas conservative en \'energie. On pose :
\begin{equation}
\overline{T} \hat{=} \dfrac{\int_{\Omega} \rho c_{P} T d\Omega}{\int_{\Omega} \rho c_{P} d\Omega}
\end{equation}
De fa\c con coh\'erente avec cette d\'efinition, la capacit\'e calorifique de l'\'el\'ement s'\'ecrit :
\begin{equation}
\overline{\rho c_{P}} \hat{=} \dfrac{\int_{\Omega} \rho c_{P} d\Omega}{\overline{V_{\Omega}}} = \rho_{v} c_{Pv} \overline{I} + \rho_{l} c_{Pl} (1 - \overline{I})
\end{equation}

\subsubsection{Transfert de chaleur et de masse aux interfaces}

\textit{\textbf{Temp\'erature de saturation}}
\smallskip \\

D'apr\`es les \'equations continues utilis\'ees, la temp\'erature de saturation locale d\'epend de la courbure des interfaces et s'exprime en fonction des pressions de part et d'autre de l'interface :
\begin{equation}
T_{sat} = T_{sat}(P), avec \:P \begin{array}{rcl} = P_{v} + \dfrac{\rho_{v}}{\rho_{l} - \rho_{v}} \sigma c + \dfrac{1}{2} P_{r}\\
 = P_{l} + \dfrac{\rho_{l}}{\rho_{l} - \rho_{v}} \sigma c + \dfrac{1}{2} P_{r}\end{array}
\end{equation}
Avec la demie-somme des deux expressions de la pression, on obtient la relation
suivante :
\begin{equation}
P = \dfrac{1}{2} \left( P_{v} + P_{l} + \dfrac{\rho_{l} + \rho_{v}}{\rho_{l} - \rho_{v}} \sigma c\right) \label{eq:pressionFTD}
\end{equation}
Or, \`a l'\'equilibre, le champ de pression discret $\overline{P}$ v\'erifie l'\'equation construite \`a partir des termes sources de tension de surface :
\begin{equation}
\overline{P} = \overline{\kappa} \overline{I} - \overline{\rho} \overline{\phi} + P_{0}, avec \left\{ \begin{array}{rcl}
\overline{\kappa} = \sigma c + (\rho_{v} - \rho_{l})\phi = constante\\
P_{0} = constante
\end{array}\right.
\end{equation}
De cette propri\'et\'e, on d\'eduit une expression de la pression dans le liquide et dans la vapeur de part et d'autre de l'interface :
\begin{equation}
P_{l} = P_{0} - \rho_{l}\phi = \overline{P} - \overline{\rho}\overline{\phi} - \kappa\overline{I} - \rho_{l}\phi
\end{equation}
\begin{equation}
P_{v} = P_{0} - \rho_{v}\rho + \kappa = \overline{P} - \overline{\rho}\overline{\phi} - \kappa\overline{I} + \rho_{v}\phi + \kappa
\end{equation}
Cette expression est symbolique car la d\'efinition exacte de $\phi$ n'est pas pr\'ecis\'ee (la seule valeur connue est celle de $\overline{\kappa}$, dont on sait qu'elle est constante \`a l'\'equilibre). Rempla\c cons maintenant les
expressions de $P_{l}$ et $P_{v}$ dans l'\'equation \ref{eq:pressionFTD} :
\begin{equation}
P = \overline{P} + \overline{\rho}\overline{\phi} - \kappa\overline{I} + \dfrac{1}{2} \left( \kappa - (\rho_{v} + \rho_{l})\phi + \dfrac{\rho_{l}+\rho_{v}}{\rho_{l}-\rho_{v}} \sigma c\right) 
\end{equation}
\begin{equation}
P = \overline{P} + \overline{\rho}\overline{\phi} - \kappa\overline{I} + \kappa
\end{equation}
Cette propri\'et\'e sugg\`ere d'utiliser l'expression suivante, exacte \`a l'\'equilibre, de la temp\'erature de saturation en fonction des champs discrets $\overrightarrow{P}$, $\overline{I}$ et $\overline{\kappa}$ :
\begin{equation}
\overline{T_{sat}} \hat{=} T_{sat}(P), avec\:P\:\hat{=}\:\overline{P} + \overline{\rho} \overline{\phi} + (1 - I)\overline{\kappa}
\end{equation}
Cette \'equation para\^it tr\`es simple \`a mettre en oeuvre. De plus, si $\overline{\kappa}$ est constante, le système discret peut-\^etre simultan\'ement \`a l'\'equilibre m\'ecanique et \`a l'\'equilibre thermique. En effet,
d'une part le champ de vitesse nul est solution du syst\`eme ce qui correspond \`a l'\'equilibre m\'ecanique, et d'autre part $\overline{T_{sat}}$ est constant donc le champ de temp\'erature constant $T = T_{sat}$ est une solution stationnaire du syst\`eme d'\'equations (en pr\'esence de gravit\'e, ce r\'esultat n'est pas imm\'ediat sur le système discret). Malheureusement, elle introduit un couplage extr\^emement fort entre l'\'equation de temp\'erature et l'\'equation de quantit\'e de mouvement. Sans un traitement particulier (traitement implicite de la variation de temp\'erature de saturation), le sch\'ema num\'erique est tr\`es instable.
\smallskip \\

\textit{\textbf{Enthalpie de changement de phase}}
\smallskip \\

On d\'efinit maintenant la puissance volumique de changement de phase sur un volume de contr\^ole $\Omega$. C'est l'int\'egrale du flux de chaleur des phases vers les interfaces contenues dans ce volume :
\begin{equation}
\overline{h} \hat{=} \dfrac{1}{\overline{V_{\Omega}}} \int_{\Gamma\cap\Omega} \dot{q} ds
\end{equation}
Dans la formulation continue des \'equations, le flux $\dot{q}$ s'exprime \`a partir du flux de chaleur de part et d'autre de l'interface :
\begin{equation}
%\dot{q} = -k_{l} \nabla T_{l} \cdot n + k_{v} \nabla T_{v} \cdot n
\end{equation}
On peut construire un \'equivalent discret de cette formulation et calculer un flux de chaleur $\hat{\dot{q}}$ discr\'etis\'e aux noeuds de l'interface de la fa\c con suivante (la m\'ethode employ\'ee par Duquennoy \cite{Duquennoy2000}] et \cite{Shin2002}) :
\begin{equation}
\hat{\dot{q}} \hat{=} k_{l} \dfrac{\hat{T_{l}} - \hat{T_{sat}}}{\delta} + k_{v} \dfrac{\hat{T_{v}} - \hat{T_{sat}}}{\delta}
\end{equation}
Cette m\'ethode a plusieurs inconv\'enients :\\
- il faut interpoler la temp\'erature au-delà de l'interface, ce qui pose probl\`eme\\
pr\`es des bords du domaine,\\
- il faut ensuite construire explicitement un terme source pour l'\'energie sur
le maillage fixe et ce terme source peut conduire \`a des incoh\'erences (par
exemple le transfert de chaleur d'une phase \`a l'autre alors que physiquement
l'interface doit \^etre \`a $T_{sat}$),\\
- le contr\^ole pr\'ecis du flux de chaleur pr\`es des lignes de contact est compliqu\'e.\\
Pour ces raisons, on lui pr\'ef\`ere une formulation diff\'erente, plus proche des raisonnements VOF. \`A partir de la temp\'erature discrète $\overline{T}$ d'un \'el\'ement, on reconstruit un mod\`ele de sous-maille du champ de temp\'erature continu dans l'\'el\'ement en utilisant la position des interfaces. De ce champ de temp\'erature, on d\'eduit la valeur du flux de chaleur $\dot{q}$ et de l'enthalpie de changement de phase $\hat{q}$.\\
Le mod\`ele de sous-maille mis en oeuvre pour l'instant est simple et peu pr\'ecis.
On mod\'elise le flux de chaleur $\dot{q}$ \`a l'interface comme
\begin{equation}
\dot{q} \hat{=} \dfrac{k(\overline{T}-\overline{T_{sat}}}{\delta_{R}+\delta}
\end{equation}
o\`u $k$ est la conductivit\'e thermique du liquide, $\delta_{R}$ l'\'epaisseur \'equivalente de r\'esistance d'interface et $\delta$ une dimension caract\'eristique \'egale au quart de la taille de l'\'el\'ement $\Omega$ (cette valeur permet de retrouver le flux de chaleur moyen lorsque l'interface traverse un \'el\'ement de volume de part en part).\\
On \'ecrit ensuite la puissance volumique de changement de phase sous la forme
suivante :
\begin{equation}
\overline{h} \hat{=} \dfrac{\dot{q} \cdot surface(\Gamma \cap \Omega)}{\overline{V_{\Omega}}}
\end{equation}
La variation de temp\'erature $\overline{T}$ qui en r\'esulte en un pas de temps $\vartriangle t$ est :
\begin{equation}
\vartriangle \overline{T} = - \dfrac{\overline{h}}{\overline{\rho c_{P}}} \Delta t
\end{equation}
Il convient donc de majorer cette valeur pour assurer la stabilit\'e en temps du sch\'ema (pour que la temp\'erature ne passe pas d'une valeur sup\'erieure \`a $T_{sat}$ \`a une valeur inf\'erieure). On impose :
\begin{equation}
|\overline{h}| \leq \overline{\rho c_{P}} | \overline{T} - \overline{T_{sat}}| \vartriangle t
\end{equation}
On a choisi de privil\'egier le transfert de chaleur dans le liquide, ce qui convient pour les calculs du chapitre suivant. Pour \^etre plus pr\'ecis, il faudrait explorer le champ de temp\'erature des mailles voisines pour d\'eterminer quelle est la part du flux de chaleur provenant du liquide et de la vapeur. Une autre solution consiste \`a discr\'etiser deux champs de temp\'erature - un pour le liquide et un pour la vapeur - \`a l'image des m\'ethodes moyenn\'ees.\\
Un mod\`ele encore plus fruste peut \^etre utilis\'e : on utilise une conductivit\'e thermique infinie dans l'\'el\'ement. Ainsi, les \'el\'ements contenant une interface ont toujours une temp\'erature $T = T_{sat}$. La pr\'ecision spatiale de ce mod\`ele n'est pas beaucoup plus faible que celle du pr\'ec\'edent (il est consistant et d'ordre 1 en espace) mais le flux de chaleur est discontinu en temps lorsque l'interface p\'enètre un nouvel \'el\'ement. Ces discontinuit\'es provoquent des \`a-coups de vitesse et de pression qui rendent les calculs difficilement exploitables
\smallskip \\

\textit{\textbf{Flux de masse aux interfaces}}
\smallskip \\
On voudrait d\'efinir le transfert de masse volumique sur le maillage eul\'erien
comme suit :
\begin{equation}
\hat{\dot{m}} \hat{=} \dfrac{1}{\overline{V_{\Omega}}} \int_{\Omega \cap \Gamma} \dot{m} ds
\end{equation}
Puisque $\dot{q} = \mathcal{L} \dot{m}$, la relation entre $\overline{h}$ et $\overline{\dot{m}}$ s'\'ecrit :
\begin{equation}
\overline{\dot{m}} = \dfrac{\overline{h}}{\mathcal{L}}
\end{equation}
Toutefois, pour des raisons de stabilit\'e (num\'erique) de l'\'energie de tension interfaciale, il est n\'ecessaire de r\'eduire les irr\'egularit\'es du champ $\overline{h}$. Le flux de masse $\overline{\dot{m}}$ servira en effet à d\'efinir la divergence discrète du champ de vitesse pour l'\'equation de quantit\'e de mouvement. L'irr\'egularit\'e de $\overline{h}$ se traduira donc par une irr\'egularit\'e du champ de vitesse et donc de la g\'eom\'etrie des interfaces.\\
Nous avons constat\'e des instabilit\'es du profil des interfaces dans certains cas, qui ont pu \^etre r\'eduites par un l\'eger filtrage spatial du flux de masse. Ce filtrage consiste \`a passer de fa\c con conservative de $\overline{\dot{m}}$ \`a $\hat{\dot{m}}$, puis à nouveau \`a $\overline{\dot{m}}$. Le flux total est conserv\'e mais il subit une diffusion num\'erique suffisante pour r\'esoudre ce probl\`eme. De plus, on ajoute au flux de masse $\hat{\dot{m}}$ la contribution singulière $\hat{\dot{m_{s}}}$ des lignes de contact, et \`a $\overleftarrow{\dot{m}}$ la contribution $\overleftarrow{\dot{m_{f}}}$ de l'\'evaporation des films de liquide en paroi.\\
L'expression des flux de masse sur les deux maillages est donc la suivante :
\begin{equation}
\hat{\dot{m}} \hat{=} \dfrac{1}{\mathcal{L}} \hat{\mathcal{G}}(\overline{h}) + \hat{\dot{m_{s}}}
\end{equation}
\begin{equation}
\overline{\dot{m}} \hat{=} \overline{\mathcal{G}}(\hat{\dot{m}} + \overline{\dot{m_{f}}}
\end{equation}

\subsubsection{Divergence discr\`ete de la vitesse}

On consid\`ere un \'el\'ement $\Omega$ du maillage eul\'erien de bord $\partial\Omega$ travers\'e par une interface $\Gamma$. On note $\Omega_{l}$ le volume occup\'e par le liquide et $\Omega_{v}$ le volume occup\'e par la vapeur. On a la relation suivante pour le champ de vitesse continu :
\begin{equation}
\int_{\partial\Omega} v \cdot n ds = \underbrace{\int_{\Omega_{l}} \nabla \cdot v d\Omega}_{=0} + \underbrace{\int_{\Omega_{v}} \nabla \cdot v d\Omega}_{=0} + \int_{\Gamma} (v_{v} - v_{l}) \cdot n ds
\end{equation}
\begin{equation}
\int_{\partial\Omega} v \cdot n ds = \int_{\Gamma} \dot{m} \left( \dfrac{1}{\rho_{v}} - \dfrac{1}{\rho_{l}}\right) ds
\end{equation}
Cette propri\'et\'e des \'equations continues se traduit sous la forme suivante, o\`u l'on d\'efinit la divergence discrète de la vitesse dans l'\'el\'ement $\Omega$ :
\begin{equation}
\overline{\nabla \cdot \overline{v}}(\Omega) \hat{=} \overline{\dot{m}} \left( \dfrac{1}{\rho_{v}} - \dfrac{1}{\rho_{l}}\right) \label{eq:FTD_divV}
\end{equation}
Le sch\'ema num\'erique de projection assure alors la propri\'et\'e suivante du
champ de vitesse discret quel que soit le volume $\Omega$ r\'eunion d'\'el\'ements du maillage eul\'erien :
\begin{equation}
\int_{\partial\Omega} \overline{v} \cdot n ds =\overline{\nabla \cdot \overline{v}}(\Omega) \overline{V_{\Omega}} \label{eq:FTD_bilanMasse}
\end{equation}

\subsubsection{Variation de volume}

Consid\'erons un volume de fluide $\Omega$ de bord $\partial\Omega$ dans l'espace continu. En utilisant l'\'equation \ref{eq:Indicatrice_phase}, la variation du volume de gaz dans ce volume s'\'ecrit :
\begin{equation}
\partial_{t} \int_{\Omega} I d\Omega = \dfrac{1}{\rho_{v}-\rho_{l}} \partial_{t} \int_{\Omega} \rho d\Omega = - \dfrac{1}{\rho_{v}-\rho_{l}} \int_{\partial\Omega} \rho v \cdot n ds \label{eq:FTD_varVolumeGaz}
\end{equation}
En s'inspirant de cette \'equation, on d\'efinit la grandeur discr\`ete suivante sur les \'el\'ements du maillage eul\'erien, o\`u la valeur de $\rho$ reste à d\'efinir :
\begin{equation}
\overline{V'} \hat{=} \dfrac{1}{(\rho_{l} - \rho_{v}) volume(\Omega)} \int_{\partial\Omega} \rho v \cdot n ds
\end{equation}
Pour la suite, on veut que $V'$ v\'erifie la propri\'et\'e globale suivante pour un ensemble d'\'el\'ements $\Omega_{i}$ contenant une interface (on s\'epare les frontières de $\Omega_{i}$ en trois domaines, selon que la fronti\`ere est en contact avec un \'el\'ement plein de liquide, de vapeur ou mixte) :
\begin{equation}
\int_{\Gamma_{l}} \rho_{l} v \cdot n ds + \int_{\Gamma_{v}} \rho_{v} v \cdot n ds + \int_{\Gamma_{b}} \rho v \cdot n ds = (\rho_{l} - \rho_{v}) \int_{\Omega i} V' d\Omega \label{eq:FTD_Vprime}
\end{equation}
Pour obtenir cette propri\'et\'e, la masse volumique $\rho$ sur une face doit \^etre calcul\'ee de la fa\c con suivante en fonction des masses volumiques $\rho_{1}$ et $\rho_{2}$ des \'el\'ements voisins :
\begin{equation}
\rho_{1} = \rho_{l} ou \rho_{2} = \rho_{l} \Rightarrow \rho \hat{=} \rho_{l},
\end{equation}
\begin{equation}
\rho_{1} = \rho_{v} ou \rho_{2} = \rho_{v} \Rightarrow \rho \hat{=} \rho_{v},
\end{equation}
sinon
\begin{equation}
\rho \hat{=} (\rho_{l} + \rho_{v}) /2 \label{eq:FTD_approxMasseVolumique}
\end{equation}
Le choix fait pour le troisi\`eme cas n'est pas imp\'eratif et on a m\^eme int\'er\^et \`a utiliser une meilleure approximation de la masse volumique moyenne sur la face. Il n'a cependant pas d'incidence sur les propri\'et\'es de conservation que l'on veut obtenir.\\
Remarque : cette valeur de la masse volumique aux faces est sp\'ecifique au calcul de $V'$ (on utilise une autre expression pour les bilans de quantit\'e de mouvement).

\subsubsection{Bilan de masse des phases}

Le but de cette d\'emonstration un peu technique est d'obtenir une condition g\'eom\'etrique sur le d\'eplacement des noeuds de l'interface pour que le bilan de masse des inclusions soit respect\'e exactement. Comme nous l'avons mentionn\'e au d\'ebut du chapitre, cette propri\'et\'e est un ingr\'edient essentiel et non trivial de la m\'ethode num\'erique. La d\'emonstration commence par la d\'efinition du bilan de masse continu.
\smallskip \\

\textit{\textbf{D\'efinition du bilan de masse des phases}}
\smallskip \\

On veut assurer la conservation de la masse des phases dans la formulation discrète des \'equations. Pour un domaine $\Omega(t)$ born\'e par des interfaces mobiles $\Gamma_{i}$ et des bords fixes $\Gamma_{b}$ et ne contenant qu'une seule phase de masse volumique $\rho$, le bilan continu s'\'ecrit :
\begin{equation}
\dfrac{\partial}{\partial t} \int_{\Omega(t)} \rho d\Omega = \underbrace{\int_{\Gamma i} -\dot{m} n \cdot n_{e} ds}_{transfert\:de\:masse} - \underbrace{\int_{\Gamma b} \rho v \cdot n_{e} ds}_{entr\'ee\:de\:fluide}
\end{equation}
o\`u $n$ est la normale aux interfaces dirig\'ee vers la vapeur et $n_{e}$ est la normale ext\'erieure au domaine $\Omega$. En divisant par $\rho$ (constant dans les phases), on obtient une \'equation \'equivalente qui exprime le bilan de volume des phases :
\begin{equation}
\dfrac{\partial}{\partial t} \int_{\Omega(t)} d\Omega = \partial_{t} vol.(\Omega) = \int_{\Gamma i} - \dfrac{\dot{m}}{\rho} n \cdot n_{e} ds - \int_{\Gamma b} - v \cdot n_{e} ds \label{eq:FTD_sommeVarVolume}
\end{equation}
Le volume $\Omega$ des interfaces discr\'etis\'ees est calculable exactement, de m\^eme que l'int\'egrale du d\'ebit volumique sur les bords et du flux de masse sur les interfaces. On cherche \`a exprimer une condition locale sur le d\'eplacement des interfaces pour que le bilan discret global soit v\'erifi\'e.
\smallskip \\

\textit{\textbf{Bilan de volume d'une interface}}
\smallskip \\

Consid\'erons une portion d'interface connexe $\Gamma_{i}$ compl\`ete (si cette surface a des bords, ils sont situ\'es sur un bord du domaine fluide). Cette interface divise le fluide en deux parties $\Omega_{l}$ et $\Omega_{v}$
d\'esign\'ees en fonction de la phase directement adjacente \`a l'interface, chaque volume contenant \'eventuellement d'autres interfaces $\Gamma_{j}$ . Pour chacun des deux volumes, on consid\`ere la somme des variations de volume \ref{eq:FTD_sommeVarVolume} des phases qu'il contient. Les interfaces $\Gamma_{j}$ int\'erieures au volume contribuent pour deux termes au bilan, une fois pour le liquide et une fois pour la vapeur. Le bilan total s'\'ecrit (par exemple pour le volume $\Omega_{l}$) :
\begin{equation}
\partial_{t}vol.(\Omega_{l})\:=\:\int_{\Gamma i} - \dfrac{\dot{m}}{\rho_{l}} ds \:-\:\int_{\Gamma b} v \cdot n_{e}ds\:+\:\int_{\Gamma j} \left( -\dfrac{\dot{m}}{\rho_{l}}+\dfrac{\dot{m}}{\rho_{v}} \right) ds \label{eq:FTD_bilanTot}
\end{equation}
Consid\'erons les deux derniers termes de cette somme. Le dernier a une contrepartie discrète exacte donn\'ee par l'\'equation \ref{eq:FTD_divV} :
\begin{equation}
\int_{\Gamma j} \dot{m} \left( \dfrac{1}{\rho_{v}}-\dfrac{1}{\rho_{l}} \right) ds\:-\:\int_{\Gamma b} v \cdot n_{e}ds\:=\:\int_{\overline{\Omega}l} (\nabla \cdot v)d\Omega\:-\:\int_{\Gamma b} v \cdot n_{e}ds
\end{equation}
Le bilan de masse discret \ref{eq:FTD_bilanMasse} ne s'applique qu'\`a un domaine born\'e par des faces du maillage. Consid\'erons donc les volumes discrets $\overline{\Omega}_{v}$, $\overline{\Omega}_{l}$ et $\overline{\Omega}_{i}$. Le bilan discret appliqu\'e \`a $\overline{\Omega}_{v}$ et $\overline{\Omega}_{l}$ donne, par exemple sur $\overline{\Omega}_{l}$ :
\begin{equation}
\int_{\Gamma j} \dot{m} \left( \dfrac{1}{\rho_{v}}-\dfrac{1}{\rho_{l}} \right) ds\:-\:\int_{\Gamma b,l} v \cdot n_{e}ds\:=\:\int_{\overline{\Gamma l}} v \cdot n_{e}ds
\end{equation}
Utilisons cette expression dans l'\'equation \ref{eq:FTD_bilanTot}. Le bord $\Gamma_{b}$ se d\'ecompose en $\Gamma_{l,b}$ (pris en compte dans le bilan discret) et $\Gamma_{b,i} \cap \Gamma_{b}$ qui n'est pas pris en compte.
\begin{equation}
\partial_{t}vol.(\Omega_{l})\:=\:\int_{\Gamma i}-\dfrac{\dot{m}}{\rho_{l}}n\cdot n_{e}ds\:+\:\int_{\overline{\Gamma}l}v \cdot n_{e} ds\:+\:\int_{\Gamma b,i \cap \Omega l} v \cdot n_{e} ds
\end{equation}
\begin{equation}
=\:-\partial_{t}vol.(\Omega_{v})\:=\:-\int_{\Gamma i}\dfrac{\dot{m}}{\rho_{v}}n\cdot n_{e}ds\:-\:\int_{\overline{\Gamma}v}v \cdot n_{e} ds\:+\:\int_{\Gamma b,i \cap \Omega v} v \cdot n_{e} ds
\end{equation}
Les int\'egrales sur $\Gamma_{b,i}$ n'ont pas de contrepartie discrète exacte (le domaine d'int\'egration est une fraction de maille qui change au cours du pas de temps), sauf si la vitesse $v \cdot n_{e}$ est explicitement nulle (condition aux limites de paroi fixe par exemple). Ainsi, le bilan discret de volume des phases est exact seulement si les lignes de contact sont situ\'ees sur des bords ferm\'es du domaine. Supposons pour la suite que c'est le cas.
\smallskip \\

\textit{\textbf{Expression du bilan de volume discret d'une interface}}
\smallskip \\
En multipliant $\partial_{t}vol.(\Omega_{l})$ par $\rho_{l}$ et $\partial_{t}vol.(\Omega_{v})$ par $\rho_{v}$ et en sommant, on a :
\begin{equation}
\rho_{v}\partial_{t}vol.(\Omega_{v})\:-\:\rho_{l}\partial_{t}vol.(\Omega_{v})\:=\:\rho_{l}\int_{\overline{\Gamma}_{l}} v \cdot n_{e} ds\:+\:\rho_{v}\int_{\overline{\Gamma}_{v}} v \cdot n_{e} ds
\end{equation}
Enfin, en utilisant la propri\'et\'e \ref{eq:FTD_Vprime} :
\begin{equation}
\partial_{t}vol.(\Omega_{v})\:=\:\int_{\overline{\Omega} i} V' d\Omega
\end{equation}
Comme les vitesses discrètes sont constantes au cours d'un pas de temps $\vartriangle t$, on a :
\begin{equation}
\vartriangle vol.(\Omega_{v})\:=\:\vartriangle t \int_{\overline{\Omega}_{i}} V' d\Omega
\end{equation}
Pour assurer la conservation du volume des phases, il faut donc s'assurer que la variation de volume g\'eom\'etrique de chaque portion d'interface connexe au cours du pas de temps est \'egale \`a l'int\'egrale discr\`ete de $V'$.
Pour cela, on attribue \`a chaque noeud d'interface une variation de volume $V'_{i}$ , \`a l'aide des relations de passage conservatives. Le d\'eplacement de chaque noeud devra alors contribuer \`a la variation de volume qui lui est associ\'ee.
\smallskip \\

\textit{\textbf{Bilan de volume d"une interface}}
\smallskip \\

On a obtenu une condition g\'eom\'etrique sur le d\'eplacement des interfaces et les vitesses sur le maillage fixe pour respecter le bilan de masse exact des phases. Cette condition utilise uniquement la grandeur locale $V'$ pour le maillage fixe.On a essentiellement utilis\'e deux ingr\'edients :\\
- le bilan exact de volume sur le champ de vitesse discret dans les phases,
procur\'e par le sch\'ema de projection (\'equation \ref{eq:FTD_bilanMasse}),\\
- et la r\'epartition conservative des flux de masse $\dot{m}$ , sur les volumes contenant les interfaces.\\
Le bilan \`a partir de la variation de volume $V'$ est semi-local : les $V'$ et le d\'eplacement de chaque noeud ne d\'ependent que des vitesses eul\'eriennes sur un voisinage du noeud. On n'a cependant pas de propri\'et\'e de bilan local. En particulier, on ne peut pas \'ecrire l'\'equation id\'eale suivante sur la variation de l'indicatrice au cours du pas de temps, qui est l'\'equivalent discret de l'\'equation \ref{eq:FTD_varVolumeGaz} :
\begin{equation}
\vartriangle \int_{\Omega} I\:d\Omega\:=\:-\dfrac{\vartriangle t}{\rho_{v}-\rho_{l}} \int_{\partial\Omega} \rho v \cdot n\:ds
\end{equation}
Cette propri\'et\'e très int\'eressante n\'ecessiterait plusieurs ingr\'edients dont au moins les deux suivants :\\
- une expression exacte de $V'$, notamment dans l'expression (le calcul "exact" est \`a la base des m\'ethodes VOF, et sa construction est loin d'\^etre triviale),\\
- un transport des interfaces qui respecte exactement la variation de volume $V'$ \'el\'ement par \'el\'ement (il faudrait pour cela que l'interface ait suffisamment de noeuds, au moins un par \'el\'ement, et cette condition est
contradictoire avec le crit\`ere de stabilit\'e de la tension de surface qui impose exactement l'inverse).\\
En fait, une telle propri\'et\'e oblige pratiquement \`a utiliser la formulation VOF et m\^eme ainsi, elle est difficile \`a obtenir (voir \cite{Aulisa2003}).
Cet algorithme pr\'esente cependant deux avantages par rapport \`a une correction globale telle que celle mise en oeuvre par Juric \cite{Shin2002} :\\
- on obtient non seulement une conservation globale exacte du volume, mais aussi une conservation "quasi-locale",\\
- il tient implicitement compte des flux de masse aux entr\'ees du domaine et des variations de volume des autres interfaces du domaine.
