Ce cas test correspond \`a la diffusion d'un produit de sinus sur les composantes de la vitesse. Il a pour but le test de la coh\'erence de l'impl\'ementation de la diffusion turbulente vis-\`a-vis du sch\'ema en temps. On fait le test avec un mod\'ele particulier pour le terme sous-maille $\partial_j\left(\overline{U_i U_j} - \overline{U}_i\overline{U}_j\right)$ de l'\'equation de conservation de la quantit\'e de mouvement dans la formulation 'Favre' qui est choisit pour correspondre \`a la diffusion visqueuse classique (sans le gradient transpo\'e ou la divergence).

\subsection{Solution analytique}

Soit
\begin{equation}
\tau_{ij} = \overline{U_i U_j} - \overline{U}_i\overline{U}_j
\end{equation}
Dans le cadre de cas test, on a le mod\`ele
\begin{equation}
\tau_{ij} = - \mu \frac{\partial U_i}{\partial x_j}
\end{equation}

On note
\begin{equation}
D_i = \frac{1}{\rho} \frac{\partial \rho \tau_{ij}}{\partial x_j}.
\end{equation}
Dans la formulation 'Favre', le mod\`ele est impl\'ement\'e comme
\begin{equation}
\frac{\partial U_{i}}{\partial t} = NS^* + D_i
\end{equation}
Pour le cas test, on d\'esactive $NS^*$ et on se place \`a temp\'erature constante donc on a
\begin{equation}
\frac{\partial U_{i}}{\partial t} = \mu \frac{\partial}{\partial x_j} \frac{\partial U_i}{\partial x_j}
\end{equation}

Notons $V_0 = 0.001$ et $\tau=2\pi$.
La condition initiale sur la vitesse est :
\begin{align*}
V_x ={}& 2 V_0 \cos(8 \tau x) \cos(8 \tau z) \\
V_y ={}& 0 \\
V_z ={}& 2 V_0 \sin(8 \tau x) \sin(8 \tau z)
\end{align*}

La solution analytique sur la vitesse est :
\begin{align*}
V_x ={}& e^{-\alpha t} 2 V_0 \cos(8 \tau x) \cos(8 \tau z) \\
V_y ={}& 0 \\
V_z ={}& e^{-\alpha t} 2 V_0 \sin(8 \tau x) \sin(8 \tau z)
\end{align*}
avec $\alpha = 2 (8\tau)^2 \mu$.

On va donn\'e les solutions analytiques pour $D_i$ selon diff\'erents cas.
Lors de la validation, on notera {\textsf simu\_dv\_i} le r\'esultat de la
simulation pour $D_i$, et {\textsf ana\_dv\_i} la solution analytique.
La diff\'erence entre les deux sera not\'e \textsf{error\_dv\_i}. Si tout se passe bien, les r\'esultats de la simulation sont proches de la solution analytique et l'erreur est tr\'es faible devant la valeur totale.

\subsection{V\'erification des r\'esultats de la simulation}

On pr\'esente aussi la solution analytique au temps initial dans {\textsf ana\_vz\_tinit}. Au temps final de la simulation, on compare le r\'esultat de la simulation {\textsf simu\_vz} \`a la solution analytique d\'ecrite pr\'ec\'edemment {\textsf ana\_vz\_tfinal}. La diff\'erence entre les deux est \textsf{error\_vz}.

