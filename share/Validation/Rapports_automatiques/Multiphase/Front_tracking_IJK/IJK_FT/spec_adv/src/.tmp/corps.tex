\section{Introduction}

Validation made by : Gabriel Ramirez.\\Report generated  02/09/2021.
\par\subsection{Description}

\par\subsection{Parameters TRUST }
\begin{itemize}
\item Version TRUST : 1.8.2
\end{itemize}
\par\subsection{Test cases}
\begin{itemize}
\item ADV\_X00/RUN00/spec\_bulles\_point2.data : \textit{}
\item NO\_ADV\_X00/RUN00/spec\_bulles\_point2.data : \textit{}
\item ADV\_0X0/RUN00/spec\_bulles\_point2.data : \textit{}
\item NO\_ADV\_0X0/RUN00/spec\_bulles\_point2.data : \textit{}
\item ADV\_010/RUN00/spec\_bulles\_point2.data : \textit{}
\item NO\_ADV\_010/RUN00/spec\_bulles\_point2.data : \textit{}
\item ADV\_001/RUN00/spec\_bulles\_point2.data : \textit{}
\item NO\_ADV\_001/RUN00/spec\_bulles\_point2.data : \textit{}
\item ADV\_101/RUN00/spec\_bulles\_point2.data : \textit{}
\item NO\_ADV\_101/RUN00/spec\_bulles\_point2.data : \textit{}
\end{itemize}

% Debut Chapitre
\section{Purpose}
Valider l'advection du champ de force. 

% Debut Chapitre
\section{Description du probl\`{e}me}
Le texte n'est pas a jour du tout. La suite de g\'{e}nr\'{e}rateur al\'{e}atoires se trouve dans .../OUT/random...out.
On impose un champ de force dans le domaine. Le champs de force est donn\'{e} par les \'{e}quations suivantes : 
 $F_{ph} = \mathcal{TF}(b(k,t_{i+1}) - \frac{k(k \cdot b(k, t_{i+1}))}{k \cdot k})$ 
$ b(k,t_{i+1}) = b(k,t_i)(1-\frac{\Delta t}{T_L}) + e_i(k,t) (2\sigma^2 \frac{\Delta t}{T_L})^{1/2}$
Le domaine est un cube de c\^{o}t\'{e} 01, allant de -0.005 a  0.005.
Chaque face est p\'{e}riodique
La viscosit\'{e} cin\'{e}matique est fix\'{e}e \`{a} 3.0128062836164946e-07 . La masse volumique \`{a} 1171.3. Il n'y a pas de phase gazeuse.


% Debut Chapitre
\section{Description du cas}
Le domaine est discr\'{e}tis\'{e} en $128^3$ \'{e}l\'{e}ments tous de m\^{e}me taille.

 On dispose quatre segments de relev\'{e} align\'{e}s selon X, quatre segments selon Y et 4 segments selon Z. 

% Debut Chapitre
\section{R\'{e}sultats}
On montre ici la visualitation des champs de la force spectrale ajout\'{e}e,
 obtenus avec l'outil de visualisation visit .


Force spectrale, avec reprises
 obtenus avec l'outil de visualisation visit .


Forces spectrales, obtenues d'une traite
 obtenus avec l'outil de visualisation visit .


Les marques correspondent aux premiers relev\'{e}s tels que l'autocorr\'{e}lation soit nulle ou n\'{e}gative.
     


% Debut Chapitre
\section{Conclusion}
En conclusion, on n'observe pas d'abh\'{e}rrance

% Debut Chapitre
\section{Computer performance}


% Debut tableau

