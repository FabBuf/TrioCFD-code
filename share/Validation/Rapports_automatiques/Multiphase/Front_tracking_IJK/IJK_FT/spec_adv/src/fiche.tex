\documentclass[12pt,a4paper]{article}
\usepackage[utf8]{inputenc}
\usepackage[french]{babel}
\usepackage[T1]{fontenc}

\usepackage{amsmath}
\usepackage{listings}
\usepackage{amsfonts}
\usepackage{amssymb}
\usepackage{bm}
%%%%%%%%%%%%%%%%%%%%%%%%%%%%%%%%%%%%%%%%%%%%%%%
%%% POUR LA BIBLIO
\usepackage{csquotes}
%%%%%%%%%%%%%%%%%%%%%%%%%%%%%%%%%%%%%%%%%%%%%%%

%%%%%%% Pour tourner un tabular
\usepackage{adjustbox}

\usepackage{xcolor}
\usepackage{graphicx}
\usepackage{subcaption}
\usepackage[makeroom]{cancel}
\usepackage{amssymb}
\usepackage{lmodern}
\usepackage{footnote}
\usepackage{mdframed}
\makesavenoteenv{tabular}


\usepackage[left=2cm,right=2cm,top=2cm,bottom=2cm]{geometry}

\usepackage[
backend=biber,% On utilise biber pour la compilation biblatex
citestyle=numeric,
bibstyle=numeric,
sorting=nyt, % Tri dans l'ordre : nom, année, titre.
maxnames=2, % Les citations dans le texte avec \textcite n'affiche que le nom du premier auteur et met un "et al.". Changer en et coll. ?
maxbibnames=99, % Mettre tous les noms dans la biblio (ne met pas de et al. dans la bibliographie)
language=french, % test du français
alldates=year, % la date n'affiche que l'année
isbn=false, % Ne pas afficher l'isbn
url=false, % Ne pas afficher l'url
doi=false, % Afficher le DOI
eprint=false, % Ne pas afficher le numéro eprint
%refsegment=chapter, % pour faire des bibliographies par chapitre
backref=false % Ajouter une mention renvoyant où la publication a été citée
]{biblatex}
\addbibresource{/volatile/biblio/biblio.bib} 

\renewbibmacro{in:}{}
\newbibmacro{string+doiurlisbn}[1]{%
  \iffieldundef{doi}{%
    \iffieldundef{url}{%
      \iffieldundef{isbn}{%
        \iffieldundef{issn}{%
          #1%
}{\href{https://books.google.com/books?vid=ISSN\thefield{issn}}{#1}}%
}{\href{https://books.google.com/books?vid=ISBN\thefield{isbn}}{#1}}%
    }{\href{\thefield{url}}{#1}}%
  }{\href{https://doi.org/\thefield{doi}}{#1}}%
}

\DeclareFieldFormat{title}{\usebibmacro{string+doiurlisbn}{\mkbibemph{#1}}}
\DeclareFieldFormat[article,incollection,inproceedings,thesis,inbook]{title}%
    {\usebibmacro{string+doiurlisbn}{\mkbibquote{#1}}} 


\usepackage{hyperref}

\usepackage{hyperref}
\hypersetup{
  % backref=true,               %% permet d'ajouter des liens dans...
  % pagebackref=true,           %% ...les bibliographies ;
  % hyperindex=true,            %% ajoute des liens dans les index ;
  breaklinks=true,              %% permet le retour à la ligne dans les liens trop longs ;
  colorlinks=true,              %% colorise les liens ;
  urlcolor= blue,               %% couleur des hyperliens ;
  linkcolor=blue,               %% couleur des liens internes ;
  citecolor=blue,               %% couleur des liens de citations ;
  % bookmarks=true,             %% créé des signets pour Acrobat ;
  % bookmarksopen=true,         %% si les signets Acrobat sont créés, les afficher complètement ;
  pdftitle={simus_grossieres},             %% informations apparaissant...
  pdfauthor={Gabriel Ramirez},  %% ...dans les informations...
  pdfsubject={Écoulements turbulents à bulles - expérimental}, %% ...du document...
  pdfkeywords={Écoulements à bulles, turbulence},  %% ...dans le lecteur pdf.
  pdfencoding={unicode}
}



\author{Gabriel Ramirez}
\title{Validation advection}

\setlength{\parskip}{1em}

%%%%%%%%% COMMANDES UTILES %%%%%%%%%%%
%% \let
% NOTATIONS
\let\ol\overline
\let\ul\underline
\let\la\langle
\let\ra\rangle
\let\tm\times
% ALPHABET GREC
\let\a\alpha
\let\b\beta
\let\d\delta
\let\D\Delta
\let\e\epsilon
\let\k\kappa
\let\g\gamma
\let\G\Gamma
\let\l\lambda
\let\L\Lambda
\let\n\eta
\let\o\omega
\let\p\partial
\let\r\rho
\let\s\sigma
\let\t\tau

\newcommand{\Cpx}{\mathbb{C}}
\newcommand{\Ree}{\mathbb{R}}
\newcommand{\Rey}{\mathit{Re}}
\newcommand{\rei}{\mathit{Re_{b;isolée}}}
\newcommand{\ree}{\mathit{Re_{b;essaim}}}
\newcommand{\WIF}{\mathit{WIF}}
\newcommand{\hf}{\hat{f}}
\newcommand\piL[1]{\frac{{#1} \pi}{L}}

%% \newcommand
% EDITION DE TEXTE
\newcommand\tab[1][1cm]{\hspace*{#1}}
\newcommand\tb[1]{\textbf{#1}}
\newcommand\ti[1]{\textit{#1}}
\newcommand\mc[1]{\mathcal{#1}}
\newcommand\mi[1]{\mathit{#1}}
\newcommand\fnm[1]{$^{#1}$}
% GRANDEURS CONSTRUITES
\newcommand\tc{\tilde{\chi}}
\newcommand\tq{\tilde{q}}
\newcommand\tu{\tilde{u}}
\newcommand\tv{\tilde{v}}
\newcommand\tru{\tilde{\rho u}}
\newcommand\sq{\sigma(q)_{t,sonde}}
% INTEGRALE ET SOMME
\newcommand\TF{\int\limits}
\newcommand\TFF{\iint\limits}
\newcommand\TFFF{\iiint\limits}
\newcommand\som{\sum\limits}
% VECTEURS DÉTAILLÉS
\newcommand\vect[3]{ \left(\begin{array}{c} {#1}\\ {#2} \\{#3}\end{array}\right) }
\newcommand\Vect[1]{ \vect{#1 _1}{#1 _2}{#1 _3} }
\newcommand\VectUn[1]{ \vect{#1 _{10}}{#1 _2}{#1 _3} }
\newcommand\VectUnDeux[1]{ \vect{#1 _{10}}{#1 _{20}}{#1 _3} }
\newcommand\VectZero[1]{ \vect{#1 _{10}}{#1 _{20}}{#1 _{30}} }
\newcommand\VectUnZeroUn[1]{ \vect{#1 _{10}}{0}{#1 _{10}} }
\newcommand\VectUnUnZero[1]{ \vect{#1 _{10}}{#1 _{10}}{0} }
\newcommand\VectUnUnUn[1]{ \vect{#1 _{10}}{#1 _{10}}{#1 _{10}} }
%%%%%%%%%%% COMMANDES UTILES %%%%%%%%%%%%%%


\begin{document}

\section{Description du probleme}
Un écoulement monophasique de fluide de masse volumique $\r$ est considéré. Cet écoulement périodique dans les trois directions est simulé dans un domaine cubique de côté $L$. La gravité est nulle ($\bm{g} =  \bm{0}$), la convection et la diffusion sont ignorées. Une force extérieure $f$ est imposée au fluide. Seul l'effet du champ de pression ($\bm{\na} p$) est conservé. Les équations résolues dans le domaine sont donc : 

\begin{align}
&\p_t \bm{u} = - \frac{1}{\r} \bm{\nabla} p + \bm{f} \\                
&\bm{ \nabla \cdot u } = 0 \\
\end{align}

Les différentes forces testées sont : 

\begin{align}
&\bm{f}(x;t)  = 2 cos(\k_1 x - \o_{1/2} t) \bm{e_y} \\
&\bm{f}(x;t)  = 2 cos(\k_1 x - \o_{1/2} t) \bm{e_x} \\
&\bm{f}(x;t)  = 2 cos(\k_1 y - \o_{1/2} t) \bm{e_y} \\
&\bm{f}(x;t)  = 2 cos(\k_1 z - \o_{1/2} t) \bm{e_z} \\
&\bm{f}(x;t)  = 2 cos(\k_1 (x-z) - \o_{1/2} t) \frac{1}{\sqrt{2}}(\bm{e_x - e_z}) \\
\end{align}

\begin{table}[h]
\begin{center}
\begin{tabular}{c|c|c|c|c}
$\r(kg.m^{-3})$ & $ L(m)$ &  $\k_n(rad.m^{-1})$        & $\o_q(rad.s^{-1})$ & N \\ \hline
1171.3          & 0.004   &  $2 \pi n / L$             & $2 \pi / q $       & 40
\end{tabular}
\end{center}
\caption{Paramètres physiques et numérique de la simulation}
\end{table}

$\k_n$ est défini de sorte à observer $n$ périodes de $f$ sur le domaine.
$\o_q$ est défini de sorte à translater champ $f$ d'une longueur de domaine selon $x$,$y$ ou $z$ en $q$ secondes.
\section{Description du cas 1}
\subsection{Grid}
Le maillage est cartésien et uniform. Le domaine est discrétisé en $N^3$ cellules cubiques, il s'agit du maillage de calcul. Pour le post-traitement uniquement, un maillage cartésien uniforme de $(2N)^3$ cellules est défini, il s'agit du maillage de post-traitement. Les grandeurs définies aux faces des éléments du maillage de calcul sont interpolées aux centres des mailles du maillage de post-traitement.
\subsection{Model options}
Le premier test correspond à résoudre les équations : 

\begin{align*}
&\bm{f} = 2 cos(\k_1 x - \o_{1/2} t) \bm{e_y} \\
&\p_t \bm{u} = - \frac{1}{\r} \bm{\nabla} p + \bm{f} \\                
&\bm{ \nabla \cdot u } = 0 \\
\end{align*}

$f$ ne dépend que de $x$ et de $t$. $f$ est la seule cause de mouvement dans ce problème. Ainsi les grandeurs du problème sont indépendantes de $y$ et de $z$ ($\p_y = \p_z = 0$ et $p=p(x,t)$ notamment). L'équation de Poisson pour la pression est : $\D p = 0$. Par conséquent $\p_x p  = C(t)$. Le champ de pression respecte la condition de périodicité si $C(t) = 0$. Le champ de pression est donc constant et uniforme.
La composante $y$ de l'équation de conservation de quantité de mouvement donne : 

\begin{align*}
&\p_t u_y = f_y = 2 cos(\k_1 x - \o_{1/2} t)\\ 
\end{align*}

Ce premier test a pour solution analytique :
\begin{mdframed}
\begin{align*} 
&\bm{u}(\bm{x};t) = - \frac{2}{\o_{1/2}} sin(\k_1 x - \o_{1/2} t)  \\
&p(\bm{x};t)=p.   
\end{align*}
\end{mdframed}
\ti{Onde de vitesse}

Le deuxième test correspond à résoudre les équations : 

\begin{align*}
&\bm{f} = 2 cos(\k_1 x - \o_{1/2} t) \bm{e_x} \\
&\p_t \bm{u} = - \frac{1}{\r} \bm{\nabla} p + \bm{f} \\                
&\bm{ \nabla \cdot u } = 0 \\
\end{align*}

De même que pour le test précédent, les grandeurs du problème sont indépendantes de $y$ et de $z$. L’équation de Poisson pour la pression est $\D p = \r \na \cdot f$ se développe donc :  

\begin{align*}
&\p_{xx} p= - 2 \r \k_1 sin(\k_1 x - \o_{1/2} t) \\
&p(x,t) = 2 \frac{\r}{\k_1} sin(\k_1 x - \o_{1/2} t)
\end{align*}

Ce deuxième test a donc pour solution analytique : 

\begin{mdframed}
\begin{align*} 
&p(\bm{x},t) = - 2 \frac{\r}{\k_1} sin(\k_1 x - \o_{1/2} t)  \\
&\bm{u} = \bm{0}.   
\end{align*}
\end{mdframed}
\ti{Onde de pression}

On montre facilement que les tests suivants ont respectivement pour solutions analytiques : 

Troisième test : 
\begin{mdframed}
\begin{align*} 
&p(\bm{x},t) = - 2 \frac{\r}{\k_1} sin(\k_1 y - \o_{1/2} t)  \\
&\bm{u} = \bm{0}.   
\end{align*}
\end{mdframed}
\ti{Onde de pression}

Quatrième test : 
\begin{mdframed}
\begin{align*} 
&p(\bm{x},t) = - 2 \frac{\r}{\k_1} sin(\k_1 z - \o_{1/2} t)  \\
&\bm{u} = \bm{0}.   
\end{align*}
\end{mdframed}
\ti{Onde de pression}

Cinquième test : 
\begin{mdframed}
\begin{align*} 
&p(\bm{x},t) = - \frac{2}{\sqrt{2}} \frac{\r}{\k_1} sin(\k_1 (x-z) - \o_{1/2} t) \\
&\bm{u} = \bm{0}.   
\end{align*}
\end{mdframed}
\ti{Onde de pression}

\tb{Applications numériques} : Le valeurs numériques des paramètres $\k_1$; $\o_{1/2}$ ainsi que les normes des solutions analytiques des tests sont regroupées dans le tableau \ref{tab : AN}

\begin{table}[h]
\begin{center}
\begin{tabular}{c|c|c|c|c|c}
 N  & $\k_1(rad.m^{-1})$        & $\o_{1/2}(rad.s^{-1})$ & $2/ \o_{1/2}$ (sol. 1) & $2 \r / \k_1$ (sol. 2,3,4) & $\sqrt(2) \r / \k_1$ (sol. 5)  \\ \hline
 40 & 1571                      &   12.57                & 0.1592                &   1.491                    & 1.054
\end{tabular}
\end{center}
\caption{Nombre de mailles par direction et applications numériques}
\label{tab : AN}
\end{table}


\subsection{Other options}
Le coeur des tests présentés ici est de valider que la force imposée à l'écoulement est de la forme demandée, notamment en ce qui concerne la translation du champ. Il s'agit ici de valider l'évolution temporelle en $\o_q t$ dans l'argument du cosinus. La résolution physique de l'écoulement vient dans un deuxième temps pour étoffer cette base de test mais elle n'est pas l'objet de la validation.

La force imposée est définie dans le domaine spectral. Une transformée de Fourier inverse \ref{eq : FT-1} permet de transposer cette force au domaine physique (domaine dans lequel les équations sont résolues). Les forces spectrales $\hf_i$ à définir pour observer des champs de la forme $f_i=cos(\bm{\k}_n \cdot \bm{x})$ sont donc des paires de dirac : $\hf_{i;st} = \d(\bm{\k}-\bm{\k_n}) + \d(\bm{\k}+\bm{\k_n})$. Pour translater ce champ dans le temps, deux choix équivalents sont possibles. Le premier consiste à définir les forces mobiles $\bm{\hf_{mb}}$ à partir des forces statiques $\bm{\hf_{st}} $: \ref{eq : choix1}. Le deuxième choix consiste à définir une transformée de Fourier inverse alternative : \ref{eq : choix2}.

\begin{align}
&f(\bm{x};t) = \int \hf(\bm{\k};t) e^{i \bm{\k \cdot x} } d\bm{\k}
\label{eq : FT-1} \\
    &\hf_{mb} = \hf_{st} \cdot e^{-i \bk \cdot \bl_{1/q}(t)}
\label{eq : choix1} \\
    &f(\bm{x};t) = \int \hf(\bm{\k};t) e^{i {\bk \cdot (\bx - \bl_{1/q}(t))} } d\bm{\k},
\label{eq : choix2}
\end{align}

avec $\bl_{1/q}(t)$ le vecteur de translation du champ de force à l'instant $t$. Le deuxième choix est implémenté dans le code de calcul pour des raisons informatiques. Ce qui signifie qu'à tout instant la force spectrale est définie et calculée sans avoir été translatée. La translation est effective lors du passage du domaine spectral au domaine physique uniquement. La valeur du vecteur de déplacement $\bl_{1/q}(t)$ est mise à jour en fonction de $t$. Les correspondances spectral-physique des cas test étudiés sont regroupées dans le tableau \ref{tab : spec_phys}.

\begin{table}
\begin{tabular}{lll}
    Spectral : $\ghf(\bk;t) = [\d(\bm{\k}-\bm{\k_1}) + \d(\bm{\k}+\bm{\k_1})] \bm{e}$ & Translation : $e^{-i \bk \cdot \bl_{1/q}(t)}$ & Physique : $\gf(\bx;t)$  \\ \hline
$\bk_1 = (\k_1; 0; 0)^T$; $\bm{e} = \bm{e_y}$  &  $\bl(t) = (2Lt; 0; 0)^T$      & $ 2 cos(\k_1 x - \o_{1/2} t) \bm{e_y}$       \\
$\bk_1 = (\k_1; 0; 0)^T$; $\bm{e} = \bm{e_x}$  &  $\bl(t) = (2Lt; 0; 0)^T$      & $ 2 cos(\k_1 x - \o_{1/2} t) \bm{e_x}$       \\
$\bk_1 = (0; \k_1; 0)^T$; $\bm{e} = \bm{e_y}$  &  $\bl(t) = (0; 2Lt; 0)^T$      & $ 2 cos(\k_1 y - \o_{1/2} t) \bm{e_y}$       \\
$\bk_1 = (0; 0; \k_1)^T$; $\bm{e} = \bm{e_z}$  &  $\bl(t) = (0; 0; 2Lt)^T$      & $ 2 cos(\k_1 z - \o_{1/2} t) \bm{e_z}$       \\
$\bk_1 = (\k_1; 0; -\k_1)^T$; $\bm{e} = 1/\sqrt{2}(\bm{e_x}-\bm{e_z}) $ & $\bl(t) = (Lt; 0; -Lt)^T$ & $ \frac{2}{\sqrt{2}} cos(\k_1 (x-z) - \o_{1/2} t) (\bm{e_x}-\bm{e_z})$     \\ \hline
\end{tabular}
\caption{Paramètres de  définition de la force spectrale.}
\label{tab : spec_phys}
\end{table}

La pulsation de déplacement $\o_q$ introduite plus haut peut être liée au vecteur de déplacement $\bl_{1/q}(t)$ par la relation : $\o_q = \bk_1 \cdot \bl_{1/q}(t)$. Où $\bl_{1/q}(t) = \frac{1}{q} L t$, dans notre cas $q=1/2$.


\section{results}
La simulation du premier cas est trop gourmande en temps de calcul pour être raisonnablement implémentée telle quelle dans cette validation. La pulsation d'advection du premier cas $\o_{1/2}$ est donc changée en $\o_{1/20}$ (voir définition $\o_q$) ce qui déplace le champ de force 10 fois plus vite et permet de retrouver une évolution satisfaisante en un temps de calcul raisonnable. La figure \ref{fig : 0X0} montre les évolutions des grandeurs pour le premier cas (modifié en accord avec ce qui précède) :
\begin{align}
\gf(\bx;t) = 2cos(\k_1 x - \o_{1/20}t) \bm{e_y}
\end{align} 

La force physique $f_y(x,0,0;t_i)$ en rouge, la vitesse $u_y(x,0,0;t_i)$ en bleu et la pression $p(x,0,0;t_i)$ en vert pour différents instants allant de $t=0.008s$ à $t=0.076s$ sont tracés. On constate bien qu'entre $t_0=0.008s \approx 0s$ et $t_i = 0.0503 \approx t_0 + 1/20 s$, le champ de force $f_y$ (rouge) a été translaté d'une distance $L$ dans la direction $x$. \tb{La translation du champ dans la direction $x$ est validée pour un champ d'onde transversale.}

\begin{figure}
\begin{center}
	\begin{subfigure}[t]{0.4\textwidth}                                                                                                                                   
		\includegraphics[scale=0.23]{../../build/lineout_ADV_0X0_t1.png}
		\caption{ $\bm{t=s}$}
		\label{fig : 0X0_t1}
	\end{subfigure}\hfill
	\begin{subfigure}[t]{0.4\textwidth}
		\includegraphics[scale=0.23]{../../build/lineout_ADV_0X0_t6.png}
		\caption{ $\bm{t=s}$}
		\label{fig : 0X0_t6}
	\end{subfigure}
\\
	\begin{subfigure}[t]{0.4\textwidth}
		\includegraphics[scale=0.23]{../../build/lineout_ADV_0X0_t11.png}
		\caption{ $\bm{t=s}$}
		\label{fig : 0X0_t11}
	\end{subfigure}\hfill
	\begin{subfigure}[t]{0.4\textwidth}
		\includegraphics[scale=0.23]{../../build/lineout_ADV_0X0_t16.png}
		\caption{ $\bm{t=s}$}
		\label{fig : 0X0_t16}
	\end{subfigure}
\\
	\begin{subfigure}[t]{0.4\textwidth}
		\includegraphics[scale=0.23]{../../build/lineout_ADV_0X0_t17.png}
		\caption{ $\bm{t=s}$}
		\label{fig : 0X0_t21}
	\end{subfigure}	
\end{center}
\caption{Évolution le long de $(x,y=0,z=0)$ de : la force $f_y$ (rouge), la vitesse $u_y$ (bleu), la pression $p$ (vert).}
\label{fig : 0X0}
\end{figure}


De plus, on observe que le champ de pression reste constant et uniforme comme prédit. $p = 0$. L’évolution de la vitesse $u_y$, montrée seule sur les figures \ref{fig : v_0X0} permet de voir que l'ordre de grandeur de l'amplitude maximale observée correspond à l'amplitude prédite. En mettant en regard \ref{fig : v_0X0} et \ref{fig : 0X0} on constate que la vitesse $u_y$ est déphasé de $\pi/4$ par rapport à la force $f_y$. Cependant il n'est pas possible de conclure que la solution analytique est strictement respectée car l'amplitude de la vitesse $u_y$ varie au cours du temps. On relève notamment que l'amplitude s'annule pour $t=0.05s$ avec un changement brusque de signe de $u_y$. La tendance observée en faisant défiler $u_y$ pour chaque instant laisse supposer que l'amplitude s'annule encore pour $t=0.1$. Une simulation plus longue en remplaçant $\o_{1/20}$ par $\o_{1/2}$ (non reproduite ici, mais facilement reproductible) montre que cette amplitude s'annule pour $t=0.5s$ et pour $t=1s$. On suppose donc que le champ de vitesse simulé est de la forme :

\begin{align}
u_y(x;t) = sin(\frac{\o_q \pi}{L} t ) \cdot ( - \frac{2}{\o_{q}} sin(\k_1 x - \o_{1/2} t))
\end{align}


\begin{figure}
\begin{center}
	\begin{subfigure}[t]{0.4\textwidth}                                                                                                                                   
		\includegraphics[scale=0.23]{\orig/lineout_velocity_ADV_0X0_t1.png}
		\label{fig : v_0X0_t1}
	\end{subfigure}\hfill
	\begin{subfigure}[t]{0.4\textwidth}
		\includegraphics[scale=0.23]{\orig/lineout_velocity_ADV_0X0_t6.png}
		\label{fig : v_0X0_t6}
	\end{subfigure}
\\
	\begin{subfigure}[t]{0.4\textwidth}
		\includegraphics[scale=0.23]{\orig/lineout_velocity_ADV_0X0_t11.png}
		\label{fig : v_0X0_t11}
	\end{subfigure}\hfill
	\begin{subfigure}[t]{0.4\textwidth}
		\includegraphics[scale=0.23]{\orig/lineout_velocity_ADV_0X0_t16.png}
		\label{fig : v_0X0_t16}
	\end{subfigure}
\end{center}
\caption{Évolution le long de $(x,y=0,z=0)$ de : la force $f_y$ (rouge), la vitesse $u_y$ (bleu), la pression $p$ (vert).}
\label{fig : v_0X0}
\end{figure}




La figure \ref{fig : X00} montre les évolutions des grandeurs pour le deuxième cas :
\begin{align}
\gf(\bx;t) = 2cos(\k_1 x - \o_{1/2}t) \bm{e_x}
\end{align} 

La force physique $f_x(x,0,0;t_i)$ en rouge, la vitesse $u_x(x,0,0;t_i)$ en bleu et la pression $p(x,0,0;t_i)$ en vert pour différents instants allant de $t=0.012s$ à $t=0.26s$ sont tracés. On constate bien qu'entre $t_0=0.012s \approx 0s$ et $t_i = 0.26 \approx t_0 + 1/4 s$, le champ de force $f_y$ (rouge) a été translaté d'une distance $L/2$ dans la direction $x$. \tb{La translation du champ dans la direction $x$ est validée pour un champ d'onde transversale.}

\begin{figure}
\begin{center}
	\begin{subfigure}[t]{0.4\textwidth}                                                                                                                                   
		\includegraphics[scale=0.23]{../../build/lineout_ADV_X00_t1.png}
		\caption{ $\bm{t=s}$}
		\label{fig : X00_t1}
	\end{subfigure}\hfill
	\begin{subfigure}[t]{0.4\textwidth}
		\includegraphics[scale=0.23]{../../build/lineout_ADV_X00_t6.png}
		\caption{ $\bm{t=s}$}
		\label{fig : X00_t6}
	\end{subfigure}
\\
	\begin{subfigure}[t]{0.4\textwidth}
		\includegraphics[scale=0.23]{../../build/lineout_ADV_X00_t11.png}
		\caption{ $\bm{t=s}$}
		\label{fig : X00_t11}
	\end{subfigure}\hfill
	\begin{subfigure}[t]{0.4\textwidth}
		\includegraphics[scale=0.23]{../../build/lineout_ADV_X00_t16.png}
		\caption{ $\bm{t=s}$}
		\label{fig : X00_t16}
	\end{subfigure}
\\
	\begin{subfigure}[t]{0.4\textwidth}
		\includegraphics[scale=0.23]{../../build/lineout_ADV_X00_t21.png}
		\caption{ $\bm{t=s}$}
		\label{fig : X00_t21}
	\end{subfigure}	
\end{center}
\caption{Évolution le long de $(x,y=0,z=0)$ de : la force $f_x$ (rouge), la vitesse $u_x$ (bleu), la pression $p$ (vert).}
\label{fig : X00}
\end{figure}

L'observation du champ de vitesse seul montre qu'il n'est pas strictement nul. Son amplitude est de l'ordre de $|u_x|_{max}=10^{-5} m.s^{-1}$. Son évolution est de la forme $u_y = cos(\k_1 \cdot x)cos(\D x \cdot x)$. Cette évolution est due à l'absence de phénomènes diffusifs dans l'écoulement, ce qui laisse un courant parasite s'établir. La résolution de la pression, est satisfaisante sur deux aspects. Le premier est que l'amplitude de la pression est de l'ordre de grandeur attendu : $\frac{2\r}{\k_1}\approx 1.5$, le second est l'évolution du champ qui est bien déphasée de $\pi/4$ par rapport au champ $f$. Ce qui est sujet à discussion est "l'offset" fluctuant qui est observé pour ce champ. En effet, la pression semble évoluer comme :

\begin{align}
p(\bm{x},t) = 2 \frac{\r}{\k_1} (sin(\k_1 \frac{L}{2})- sin(\k_1 x - \o_{1/2} t) )
\end{align}

Cette correction a pour effet d'assurer $p(x=x_{min}) = p(x=x_{max}) = 0$.
Les figures \ref{fig : 010} et \ref{fig : 001} montrent respectivement les évolutions des grandeurs pour le troisième et le quatrième cas :
\begin{align}
\gf(\bx;t) = 2cos(\k_1 y - \o_{1/2}t) \bm{e_y},
\gf(\bx;t) = 2cos(\k_1 z - \o_{1/2}t) \bm{e_z},
\end{align} 

La force physique $f_i(x_i;t^n)$ en rouge, la vitesse $u_i(x_i;t^n)$ en bleu et la pression $p(x_i;t^n)$ en vert pour différents instants allant de $t=0.012s$ à $t=0.26s$ sont tracés, avec $i=y$ et $i=z$ respectivement pour le troisième et pour le quatrième cas. On constate bien qu'entre $t_0=0.012s \approx 0s$ et $t^n = 0.26 \approx t_0 + 1/4 s$, le champ de force $f_i$ (rouge) a été translaté d'une distance $L/2$ dans la direction $e_i$. \tb{La translation d'un champ d'onde longitudinale est ainsi validé séparément dans les trois direction de l'espace.}

\begin{figure}
\begin{center}
	\begin{subfigure}[t]{0.4\textwidth}                                                                                                                                   
		\includegraphics[scale=0.23]{../../build/lineout_ADV_010_t1.png}
		\caption{ $\bm{t=s}$}
		\label{fig : 010_t1}
	\end{subfigure}\hfill
	\begin{subfigure}[t]{0.4\textwidth}
		\includegraphics[scale=0.23]{../../build/lineout_ADV_010_t6.png}
		\caption{ $\bm{t=s}$}
		\label{fig : 010_t6}
	\end{subfigure}
\\
	\begin{subfigure}[t]{0.4\textwidth}
		\includegraphics[scale=0.23]{../../build/lineout_ADV_010_t11.png}
		\caption{ $\bm{t=s}$}
		\label{fig : 010_t11}
	\end{subfigure}\hfill
	\begin{subfigure}[t]{0.4\textwidth}
		\includegraphics[scale=0.23]{../../build/lineout_ADV_010_t16.png}
		\caption{ $\bm{t=s}$}
		\label{fig : 010_t16}
	\end{subfigure}
\\
	\begin{subfigure}[t]{0.4\textwidth}
		\includegraphics[scale=0.23]{../../build/lineout_ADV_010_t21.png}
		\caption{ $\bm{t=s}$}
		\label{fig : 010_t21}
	\end{subfigure}	
\end{center}
\caption{Évolution le long de $(x=0,y,z=0)$ de : la force $f_y$ (rouge), la vitesse $u_y$ (bleu), la pression $p$ (vert).}
\label{fig : 010}
\end{figure}

\begin{figure}
\begin{center}
	\begin{subfigure}[t]{0.4\textwidth}                                                                                                                                   
		\includegraphics[scale=0.23]{../../build/lineout_ADV_001_t1.png}
		\caption{ $\bm{t=s}$}
		\label{fig : 001_t1}
	\end{subfigure}\hfill
	\begin{subfigure}[t]{0.4\textwidth}
		\includegraphics[scale=0.23]{../../build/lineout_ADV_001_t6.png}
		\caption{ $\bm{t=s}$}
		\label{fig : 001_t6}
	\end{subfigure}
\\
	\begin{subfigure}[t]{0.4\textwidth}
		\includegraphics[scale=0.23]{../../build/lineout_ADV_001_t11.png}
		\caption{ $\bm{t=s}$}
		\label{fig : 001_t11}
	\end{subfigure}\hfill
	\begin{subfigure}[t]{0.4\textwidth}
		\includegraphics[scale=0.23]{../../build/lineout_ADV_001_t16.png}
		\caption{ $\bm{t=s}$}
		\label{fig : 001_t16}
	\end{subfigure}
\\
	\begin{subfigure}[t]{0.4\textwidth}
		\includegraphics[scale=0.23]{../../build/lineout_ADV_001_t21.png}
		\caption{ $\bm{t=s}$}
		\label{fig : 001_t21}
	\end{subfigure}	
\end{center}
\caption{Évolution le long de $(x,y,z=0)$ de : la force $f_z$ (rouge), la vitesse $u_z$ (bleu), la pression $p$ (vert).}
\label{fig : 001}
\end{figure}


L'observation du champ de vitesse seul montre qu'il n'est pas strictement nul. Son amplitude est de l'ordre de $|u_x|_{max}=10^{-5} m.s^{-1}$. Son évolution est de la forme $u_y = cos(\k_1 \cdot x)cos(\D x \cdot x)$. Cette évolution est due à l'absence de phénomènes diffusifs dans l'écoulement, ce qui laisse un courant parasite s'établir. La résolution de la pression, est satisfaisante sur deux aspects. Le premier est que l'amplitude de la pression est de l'ordre de grandeur attendu : $\frac{2\r}{\k_1}\approx 1.5$, le second est l'évolution du champ qui est bien déphasée de $\pi/4$ par rapport au champ $f$. Ce qui est sujet à discussion est "l'offset" fluctuant qui est observé pour ce champ. En effet, la pression semble évoluer comme :

\begin{align}
p(\bm{x},t) = 2 \frac{\r}{\k_1} (sin(\k_1 \frac{L}{2})- sin(\k_1 x - \o_{1/2} t) )
\end{align}

Cette correction a pour effet d'assurer $p(x=x_{min}) = p(x=x_{max}) = 0$.

La figure \ref{fig : 101_alongX} montre les évolutions des grandeurs pour le cinquième et dernier cas :
\begin{align}
&\bm{f}(x;t)  = \sqrt{2} cos(\k_1 (x-z) - \o_{1/2} t) (\bm{e_x - e_z}) \\
\end{align} 

La force physique $f_x(x,0,0;t_i)$ en trait plein rouge, la force physique $f_z(x,0,0;t_i)$ en symboles "+", la norme de la vitesse $|u|(x,0,0;t_i)$ en bleu et la pression $p(x,0,0;t_i)$ en vert pour différents instants allant de $t=0.012s$ à $t=0.26s$ sont tracés. On constate bien qu'entre $t_0=0.012s \approx 0s$ et $t_i = 0.26 \approx t_0 + 1/4 s$, le champ de force $f_y$ (rouge) a été translaté d'une distance $L/2$ dans la direction $x$, ce qui est attendu. \tb{La translation du champ dans la direction $x$ est validée pour un champ d'onde transversale ne se déplaçant pas selon $x$ uniquement.}

Les évolutions de la pression $p$ en vert et de la vitesse e bleu sont sujettes aux mêmes remarques que pour les trois cas cas précédents. Elles ne sont pas recopiées ici.


\begin{figure}
\begin{center}
	\begin{subfigure}[t]{0.4\textwidth}                                                                                                                                   
		\includegraphics[scale=0.23]{\orig/lineout_alongX_ADV_101_t1.png}
		\caption{ $\bm{t=s}$}
		\label{fig : 101_t1}
	\end{subfigure}\hfill
	\begin{subfigure}[t]{0.4\textwidth}
		\includegraphics[scale=0.23]{\orig/lineout_alongX_ADV_101_t12.png}
		\caption{ $\bm{t=s}$}
		\label{fig : 101_t6}
	\end{subfigure}
\\
	\begin{subfigure}[t]{0.4\textwidth}
		\includegraphics[scale=0.23]{\orig/lineout_alongX_ADV_101_t24.png}
		\caption{ $\bm{t=s}$}
		\label{fig : 101_t11}
	\end{subfigure}\hfill
	\begin{subfigure}[t]{0.4\textwidth}
		\includegraphics[scale=0.23]{\orig/lineout_alongX_ADV_101_t36.png}
		\caption{ $\bm{t=s}$}
		\label{fig : 101_t16}
	\end{subfigure}
\end{center}
\caption{Évolution le long de $(x,y=0,z=0)$ de : la force $f_x$ (rouge), la vitesse $u_x$ (bleu), la pression $p$ (vert).}
\label{fig : 101_alongX}
\end{figure}


La figure \ref{fig : 101} trace l'évolution des mêmes grandeurs que précédemment, à la différence que le tracé est effectué le long de $\D : (x=-z, y=0)$. Le motif parcourt $L/2$ en un quart de seconde une fois de plus. \tb{La translation du champ dans la direction $x-z$ est validée pour un champ d'onde transversale se déplaçant selon $x-z$.}



\begin{figure}
\begin{center}
	\begin{subfigure}[t]{0.4\textwidth}                                                                                                                                   
		\includegraphics[scale=0.23]{\orig/lineout_ADV_101_t1.png}
		\caption{ $\bm{t=s}$}
		\label{fig : 101_t1}
	\end{subfigure}\hfill
	\begin{subfigure}[t]{0.4\textwidth}
		\includegraphics[scale=0.23]{\orig/lineout_ADV_101_t12.png}
		\caption{ $\bm{t=s}$}
		\label{fig : 101_t6}
	\end{subfigure}
\\
	\begin{subfigure}[t]{0.4\textwidth}
		\includegraphics[scale=0.23]{\orig/lineout_ADV_101_t24.png}
		\caption{ $\bm{t=s}$}
		\label{fig : 101_t11}
	\end{subfigure}\hfill
	\begin{subfigure}[t]{0.4\textwidth}
		\includegraphics[scale=0.23]{\orig/lineout_ADV_101_t36.png}
		\caption{ $\bm{t=s}$}
		\label{fig : 101_t16}
	\end{subfigure}
\end{center}
\caption{Évolution le long de $\D : (x=-z, y=0)$ de : la force $f_x$ (trait plein rouge), $f_z$ (symbole "+"), la vitesse $u_x$ (bleu), la pression $p$ (vert).}
\label{fig : 101}
\end{figure}

\section{Conclusions}
La translation forçée d'un champ d'onde transversale et d'un champ d'onde longitudinale est validée. La résolution physique de l'écoulement pour une onde lo    ngitudinale n'est pas exact. La résolution physique de l'écoulement pour une onde transversale est satisfaisant.
%\section{Pour les fiches de validation}
%%%%%%%%%% COMMANDES UTILES %%%%%%%%%%%
%% \let
% NOTATIONS
\let\ol\overline
\let\ul\underline
\let\la\langle
\let\ra\rangle
\let\tm\times
% ALPHABET GREC
\let\a\alpha
\let\b\beta
\let\d\delta
\let\D\Delta
\let\e\epsilon
\let\k\kappa
\let\g\gamma
\let\G\Gamma
\let\l\lambda
\let\L\Lambda
\let\n\eta
\let\o\omega
\let\p\partial
\let\r\rho
\let\s\sigma
\let\t\tau

\newcommand{\Cpx}{\mathbb{C}}
\newcommand{\Ree}{\mathbb{R}}
\newcommand{\Rey}{\mathit{Re}}
\newcommand{\rei}{\mathit{Re_{b;isolée}}}
\newcommand{\ree}{\mathit{Re_{b;essaim}}}
\newcommand{\WIF}{\mathit{WIF}}
\newcommand{\hf}{\hat{f}}
\newcommand\piL[1]{\frac{{#1} \pi}{L}}

%% \newcommand
% EDITION DE TEXTE
\newcommand\tab[1][1cm]{\hspace*{#1}}
\newcommand\tb[1]{\textbf{#1}}
\newcommand\ti[1]{\textit{#1}}
\newcommand\mc[1]{\mathcal{#1}}
\newcommand\mi[1]{\mathit{#1}}
\newcommand\fnm[1]{$^{#1}$}
% GRANDEURS CONSTRUITES
\newcommand\tc{\tilde{\chi}}
\newcommand\tq{\tilde{q}}
\newcommand\tu{\tilde{u}}
\newcommand\tv{\tilde{v}}
\newcommand\tru{\tilde{\rho u}}
\newcommand\sq{\sigma(q)_{t,sonde}}
% INTEGRALE ET SOMME
\newcommand\TF{\int\limits}
\newcommand\TFF{\iint\limits}
\newcommand\TFFF{\iiint\limits}
\newcommand\som{\sum\limits}
% VECTEURS DÉTAILLÉS
\newcommand\vect[3]{ \left(\begin{array}{c} {#1}\\ {#2} \\{#3}\end{array}\right) }
\newcommand\Vect[1]{ \vect{#1 _1}{#1 _2}{#1 _3} }
\newcommand\VectUn[1]{ \vect{#1 _{10}}{#1 _2}{#1 _3} }
\newcommand\VectUnDeux[1]{ \vect{#1 _{10}}{#1 _{20}}{#1 _3} }
\newcommand\VectZero[1]{ \vect{#1 _{10}}{#1 _{20}}{#1 _{30}} }
\newcommand\VectUnZeroUn[1]{ \vect{#1 _{10}}{0}{#1 _{10}} }
\newcommand\VectUnUnZero[1]{ \vect{#1 _{10}}{#1 _{10}}{0} }
\newcommand\VectUnUnUn[1]{ \vect{#1 _{10}}{#1 _{10}}{#1 _{10}} }
%%%%%%%%%%% COMMANDES UTILES %%%%%%%%%%%%%%


\begin{align}
| f(x;t) | = 2 cos(\k x - \o t)
\end{align}

Avec $\k = 2 n \pi / L$ pour $n$ périodes sur le domaine et $\o = 2 \pi / q$  pour translater $F$ d'une distance $L$ en $q$ secondes.

{\color{red} Sans convection, sans diffusion, sans gravité, avec pression}
\begin{align}
 &\bm{f}(x;t)  = 2 cos(\k x - \o t) \bm{e_x} \\
 &\p_t \bm{u} = - \frac{1}{\r} \bm{\nabla} p + \bm{f} \\
  &\bm{ \nabla \cdot u } = 0 \\
 &\Rightarrow \p_t u_x = - \frac{1}{\r} \p_x p + f_x \\
 &\Rightarrow \D p = \r \nabla \cdot f \\
 &\Rightarrow \p_{xx} p = - 2 \k \r sin(\k x - \o t) \\
 &\Rightarrow p(x;t) = 2 \frac{\r}{\k} sin(\k x - \o t) \text{ et } u(x;t) = 0 
\end{align}

\tb{Application numérique} : $\r = 1000$; $\k = \frac{4 \pi}{L}$; $L = 4 \cdot 10^{-3}$ : $\frac{\r}{\k} = 0.3183098861837907$.
 
\begin{align}
&\bm{f}(x;t) = 2 cos(\k x - \o t) \bm{e_y} \\
 &\p_t \bm{u} = - \frac{1}{\r} \bm{\nabla} p + \bm{f} \\
 &\bm{ \nabla \cdot u } = 0 \\
 &\Rightarrow  \p_t u_y = - \frac{1}{\r} \p_y p + f_y \\
 &\Rightarrow \p_t u_y = 2 cos(\k x - \o t) \\
 &\Rightarrow u_y = - 2 \frac{1}{\o} sin(\k x - \o t) \text{ et } p(x;t) = 0 \\
 &u_y = - 2 \frac{1}{\o} sin(\k x - \o t)  \cdot sin(2 \pi t)\\
 \mc{F} \tru
\end{align}

\tb{Application numérique} : $\o = 4 \pi$ : $\frac{1}{\o} = 0.07957747154594767$.
 

\newpage
\printbibliography


\end{document}