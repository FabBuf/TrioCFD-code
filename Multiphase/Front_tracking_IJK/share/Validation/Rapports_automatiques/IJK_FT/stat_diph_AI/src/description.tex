Les solutions analytiques sont obtenues par integration dans un repere spherique, sur une tranche d'epaisseur dz centree sur z, ou z est variable. 

$\phi \in [0,\pi]$ est l'angle signe entre l'axe (Oz) et le vecteur OP.

$\theta \in [0,2\pi]$ est l'angle signe entre l'axe (Ox) et le vecteur OP' ou P' est la projection de P sur le plan (Oxy).

Le changement de coordonnees s'ecrit:
\begin{subequations}
\begin{align}
x&=\rho \sin \phi \cos \theta \\
y&=\rho \sin \phi \sin \theta \\
z&=\rho \cos \phi
\end{align}
\end{subequations}

Le vecteur normale a l'interface s'ecrit:
\begin{subequations}
\begin{align}
nx&=\sin \phi \cos \theta\\
ny&=\sin \phi \sin \theta\\
nz&=\cos \phi 
\end{align}
\end{subequations}

L'element de surface infinitesimal pour $\rho$ constant est:
$d^2s = \rho^2 sin \phi d\phi d\theta$.

Sur la sphere, $\rho = R=0.004$. La courbure est definie par $\kappa = -2/R$.

On definit les grandeurs moyennes:
$$\overline{\phi} = \int_v \phi \delta^i dv $$
ou $v=L_x L_y \Delta z$ est le volume d'une tranche d'epaisseur $\Delta z$

\subsection{CAS 1 : SPHERE }

Dans le cas de la sphere, on obtient:
\begin{subequations}
\begin{align}
  a_i    &= \int_v \delta^i dv = \frac{2\pi R}{L_xL_y} \\
  a_iN_x &= \int_v N_x \delta^i dv = 0 \\
  a_iN_y &= 0 \\
  a_iN_z &= \frac{2\pi}{L_xL_y} z 
\end{align}
\end{subequations}

Comme $\kappa = -2/R$ est constante pour le sphere, les int�grales suivante se d�duisent rapidement:
\begin{subequations}
\begin{align}
  \kappa a_iN_x &= \int_v N_x \delta^i dv = 0  \\
  \kappa a_iN_y &= 0  \\
  \kappa a_iN_z &= -\frac{2}{R} \frac{2\pi}{L_xL_y} z
\end{align}
\end{subequations}

\subsection{CAS 2 : HEMISPHERE }

L'hemisphere est composee de la moitie superieure de la sphere et d'un disque horizontal de rayon $R$. 
A la jonction entre ces 2 elements, grace au lissage, on cree une petite zone de transition dont l'aire est
faible, mais dont la courbure est grande. Pour evaluer analytiquement les grandeurs, on suppose cette zone de
transition negligeable. Cette hypothese est acceptable, sauf dans le cas des termes proportionnels a la 
courbure. Pour l'integration d'une tranche de plan, les formules de la page "segment circulaire" de wiki 
permette de determiner l'aire de l'intersection de la partie plane par la tranche de prise de moyenne ($ai_p$)
alors que la partie sph�rique ($ai_{hs}$) est donnee par la moitie de la valeur pour la sphere ci-dessus :

\begin{subequations}
\begin{align}
ai_p &= \frac{1}{L_xL_y \Delta z} \frac{R^2}{2} \left[2\theta - \sin 2\theta \right]_{\theta_p}^{\theta_m} \\
ai_{hs} &= \frac{\pi R}{L_x L_y} \\
ai  &= ai_p+ai_{hs}
\end{align}
\end{subequations}
avec $\theta=arccos{z/R}$

Pour la partie plane, la normale est constante $\vec{N} = {-1,0,0}$ donc:

\begin{subequations}
\begin{align}
  (a_iN_x)_p &= -ai_p \\
  (a_iN_y)_p &= 0 \\
  (a_iN_z)_p &= 0
\end{align}
\end{subequations}
 

Pour la partie spherique, la composante selon x s'obtient par integration ou plus simplement en remarquant que le
produit par $N_x$ correspond a une projection sur le plan (Oyz), ce qui conduit a : 
$$(a_iN_x)_{hs} = ai_p $$

Les autres composantes se deduisent aisement du cas precedent. En raison de la symetrie, elle valent la moitie du cas de la 
sphere:

\begin{subequations}
\begin{align}
  (a_iN_y)_{hs} &= 0 \\
  (a_iN_z)_{hs} &= \frac{\pi}{L_xL_y} z
\end{align}
\end{subequations}


En ponderant par une courbure nulle pour la partie plane et une courbure $\kappa = -2/R$ pour la partie spherique, on obtient 
la solution analytique approchee 
\begin{subequations}
\begin{align}
  \kappa a_iN_x &= -2/R ai_p \\
  \kappa a_iN_y &= 0  \\
  \kappa a_iN_z &= -\frac{2}{R} \frac{\pi}{L_xL_y} z
\end{align}
\end{subequations}

qui n'est que partielle car la zone de jonction entre les 2 parties a une courbure inversement proportionnelle a la surface
(cad plus la surface est petite plus la courbure y est grande), ce qui conduit a un terme non negligeable du premier ordre.
Il est difficile a evaluer analytiquement. 

On peut simplement dire que l'erreur commise est negative pour l'evaluation de kaiNx et qu'elle est du meme signe que z pour kaiNz


Pour valeurs numeriques, on prend : 

Lx=0.018
Ly=0.015
Lz=0.01
N=80
R=0.004
