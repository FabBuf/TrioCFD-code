

Les figures \ref{fig : 010} et \ref{fig : 001} montrent respectivement les évolutions des grandeurs pour le troisième et le quatrième cas :
\begin{align*}
\gf(\bx;t) = 2cos(\k_1 y - \o_{1/2}t) \bm{e_y}, \\ 
\gf(\bx;t) = 2cos(\k_1 z - \o_{1/2}t) \bm{e_z},
\end{align*} 

La force physique $f_i(x_i;t^n)$ en rouge, la vitesse $u_i(x_i;t^n)$ en bleu et la pression $p(x_i;t^n)$ en vert pour différents instants allant de $t=0.012s$ à $t=0.26s$ sont tracés, avec $i=y$ et $i=z$ respectivement pour le troisième et pour le quatrième cas. On constate bien qu'entre $t_0$ et $t^n $, le champ de force $f_i$ (rouge) a été translaté d'une distance $L/2$ dans la direction $e_i$. \tb{La translation d'un champ d'onde longitudinale est ainsi validé séparément dans les trois direction de l'espace.} Une fois encore, le champ vérifie $f_i(x_{i;max};t)=0$

\begin{figure}
\begin{center}
	\begin{subfigure}[t]{0.4\textwidth}                                                                                                                                   
		\includegraphics[scale=0.23]{\orig/lineout_ADV_010_t1.png}
		\caption{ $\bm{t=s}$}
		\label{fig : 010_t1}
	\end{subfigure}\hfill
	\begin{subfigure}[t]{0.4\textwidth}
		\includegraphics[scale=0.23]{\orig/lineout_ADV_010_t6.png}
		\caption{ $\bm{t=s}$}
		\label{fig : 010_t6}
	\end{subfigure}
\\
	\begin{subfigure}[t]{0.4\textwidth}
		\includegraphics[scale=0.23]{\orig/lineout_ADV_010_t12.png}
		\caption{ $\bm{t=s}$}
		\label{fig : 010_t11}
	\end{subfigure}\hfill
	\begin{subfigure}[t]{0.4\textwidth}
		\includegraphics[scale=0.23]{\orig/lineout_ADV_010_t18.png}
		\caption{ $\bm{t=s}$}
		\label{fig : 010_t16}
	\end{subfigure}
\end{center}
\caption{Évolution le long de $(x=0,y,z=0)$ de : la force $f_y$ (rouge), la vitesse $u_y$ (bleu), la pression $p$ (vert).}
\label{fig : 010}
\end{figure}

\begin{figure}
\begin{center}
	\begin{subfigure}[t]{0.4\textwidth}                                                                                                                                   
		\includegraphics[scale=0.23]{\orig/lineout_ADV_001_t1.png}
		\caption{ $\bm{t=s}$}
		\label{fig : 001_t1}
	\end{subfigure}\hfill
	\begin{subfigure}[t]{0.4\textwidth}
		\includegraphics[scale=0.23]{\orig/lineout_ADV_001_t6.png}
		\caption{ $\bm{t=s}$}
		\label{fig : 001_t6}
	\end{subfigure}
\\
	\begin{subfigure}[t]{0.4\textwidth}
		\includegraphics[scale=0.23]{\orig/lineout_ADV_001_t12.png}
		\caption{ $\bm{t=s}$}
		\label{fig : 001_t11}
	\end{subfigure}\hfill
	\begin{subfigure}[t]{0.4\textwidth}
		\includegraphics[scale=0.23]{\orig/lineout_ADV_001_t18.png}
		\caption{ $\bm{t=s}$}
		\label{fig : 001_t16}
	\end{subfigure}
\end{center}
\caption{Évolution le long de $(x,y,z=0)$ de : la force $f_z$ (rouge), la vitesse $u_z$ (bleu), la pression $p$ (vert).}
\label{fig : 001}
\end{figure}


L'observation du champ de vitesse seul (non reproduit) montre le même type d'évolution que pour le cas précédent. La résolution de la pression, est elle aussi satisfaisante en ce qui concerne l'amplitude et le déphasage. "l'offset" fluctuant est aussi observé pour les troisième et quatrième cas.