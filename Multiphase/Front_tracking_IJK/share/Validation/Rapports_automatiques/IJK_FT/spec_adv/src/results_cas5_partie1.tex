La figure \ref{fig : 101_alongX} montre les évolutions des grandeurs pour le cinquième et dernier cas :
\begin{align}
&\bm{f}(x;t)  = \sqrt{2} cos(\k_1 (x-z) - \o_{1/2} t) (\bm{e_x - e_z}) \\
\end{align} 

La force physique $f_x(x,0,0;t_i)$ en trait plein rouge, la force physique $f_z(x,0,0;t_i)$ en symboles "+", la norme de la vitesse $|u|(x,0,0;t_i)$ en bleu et la pression $p(x,0,0;t_i)$ en vert pour différents instants allant de $t=0.012s$ à $t=0.26s$ sont tracés. On constate bien qu'entre $t_0=0.012s \approx 0s$ et $t_i = 0.26 \approx t_0 + 1/4 s$, le champ de force $f_y$ (rouge) a été translaté d'une distance $L/2$ dans la direction $x$, ce qui est attendu. \tb{La translation du champ dans la direction $x$ est validée pour un champ d'onde transversale ne se déplaçant pas selon $x$ uniquement.}

Les évolutions de la pression $p$ en vert et de la vitesse e bleu sont sujettes aux mêmes remarques que pour les trois cas cas précédents. Elles ne sont pas recopiées ici.


\begin{figure}
\begin{center}
	\begin{subfigure}[t]{0.4\textwidth}                                                                                                                                   
		\includegraphics[scale=0.23]{\orig/lineout_alongX_ADV_101_t1.png}
		\caption{ $\bm{t=s}$}
		\label{fig : 101_t1}
	\end{subfigure}\hfill
	\begin{subfigure}[t]{0.4\textwidth}
		\includegraphics[scale=0.23]{\orig/lineout_alongX_ADV_101_t12.png}
		\caption{ $\bm{t=s}$}
		\label{fig : 101_t6}
	\end{subfigure}
\\
	\begin{subfigure}[t]{0.4\textwidth}
		\includegraphics[scale=0.23]{\orig/lineout_alongX_ADV_101_t24.png}
		\caption{ $\bm{t=s}$}
		\label{fig : 101_t11}
	\end{subfigure}\hfill
	\begin{subfigure}[t]{0.4\textwidth}
		\includegraphics[scale=0.23]{\orig/lineout_alongX_ADV_101_t36.png}
		\caption{ $\bm{t=s}$}
		\label{fig : 101_t16}
	\end{subfigure}
\end{center}
\caption{Évolution le long de $(x,y=0,z=0)$ de : la force $f_x$ (rouge), la vitesse $u_x$ (bleu), la pression $p$ (vert).}
\label{fig : 101_alongX}
\end{figure}


La figure \ref{fig : 101} trace l'évolution des mêmes grandeurs que précédemment, à la différence que le tracé est effectué le long de $\D : (x=-z, y=0)$. Le motif parcourt $L/2$ en un quart de seconde une fois de plus. \tb{La translation du champ dans la direction $x-z$ est validée pour un champ d'onde transversale se déplaçant selon $x-z$.}



\begin{figure}
\begin{center}
	\begin{subfigure}[t]{0.4\textwidth}                                                                                                                                   
		\includegraphics[scale=0.23]{\orig/lineout_ADV_101_t1.png}
		\caption{ $\bm{t=s}$}
		\label{fig : 101_t1}
	\end{subfigure}\hfill
	\begin{subfigure}[t]{0.4\textwidth}
		\includegraphics[scale=0.23]{\orig/lineout_ADV_101_t12.png}
		\caption{ $\bm{t=s}$}
		\label{fig : 101_t6}
	\end{subfigure}
\\
	\begin{subfigure}[t]{0.4\textwidth}
		\includegraphics[scale=0.23]{\orig/lineout_ADV_101_t24.png}
		\caption{ $\bm{t=s}$}
		\label{fig : 101_t11}
	\end{subfigure}\hfill
	\begin{subfigure}[t]{0.4\textwidth}
		\includegraphics[scale=0.23]{\orig/lineout_ADV_101_t36.png}
		\caption{ $\bm{t=s}$}
		\label{fig : 101_t16}
	\end{subfigure}
\end{center}
\caption{Évolution le long de $\D : (x=-z, y=0)$ de : la force $f_x$ (trait plein rouge), $f_z$ (symbole "+"), la vitesse $u_x$ (bleu), la pression $p$ (vert).}
\label{fig : 101}
\end{figure}
