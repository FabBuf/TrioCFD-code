Le premier test correspond à résoudre les équations : 

\begin{align*}
&\bm{f} = 2 cos(\k_1 x - \o_{1/2} t) \bm{e_y} \\
&\p_t \bm{u} = - \frac{1}{\r} \bm{\nabla} p + \bm{f} \\                
&\bm{ \nabla \cdot u } = 0 \\
\end{align*}

$f$ ne dépend que de $x$ et de $t$. $f$ est la seule cause de mouvement dans ce problème. Ainsi les grandeurs du problème sont indépendantes de $y$ et de $z$ ($\p_y = \p_z = 0$ et $p=p(x,t)$ notamment). L'équation de Poisson pour la pression est : $\D p = 0$. Par conséquent $\p_x p  = C(t)$. Le champ de pression respecte la condition de périodicité si $C(t) = 0$. Le champ de pression est donc constant et uniforme.
La composante $y$ de l'équation de conservation de quantité de mouvement donne : 

\begin{align*}
&\p_t u_y = f_y = 2 cos(\k_1 x - \o_{1/2} t)\\ 
\end{align*}

Ce premier test a pour solution analytique :
\begin{mdframed}
\begin{align*} 
&\bm{u}(\bm{x};t) = - \frac{2}{\o_{1/2}} sin(\k_1 x - \o_{1/2} t)  \\
&p(\bm{x};t)=p.   
\end{align*}
\end{mdframed}
\ti{Onde de vitesse}

Le deuxième test correspond à résoudre les équations : 

\begin{align*}
&\bm{f} = 2 cos(\k_1 x - \o_{1/2} t) \bm{e_x} \\
&\p_t \bm{u} = - \frac{1}{\r} \bm{\nabla} p + \bm{f} \\                
&\bm{ \nabla \cdot u } = 0 \\
\end{align*}

De même que pour le test précédent, les grandeurs du problème sont indépendantes de $y$ et de $z$. L’équation de Poisson pour la pression est $\D p = \r \na \cdot f$ et se développe :  

\begin{align*}
&\p_{xx} p= - 2 \r \k_1 sin(\k_1 x - \o_{1/2} t) \\
&p(x,t) = 2 \frac{\r}{\k_1} sin(\k_1 x - \o_{1/2} t)
\end{align*}

Ce deuxième test a donc pour solution analytique (\ti{ce cas a un peu plus de justificaitons a dooner. En relisant je me is que c'est mieux de les donner qd même}): 

\begin{mdframed}
\begin{align*} 
&p(\bm{x},t) = - 2 \frac{\r}{\k_1} sin(\k_1 x - \o_{1/2} t)  \\
&\bm{u} = \bm{0}.   
\end{align*}
\end{mdframed}
\ti{Onde de pression}

On montre facilement que les tests suivants ont respectivement pour solutions analytiques : 

Troisième test : 
\begin{mdframed}
\begin{align*} 
&p(\bm{x},t) = - 2 \frac{\r}{\k_1} sin(\k_1 y - \o_{1/2} t)  \\
&\bm{u} = \bm{0}.   
\end{align*}
\end{mdframed}
\ti{Onde de pression}

Quatrième test : 
\begin{mdframed}
\begin{align*} 
&p(\bm{x},t) = - 2 \frac{\r}{\k_1} sin(\k_1 z - \o_{1/2} t)  \\
&\bm{u} = \bm{0}.   
\end{align*}
\end{mdframed}
\ti{Onde de pression}

Cinquième test : 
\begin{mdframed}
\begin{align*} 
&p(\bm{x},t) = - \frac{2}{\sqrt{2}} \frac{\r}{\k_1} sin(\k_1 (x-z) - \o_{1/2} t) \\
&\bm{u} = \bm{0}.   
\end{align*}
\end{mdframed}
\ti{Onde de pression}

\tb{Applications numériques} : Le valeurs numériques des paramètres $\k_1$; $\o_{1/2}$ ainsi que les normes des solutions analytiques des tests sont regroupées dans le tableau \ref{tab : AN}

\begin{table}[h]
\begin{center}
\begin{tabular}{c|c|c|c|c|c}
 N  & $\k_1(rad.m^{-1})$        & $\o_{1/2}(rad.s^{-1})$ & $2/ \o_{1/2}$ (sol. 1) & $2 \r / \k_1$ (sol. 2,3,4) & $\sqrt(2) \r / \k_1$ (sol. 5)  \\ \hline
 40 & 1571                      &   12.57                & 0.1592                &   1.491                    & 1.054
\end{tabular}
\end{center}
\caption{Nombre de mailles par direction et applications numériques}
\label{tab : AN}
\end{table}

