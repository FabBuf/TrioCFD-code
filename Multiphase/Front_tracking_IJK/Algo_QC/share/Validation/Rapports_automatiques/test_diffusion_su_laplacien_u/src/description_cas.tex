Ce cas test correspond \`a la diffusion d'un produit de sinus sur une des composantes de la vitesse. Il a pour but le test du mod\`ele S U Laplacien U pour le terme sous-maille $\partial_j\left(\overline{U_i U_j} - \overline{U}_i\overline{U}_j\right)$ de l'\'equation de conservation de la quantit\'e de mouvement dans la formulation 'Favre'.

\section{Solution analytique}

Soit
\begin{equation}
\tau_{ij} = \overline{U_i U_j} - \overline{U}_i\overline{U}_j
\end{equation}
Le mod\`ele S U Laplacien U correspond au mod\`ele
\begin{equation}
\tau_{ij} = - \frac{\Delta_k^2}{24} \left(U_j \frac{\partial^2 U_i}{\partial x_k^2} + U_i \frac{\partial^2 U_j}{\partial x_k^2}\right)
\end{equation}

On note
\begin{equation}
D_i = \frac{1}{\rho} \frac{\partial \rho \tau_{ij}}{\partial x_j}.
\end{equation}
Dans la formulation 'Favre', le mod\`ele est impl\'ement\'e comme
\begin{equation}
\frac{\partial U_{i}}{\partial t} = NS^* + D_i
\end{equation}

Notons par ailleurs, $V_0 = 0.001$ et $\tau=2\pi$.

On va donn\'e les solutions analytiques pour $D_i$ selon diff\'erents cas.
Lors de la validation, on notera {\textsf simu\_dv\_i} le r\'esultat de la
simulation pour $D_i$, et {\textsf ana\_dv\_i} la solution analytique.
La diff\'erence entre les deux sera not\'e \textsf{error\_dv\_i}. Si tout se passe bien, les r\'esultats de la simulation sont proches de la solution analytique et l'erreur est tr\'es faible devant la valeur totale.



\subsection{Cas VX\_DIRX}

La condition initiale sur la vitesse est :

\begin{align*}
V_x ={}& 2 V_0 \cos(\tau x) (\cos(2 \tau z) - 1) \\
V_y ={}& 0 \\
V_z ={}& 0
\end{align*}

\subsection{Cas VX\_DIRY}

La condition initiale sur la vitesse est :

\begin{align*}
V_x ={}& 2 V_0 \cos(\tau y) (\cos(2 \tau z) - 1) \\
V_y ={}& 0 \\
V_z ={}& 0
\end{align*}

\subsection{Cas VY\_DIRX}

La condition initiale sur la vitesse est :

\begin{align*}
V_x ={}& 0 \\
V_y ={}& 2 V_0 \cos(\tau x) (\cos(2 \tau z) - 1) \\
V_z ={}& 0
\end{align*}

\subsection{Cas VY\_DIRY}

La condition initiale sur la vitesse est :

\begin{align*}
V_x ={}& 0 \\
V_y ={}& 2 V_0 \cos(\tau y) (\cos(2 \tau z) - 1) \\
V_z ={}& 0
\end{align*}

\subsection{Cas VZ\_DIRX}

La condition initiale sur la vitesse est :

\begin{align*}
V_x ={}& 0 \\
V_y ={}& 0 \\
V_z ={}& 2 V_0 \cos(\tau x) (\cos(2 \tau z) - 1)
\end{align*}

\subsection{Cas VZ\_DIRY}

La condition initiale sur la vitesse est :

\begin{align*}
V_x ={}& 0 \\
V_y ={}& 0 \\
V_z ={}& 2 V_0 \cos(\tau y) (\cos(2 \tau z) - 1)
\end{align*}

